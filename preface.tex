\renewcommand{\lastmod}{\ \ }
\renewcommand{\chapterauthors}{\ \ }

\chapter*{Vorwort}

Dies ist das Vorlesungsskript meiner Vorlesung 'Festkörperphysik II'. Sie ist eine Kursvorlesung für  Studierende im dritten Jahr des Bachelorstudiums. Bei der Auswahl und Gewichtung der Themen folgt sie sehr stark dem in Bayreuth Üblichen. Ich danke an dieser Stelle insbesondere Werner Köhler und Anna Köhler, deren Vorlesungsskripte mir eine große Hilfe waren.

Neben dem Skript gibt es zu jedem Kapitel  insgesamt circa eine Stunde 'Vorlesung' auf Video\sidenote{\href{https://mms.uni-bayreuth.de/Panopto/Pages/Sessions/List.aspx?folderID=514f4be0-2223-4111-831f-b06800ede4ed}{mms.uni-bayreuth.de}}, in der ich mündlich durch den Text führe und dabei an den Rand kritzle.
Ich habe den Eindruck, dass es mir beim Sprechen leichter fällt, die Dinge in einen Zusammenhang zu bringen als beim Schreiben, da ich mich traue, schlampiger zu sein. Zur Vorbereitung gab es dann noch ein online multiple-choice Quiz, sowie die Möglichkeit, jederzeit anonym Fragen zu stellen.\sidenote{\href{http://frag.jetzt}{frag.jetzt}}  Im Plenum  besprechen wir offene Fragen und diskutierten Aufgaben ähnlich zu Eric Mazurs 
ConcepTests.\sidenote{\href{https://mazur.harvard.edu/research-areas/peer-instruction}{mazur.harvard.edu}}  Schließlich gibt es die in der Physik üblichen Übungszettel und Kleingruppen-Übungen. Manche Übungsaufgaben und Beispiele benutzen Julia\sidenote{\href{https://julialang.org/}{julialang.org}}  und Pluto.\sidenote{\href{https://github.com/fonsp/Pluto.jl}{Pluto.jl}} 



Dieses Skript ist 'work in progress', und wahrscheinlich nie wirklich fertig.  Ich danke allen Studierenden des Jahrgangs 2023, die den Text und die Gleichungen aufmerksam gelesen haben, wodurch wir viele Fehler korrigieren konnten. Trotzdem wird es noch welche geben. Wenn Sie Fehler finden, sagen Sie es mir bitte. 
Die aktuellste Version des Vorlesungsskripts finden Sie auf github.\sidenote{\href{https://github.com/MarkusLippitz/Festkoerper-II}{Festkoerper-II}}  Ich habe alles unter eine CC-BY-SA-Lizenz gestellt (siehe Fußzeile). In meinen Worten: Sie können damit machen, was Sie wollen. Wenn Sie Ihre Arbeit der Öffentlichkeit zur Verfügung stellen, erwähnen Sie mich und verwenden Sie eine ähnliche Lizenz. 


Der Text wurde mit der LaTeX-Klasse 'tufte-book' von Bil Kleb, Bill Wood und Kevin Godby\sidenote{\href{https://tufte-latex.github.io/tufte-latex/}{tufte-latex}} gesetzt, die sich der Arbeit von Edward Tufte\sidenote{\href{https://www.edwardtufte.com/}{edwardtufte.com}} annähert. Ich habe viele der Modifikationen angewandt, die von Dirk Eddelbüttel im R-Paket 'tint' eingeführt wurden\sidenote{\href{https://dirk.eddelbuettel.com/code/tint.html}{tint: tint is not Tufte}}. Die Quelle ist vorerst LaTeX, nicht Markdown.




\vspace{2\baselineskip}

Markus Lippitz \\ Bayreuth, 24. August 2023

 
 



