
\documentclass[notoc,nofonts,a4paper,twoside,nobib]{tufte-book}
%\documentclass[nofonts,a4paper,twoside]{book}

\usepackage[ngerman]{babel}
\usepackage{currfile,hyperxmp}

\usepackage{filemod}
   \usepackage{dsfont}
\usepackage[
    type={CC},
    modifier={by-sa},
    version={4.0},
    imagewidth = 17mm,
 ]{doclicense}
  

\usepackage[refsegment=chapter,style=authoryear-comp,natbib=true,url=true,
isbn=false]{biblatex}

\addbibresource{literature.bib}

%rm -rf `biber --cache`


%\AtBeginBibliography{\urlstyle{rm}}

\RequirePackage{fontawesome}

\DeclareFieldFormat{doi}{%
  \ifhyperref
    {\href{http://doi.org/#1}{\small \faExternalLink}}
    {\nolinkurl{#1}}}

\DeclareFieldFormat{url}{%
  \ifhyperref
    {\href{#1}{\small \faExternalLink}}
    {\nolinkurl{#1}}}
    
\renewbibmacro*{doi+eprint+url}{%   
  \iftoggle{bbx:url}     
    {\iffieldundef{doi}{\usebibmacro{url+urldate}}{}}     
    {}%   
  \newunit\newblock   
  \iftoggle{bbx:eprint}     
    {\usebibmacro{eprint}}     
    {}%   
  \newunit\newblock   
  \iftoggle{bbx:doi}     
    {\printfield{doi}}     
    {}}  

\usepackage{amssymb,amsmath}
\usepackage{mathtools,bm}
 
\usepackage{modiagram}
\usepackage{chemformula}
\usepackage{chemfig}
\renewcommand*\printatom[1]{\ensuremath{\mathsf{#1}}}


\usepackage{tikz,tikz-3dplot}

\DeclareUnicodeCharacter{0393}{$\Gamma$} 
\DeclareUnicodeCharacter{03C3}{$\sigma$} 


\newcommand{\inputtikz}[1]{%

 \tikzexternalenable
  \tikzsetnextfilename{#1}%
  \input{#1.tikz}%
  \tikzexternaldisable

}


\usetikzlibrary{math,matrix,fit,positioning,intersections}

\usetikzlibrary{calc}
\usetikzlibrary{arrows.meta} %needed tikz library

\usepackage{standalone}
\usepackage{pgfplots}
 \pgfplotsset{compat=newest}
\usepgfplotslibrary{groupplots}
\usepgfplotslibrary{fillbetween}

\tikzset{>=latex}

\usepackage{tikzorbital}
 \usepackage{tikzsymbols}
\usetikzlibrary{quotes,angles}

\usepackage{currfile,hyperxmp}


\pgfplotsset{
tufte line/.style={
    axis line style={draw opacity=0},
    ytick=\empty,
    axis x line*=bottom,
    x axis line style={
      draw opacity=1,
      gray,
      thick
},
 %   yticklabel=\pgfmathprintnumber{\tick}
  }
  }

\tikzset{
mymat/.style={
    matrix of math nodes,
    left delimiter=|, right delimiter=|,
    align=center,
    column sep=-\pgflinewidth,
}
%,mymats/.style={
%    mymat,
%    nodes={draw,fill=#1}
%} 
 }
 
\newcommand{\myarrow}[5]{\draw[#4](#1.south -| #2)  -- ++(#3 :6mm) node[above,pos=0.55]{$#5$};
} 

\newcommand{\interactLp}[3]{\myarrow{#1-#2-1}{#1.west}{-135}{<-}{#3}} 
\newcommand{\interactLm}[3]{\myarrow{#1-#2-1}{#1.west}{+135}{->}{#3}} 
\newcommand{\interactRp}[3]{\myarrow{#1-#2-2}{#1.east}{ -45}{<-}{#3}} 
\newcommand{\interactRm}[3]{\myarrow{#1-#2-2}{#1.east}{ +45}{->}{#3}}  

\newcommand{\interactout}[2]{\myarrow{#1-1-1}{#1.west}{+135}{->,dashed}{#2}} 


\newcommand{\benzene}[8]{%
\tikzmath{\x1 = #1; \dx1 = 0.5; \dx2 = 0.9; \ps=0.5;}
\tikzmath{\x2 = \x1 + \dx1 ;}
\tikzmath{\x3 = \x2 + \dx2 ;}
\tikzmath{\x4 = \x3 + \dx1 ;}

\tikzmath{\y1 = #2; \dy = 0.5;}
\tikzmath{\y2 = \y1 + \dy ;}
\tikzmath{\y3 = \y2 + \dy ;}

\orbital[pos = {(\x1,\y2)},scale=#3 * \ps]{pz}
\orbital[pos = {(\x2,\y1)},scale=#4 * \ps]{pz}
\orbital[pos = {(\x3,\y1)},scale=#5 * \ps]{pz}
\orbital[pos = {(\x4,\y2)},scale=#6 * \ps]{pz}
\orbital[pos = {(\x3,\y3)},scale=#7 * \ps]{pz}
\orbital[pos = {(\x2,\y3)},scale=#8 * \ps]{pz}

\draw (\x1,\y2) -- (\x2,\y1) -- (\x3,\y1) -- (\x4,\y2) --(\x3,\y3) 
-- (\x2,\y3) -- (\x1,\y2);
}
\usetikzlibrary{external}
\tikzexternalize[prefix=tikz_external/]



\usepackage[T1]{fontenc}
\usepackage[utf8]{inputenc}


\usepackage{graphicx}
\setkeys{Gin}{width=\linewidth,totalheight=\textheight,keepaspectratio}


\usepackage{booktabs}
\usepackage{url}
\usepackage{hyperref}

\usepackage{units}

\usepackage{chemformula}

\usepackage{braket}
\setcounter{secnumdepth}{0}

% citations
%\usepackage{natbib}
%\bibliographystyle{plainnat}
%\setcitestyle{round} 

% pandoc syntax highlighting
%\usepackage{color}
%\usepackage{fancyvrb}



% longtable
\usepackage{longtable,booktabs}
\usepackage{multicol}
\usepackage[normalem]{ulem}

% morefloats
\usepackage{morefloats}

\usepackage{calc}
\usepackage{tcolorbox}




%% -- tint overrides
%% fonts, using roboto (condensed) as default
\usepackage[sfdefault,condensed]{roboto}
%% also nice: \usepackage[default]{lato}

%% colored links, setting 'borrowed' from RJournal.sty with 'Thanks, Achim!'
%\RequirePackage{color}
%\definecolor{link}{rgb}{0.1,0.1,0.8} %% blue with some grey
%\hypersetup{
%  colorlinks,%
%  citecolor=link,%
%  filecolor=link,%
%  linkcolor=link,%
%  urlcolor=link
%}

%% macros
\makeatletter

%% -- tint does not use italics or allcaps in title
\renewcommand{\maketitle}{%     
  \newpage
  \global\@topnum\z@% prevent floats from being placed at the top of the page
  \begingroup
    \setlength{\parindent}{0pt}%
    \setlength{\parskip}{4pt}%
    \let\@@title\@empty
    \let\@@author\@empty
    \let\@@date\@empty
    \ifthenelse{\boolean{@tufte@sfsidenotes}}{%
      %\gdef\@@title{\sffamily\LARGE\allcaps{\@title}\par}%
      %\gdef\@@author{\sffamily\Large\allcaps{\@author}\par}%
      %\gdef\@@date{\sffamily\Large\allcaps{\@date}\par}%
      \gdef\@@title{\begingroup\fontseries{b}\selectfont\LARGE{\@title}\par}%
      \gdef\@@author{\begingroup\fontseries{l}\selectfont\Large{\@author}\par}%
      \gdef\@@date{\begingroup\fontseries{l}\selectfont\Large{\@date}\par}%
    }{%
      %\gdef\@@title{\LARGE\itshape\@title\par}%
      %\gdef\@@author{\Large\itshape\@author\par}%
      %\gdef\@@date{\Large\itshape\@date\par}%
      %\gdef\@@title{\begingroup\fontseries{b}\selectfont\LARGE\@title\par\endgroup}%
      %\gdef\@@author{\begingroup\fontseries{l}\selectfont\Large\@author\par\endgroup}%
      %\gdef\@@date{\begingroup\fontseries{l}\selectfont\Large\@date\par\endgroup}%
      \gdef\@@title{\begingroup\fontseries{b}\fontsize{28}{60}\selectfont\@title\par\endgroup}%
      \gdef\@@author{\begingroup\fontseries{l}\fontsize{16}{20}\selectfont\@author\par\endgroup}%
      \gdef\@@date{\begingroup\fontseries{l}\fontsize{16}{20}\selectfont\@date\par\endgroup}%
    }%
    %\phantom{XXX}%
    \vspace{12pc}%
    \@@title%
    \vspace{4pc}%
    \@@author
    \@@date
  \endgroup
  \thispagestyle{plain}% suppress the running head
  \tuftebreak% add some space before the text begins
  \@afterindentfalse\@afterheading% suppress indentation of the next paragraph
}

%% -- tint does not use italics or allcaps in section/subsection/paragraph
\titleformat{\chapter}%
  [display]% shape
  {\relax\ifthenelse{\NOT\boolean{@tufte@symmetric}}{\begin{fullwidth}}{}}% format applied to label+text
  %{\itshape\huge\thechapter}% label
  {\huge \kapitelname \thechapter}% label
  {0pt}% horizontal separation between label and title body
  %{\huge\rmfamily\itshape}% before the title body
  {\fontseries{b}\selectfont\huge}% before the title body
  [\ifthenelse{\NOT\boolean{@tufte@symmetric}}{\end{fullwidth}}{}]% after the title body

\titleformat{\section}%
  [hang]% shape
  %{\normalfont\Large\itshape}% format applied to label+text
  {\fontseries{b}\selectfont\Large}% format applied to label+text
  {\thesection}% label
  {1em}% horizontal separation between label and title body
  {}% before the title body
  []% after the title body

\titleformat{\subsection}%
  [hang]% shape
  %{\normalfont\large\itshape}% format applied to label+text
  {\fontseries{m}\selectfont\large}% format applied to label+text
  {\thesubsection}% label
  {1em}% horizontal separation between label and title body
  {}% before the title body
  []% after the title body

\titleformat{\paragraph}%
  [runin]% shape
  %{\normalfont\itshape}% format applied to label+text
  {\fontseries{l}\selectfont}% format applied to label+text
  {\theparagraph}% label
  {1em}% horizontal separation between label and title body
  {}% before the title body
  []% after the title body

%% -- tint does not use italics here either
% Formatting for main TOC (printed in front matter)
% {section} [left] {above} {before w/label} {before w/o label} {filler + page} [after]
\ifthenelse{\boolean{@tufte@toc}}{%
  \titlecontents{part}% FIXME
    [0em] % distance from left margin
    %{\vspace{1.5\baselineskip}\begin{fullwidth}\LARGE\rmfamily\itshape} % above (global formatting of entry)
    {\vspace{1.5\baselineskip}\begin{fullwidth}\fontseries{m}\selectfont\LARGE} % above (global formatting of entry)
    {\contentslabel{2em}} % before w/label (label = ``II'')
    {} % before w/o label
    {\rmfamily\upshape\qquad\thecontentspage} % filler + page (leaders and page num)
    [\end{fullwidth}] % after
  \titlecontents{chapter}%
    [0em] % distance from left margin
    %{\vspace{1.5\baselineskip}\begin{fullwidth}\LARGE\rmfamily\itshape} % above (global formatting of entry)
    {\vspace{1.5\baselineskip}\begin{fullwidth}\fontseries{m}\selectfont\LARGE} % above (global formatting of entry)
    {\hspace*{0em}\contentslabel{2em}} % before w/label (label = ``2'')
    {\hspace*{0em}} % before w/o label
    %{\rmfamily\upshape\qquad\thecontentspage} % filler + page (leaders and page num)
    {\upshape\qquad\thecontentspage} % filler + page (leaders and page num)
    [\end{fullwidth}] % after
  \titlecontents{section}% FIXME
    [0em] % distance from left margin
    %{\vspace{0\baselineskip}\begin{fullwidth}\Large\rmfamily\itshape} % above (global formatting of entry)
    {\vspace{0\baselineskip}\begin{fullwidth}\fontseries{m}\selectfont\Large} % above (global formatting of entry)
    {\hspace*{2em}\contentslabel{2em}} % before w/label (label = ``2.6'')
    {\hspace*{2em}} % before w/o label
    %{\rmfamily\upshape\qquad\thecontentspage} % filler + page (leaders and page num)
    {\upshape\qquad\thecontentspage} % filler + page (leaders and page num)
    [\end{fullwidth}] % after
  \titlecontents{subsection}% FIXME
    [0em] % distance from left margin
    %{\vspace{0\baselineskip}\begin{fullwidth}\large\rmfamily\itshape} % above (global formatting of entry)
    {\vspace{0\baselineskip}\begin{fullwidth}\fontseries{m}\selectfont\large} % above (global formatting of entry)
    {\hspace*{4em}\contentslabel{4em}} % before w/label (label = ``2.6.1'')
    {\hspace*{4em}} % before w/o label
    %{\rmfamily\upshape\qquad\thecontentspage} % filler + page (leaders and page num)
    {\upshape\qquad\thecontentspage} % filler + page (leaders and page num)
    [\end{fullwidth}] % after
  \titlecontents{paragraph}% FIXME
    [0em] % distance from left margin
    %{\vspace{0\baselineskip}\begin{fullwidth}\normalsize\rmfamily\itshape} % above (global formatting of entry)
    {\vspace{0\baselineskip}\begin{fullwidth}\fontseries{m}\selectfont\normalsize\rmfamily} % above (global formatting of entry)
    {\hspace*{6em}\contentslabel{2em}} % before w/label (label = ``2.6.0.0.1'')
    {\hspace*{6em}} % before w/o label
    %{\rmfamily\upshape\qquad\thecontentspage} % filler + page (leaders and page num)
    {\upshape\qquad\thecontentspage} % filler + page (leaders and page num)
    [\end{fullwidth}] % after
}{}

% tint: no smallcaps in header 
% The 'fancy' page style is the default style for all pages.
\fancyhf{} % clear header and footer fields
\ifthenelse{\boolean{@tufte@twoside}}
%  {\fancyhead[LE]{\thepage\quad\smallcaps{\newlinetospace{\plaintitle}}}%
%    \fancyhead[RO]{\smallcaps{\newlinetospace{\plainauthor}}\quad\thepage}}
%  {\fancyhead[RE,RO]{\smallcaps{\newlinetospace{\plaintitle}}\quad\thepage}}
  {\fancyhead[LE]{\thepage\quad{\newlinetospace{\plaintitle}}}%
    \fancyhead[RO]{{\newlinetospace{\plaintitle}}\quad\thepage}}%
  {\fancyhead[RE,RO]{{\newlinetospace{\plaintitle}}\quad\thepage}}
  



\makeatother




\renewcommand{\chaptermark}[1]{\markboth{#1}{}}%


\ifthenelse{\boolean{@tufte@twoside}}
  {\fancyhead[LE]{\thepage\quad{\newlinetospace{Festkörperphysik II}}}%
    \fancyhead[RO]{{\newlinetospace{\leftmark}}\quad\thepage}}%
  {\fancyhead[RE,RO]{{\newlinetospace{c}}\quad\thepage}}
  
 
%\makeatletter
\fancypagestyle{mystyle}{%
\fancyhf{}%
\fancyfoot[L]{%
\begin{minipage}{17mm}
\doclicenseImage
\end{minipage}
\begin{minipage}{90mm}
 \footnotesize
 \doclicenseLongText
\end{minipage}%
}% 
%\fancyfoot[L]{\doclicenseThis}% 
}
%\makeatother

\usepackage{etoolbox}
\patchcmd{\chapter}{\thispagestyle{plain}}{\thispagestyle{mystyle}}{}{}



\hypersetup{
 linktocpage,
  colorlinks,
  citecolor=Maroon,
  filecolor=Maroon,
  linkcolor=RoyalBlue,
  urlcolor=RoyalBlue
}


\usepackage[theme=default-plain,charsperline=62]{jlcode}


%default, default-plain, grayscale, grayscale-plain and darkbeamer.





\newcommand{\kapitelname}{Kapitel\ }
\newcommand{\chapterauthors}{Markus Lippitz}
\newcommand{\lastmod}{\Filemodtoday{\currfilepath}}


\newcommand{\addtochapter}{%
\vspace*{-12mm}{
\setlength{\parindent}{0pt}
\chapterauthors  \newline \lastmod
}
\vspace*{12mm}
}

\makeatletter
\let\stdchapter\chapter
\renewcommand*\chapter{%
  \@ifstar{\starchapter}{\@dblarg\nostarchapter}}
\newcommand*\starchapter[1]{\stdchapter*{#1}}
\def\nostarchapter[#1]#2{\stdchapter[{#1}]{#2} \addtochapter}
\makeatother

\makeatletter
  \def\my@tag@font{\scriptsize}
  \def\maketag@@@#1{\hbox{\m@th\normalfont\color{gray}\my@tag@font#1}}
  \let\amsmath@eqref\eqref
  \renewcommand\eqref[1]{{\let\my@tag@font\relax\amsmath@eqref{#1}}}
\makeatother

\newcounter{questions}[chapter]

\newenvironment{questions}{
\subsection{\normalsize Zur Selbstkontrolle}
\begin{enumerate} \small
\setcounter{enumi}{\value{questions}}
}{
\setcounter{questions}{\value{enumi}}
\end{enumerate} 
}

\newtcolorbox{zusammen}{%
  breakable,
  enhanced jigsaw,
 % borderline west={1pt}{0pt}{black},
  sharp corners,
  %boxrule=0pt,
  %frame hidden,
  left=1ex,right=1ex,
  fonttitle={\bfseries},
  coltitle={black},
  title={Zusammenfassung:\ },
  attach title to upper}
  
  
  \newcommand{\pluto}[1]{%
  %
  \edef\cfd{\currfiledir}%
  \StrGobbleRight{\cfd}{1}[\mystring]%
  %
  \sidenote{%
  \begin{tikzpicture}
  [baseline={([yshift=-2pt]current bounding box.center)}]
  \definecolor{redline}{RGB}{201,61,57}
  \definecolor{redfill}{RGB}{214,102,97}
  \definecolor{blueline}{RGB}{148,91,176}
  \definecolor{bluefill}{RGB}{170,125,192}
  \definecolor{greenline}{RGB}{59,151,46}
  \definecolor{greenfill}{RGB}{107,171,91}
  \path[draw=redline,fill=redfill,line width=0.8pt] (0,-5.4pt) circle (4.4pt);
  \path[draw=blueline,fill=bluefill,line width=0.8pt] (0,0) circle (4.4pt);
  \path[draw=greenline,fill=greenfill,line width=0.8pt] (0,5.4pt) circle (4.4pt);
  \end{tikzpicture} \ \ 
  \href{https://raw.githubusercontent.com/MarkusLippitz/Festkoerper-II/main/\mystring/pluto/#1.jl}{download}
  \ \ 
  \href{https://binder.plutojl.org/v0.19.12/open?url=https\%253A\%252F\%252Fraw.githubusercontent.com\%252FMarkusLippitz\%252FFestkoerper-II\%252Fmain\%252F\mystring\%252Fpluto\%252F#1.jl}{run on binder}
  }}  
  


\usepackage[titletoc]{appendix}
\renewcommand{\appendixname}{Anhang}
\renewcommand{\appendixtocname}{Anhang}
\renewcommand{\appendixpagename}{Anhang}

%\includeonly{3_elec_transport/3_electron_transport}
%\includeonly{5_semiconductor/5_semiconductor}
%\includeonly{6_superconductors/6_superconductors}


\begin{document}

  \tikzexternaldisable


\title{Experimentalphysik C2 \\ Festkörperphysik II}

\author{Markus Lippitz}
\date{\today}


\maketitle

 
%
\tableofcontents

%\renewcommand{\lastmod}{\ \ }
\renewcommand{\chapterauthors}{\ \ }

\chapter*{Vorwort}

Dies ist das Vorlesungsskript meiner Vorlesung 'Festkörperphysik II'. Sie ist eine Kursvorlesung für  Studierende im dritten Jahr des Bachelorstudiums. Bei der Auswahl und Gewichtung der Themen folgt sie sehr stark dem in Bayreuth Üblichen. Ich danke an dieser Stelle insbesondere Werner Köhler und Anna Köhler, deren Vorlesungsskripte mir eine große Hilfe waren.

Neben dem Skript gibt es zu jedem Kapitel  insgesamt circa eine Stunde 'Vorlesung' auf Video\sidenote{\href{https://mms.uni-bayreuth.de/Panopto/Pages/Sessions/List.aspx?folderID=514f4be0-2223-4111-831f-b06800ede4ed}{mms.uni-bayreuth.de}}, in der ich mündlich durch den Text führe und dabei an den Rand kritzle.
Ich habe den Eindruck, dass es mir beim Sprechen leichter fällt, die Dinge in einen Zusammenhang zu bringen als beim Schreiben, da ich mich traue, schlampiger zu sein. Zur Vorbereitung gab es dann noch ein online multiple-choice Quiz, sowie die Möglichkeit, jederzeit anonym Fragen zu stellen.\sidenote{\href{http://frag.jetzt}{frag.jetzt}}  Im Plenum  besprechen wir offene Fragen und diskutierten Aufgaben ähnlich zu Eric Mazurs 
ConcepTests.\sidenote{\href{https://mazur.harvard.edu/research-areas/peer-instruction}{mazur.harvard.edu}}  Schließlich gibt es die in der Physik üblichen Übungszettel und Kleingruppen-Übungen. 
%Manche Übungsaufgaben und Beispiele benutzen Julia\sidenote{\href{https://julialang.org/}{julialang.org}}  und Pluto.\sidenote{\href{https://github.com/fonsp/Pluto.jl}{Pluto.jl}} 



Dieses Skript ist 'work in progress', und wahrscheinlich nie wirklich fertig.  Ich danke allen Studierenden des Jahrgangs 2023, die den Text und die Gleichungen aufmerksam gelesen haben, wodurch wir viele Fehler korrigieren konnten. Trotzdem wird es noch welche geben. Wenn Sie Fehler finden, sagen Sie es mir bitte. 
Die aktuellste Version des Vorlesungsskripts finden Sie auf github.\sidenote{\href{https://github.com/MarkusLippitz/Festkoerper-II}{Festkoerper-II}}  Ich habe alles unter eine CC-BY-SA-Lizenz gestellt (siehe Fußzeile). In meinen Worten: Sie können damit machen, was Sie wollen. Wenn Sie Ihre Arbeit der Öffentlichkeit zur Verfügung stellen, erwähnen Sie mich und verwenden Sie eine ähnliche Lizenz. 


Der Text wurde mit der LaTeX-Klasse 'tufte-book' von Bil Kleb, Bill Wood und Kevin Godby\sidenote{\href{https://tufte-latex.github.io/tufte-latex/}{tufte-latex}} gesetzt, die sich der Arbeit von Edward Tufte\sidenote{\href{https://www.edwardtufte.com/}{edwardtufte.com}} annähert. Ich habe viele der Modifikationen angewandt, die von Dirk Eddelbüttel im R-Paket 'tint' eingeführt wurden\sidenote{\href{https://dirk.eddelbuettel.com/code/tint.html}{tint: tint is not Tufte}}. Die Quelle ist vorerst LaTeX, nicht Markdown.




\vspace{2\baselineskip}

Markus Lippitz \\ Bayreuth, 24. August 2023

 
 







%\renewcommand{\lastmod}{\today}
\renewcommand{\chapterauthors}{Markus Lippitz}
\renewcommand{\lastmod}{4. April 2023}

\chapter{Wärmeleitung und anharmonische Effekte}




\section{Ziele}
 


\begin{itemize}
\item Sie können die unten gezeigte Wärmeleitfähigkeit im Zusammenspiel von Wärmekapazität und Umklappprozessen erklären.
\item Sie können die Konzepte Wechselwirkungsquerschnitt und mittlere freie Weglänge benutzen, um Streuprozesse zu beschreiben.

\end{itemize}


\begin{figure}
    \inputtikz{\currfiledir fig_si}
    \caption{Wärmeleitfähigkeit $K$  von Silizium (\cite{Glassbrenner1964}).}
    \label{fig:1_WL_Si}
\end{figure}
 

\section{Überblick}

Bis zu diesem Punkt wurden das Bindungspotential der Atome im Festkörper als harmonisch, also rein parabelförmig angenommen. In diesem Kapitel gehen wir darüber hinaus und betrachten, wie in der Molekülphysik, anharmonische höhere Terme in Bindungspotential. Dies führt zur thermischen Ausdehnung und zur  Phonon--Phonon--Wechselwirkung. Mit ihr werden wir die oben gezeigte Temperaturabhängigkeit der Wärmeleitfähigkeit erklären.

Dieses Kapitel stellt die Verbindung zum vorangegangenen Semester her. Sie haben Gelegenheit, die zentralen Konzepte der ersten Kapitel der Festkörperphysik aus dem letzten Semester zu wiederholen. Wir brauchen den reziproken Raum, die Dispersionsrelation und die sich daraus ergebende Zustandsdichte, wenn wir im nächsten Kapitel zu Elektronen wechseln.


\section{Wiederholung}

Vergewissern Sie sich, dass Sie die folgenden Fragen beantworten können, und lesen Sie ggf. noch einmal in Ihren Aufzeichnungen des letzten Semesters oder in meinem Skript\footcite{lippitz_epc1} nach.


\subsection*{Kristallstruktur}
 
\begin{itemize}\setlength{\itemsep}{0pt}
    \item Was ist ein Bravais-Gitter, eine Basis, eine Kristallstruktur?
    \item Wie sehen häufig vorkommende Bravais-Gitter aus? Welche Symmetrien haben sie?
    \item Welche Arten von Bindungen gibt es in Festkörpern? Wo sind dabei die Elektronen, auch relativ zu 'ihrem' Atomkern?
\end{itemize}


\subsection*{Reziproker Raum}

\begin{itemize}\setlength{\itemsep}{0pt}
    \item Was ist der reziproke Raum, die Brillouin-Zone, ein Miller'scher Index?
    \item Wie sehen die reziproken Gitter von häufig vorkommenden Bravais-Gittern aus?
    \item Wie bestimmt man Gitterparameter experimentell ?
    \item Was besagt die Laue-Theorie der Beugung? Und die Bragg-Theorie?
    \item Was ist ein Strukturfaktor und ein Atomformfaktor?
\end{itemize}

\subsection*{Phononen}

\begin{itemize}\setlength{\itemsep}{0pt}
    \item Was ist ein Phonon, eine Dispersionsrelation, eine Zustandsdichte?
    \item Wie sieht die Dispersionsrelation einer ein- oder zwei-atomaren linearen Kette aus? Wie die zugehörige Zustandsdichte?
    \item Warum nennt man die Zweige optisch bzw. akustisch? Wie geht das im Dreidimensionalen?
    \item Wie kann man durch inelastische Neutronenstreuung diese Dispersionsrelation messen?
\end{itemize}

\subsection*{Wärmekapazität der Phononen}

\begin{itemize}\setlength{\itemsep}{0pt}
    \item Wie erklärt man mikroskopisch die Wärmekapazität (von Isolatoren), insbesondere deren Temperaturabhängigkeit?
    \item Was ist der Unterschied zwischen den Modellen von Debye und Einstein? Wann stimmt welches besser mit den Messungen überein? 
\end{itemize}



%\section{Was wir bislang noch nicht erklären können}



\section{Thermische Ausdehnung}

Bislang haben wir das Bindungspotential $U(x)$ der Atome als harmonisch angenommen. Die Abhängigkeit von der Auslenkung $x$ um die Ruheposition war also $U(x) \propto x^2$. Nun wollen wir betrachten, welchen Effekt höhere Terme im Potential haben. Bei den Molekül-Schwingungen hatten wir bereits das Morse-Potential besprochen, mit dem ebenfalls die Abweichungen von der harmonischen Form modelliert wurde. Damals hat dies zu Verschiebung der ansonsten äquidistanten Schwingungsniveaus und zu einer Änderung das Auswahlregel für Schwingungsübergänge geführt.

Sei also\sidenote{Siehe auch \cite{Kittel_FK} eq. 5.38 und \cite{Kopitzki_FK} eq. 2.57} 
\begin{eqnarray}
    U(x) = c x^2 - g x^3 - f x^4 \quad ,
\end{eqnarray}
mit $c$, $g$ und $f$ als positive Konstanten. Die Nullpunktsenergie ist hier der Einfachheit halber weggelassen. Der ungerade $x^3$-Term flacht die positive $x$ Seite ab und macht die negative Seite steiler. Der $x^4$-Term wirkt symmetrisch, macht aber das Potential bei hohen Energien bzw. großen $x$  breiter und so die Bindung weicher.

Nun interessiert die mittlere Auslenkung $\braket{x}$ bei einer durch die Boltzmann-Verteilung gegebenen Besetzung der Schwingungszustände. Ein Zustand mit dem Bindungsabstand $x$ tritt auf mit der Wahrscheinlichkeit\sidenote{Siehe z.B. Gl. 22.8 in \cite{Fliessbach_statistik}.}
\begin{equation}
    \frac{e^{- \beta U(x)}}{\int  e^{- \beta U(x')} dx'} \quad ,
\end{equation}
wobei wie immer $\beta = 1 / k_B T$.
Damit ist die mittlere Auslenkung
\begin{equation}
  \braket{x} =   \frac{\int x e^{- \beta U(x)} dx}{\int e^{- \beta U(x')} dx'} \quad .
\end{equation}
Der Nenner hängt ja nicht von $x$, sondern nur von $x'$ ab und kann so vor das $dx$-Integral gezogen werden. Nun machen wir die Annahme, das $U(x) \ll k_B T$, also $\beta U(x) \ll 1$ und schreiben im Zähler
\begin{equation}
    e^{- \beta U(x)} = e^{- \beta c x^2} e^{+ \beta (g x^3 + f x^4)} \approx e^{- \beta c x^2}  \left( 1+ \beta g x^3 + \beta f x^4 \right)
    \quad .
\end{equation}
Im Nenner ignorieren wir gleich alle Terme jenseits von $c x^2$. Damit erhält man 
\begin{equation}
    \braket{x} = \frac{3 g}{4 c^2} k_B T \quad .
\end{equation}
Wie erwartet spielt der $f x^4$-Term keine Rolle für die Änderung des Bindungsabstands.
Sobald aber ein kubischer Term im Potential vorhanden ist ($g \neq 0$), dann ändert sich die mittlere Auslenkung hin zu größeren Werten, proportional zur Temperatur $T$. Die Gitterkonstante ändert sich also linear in der Temperatur, bzw. der Wärme-Ausdehnungskoeffizient $\alpha$ ist 
\begin{equation}
    \alpha = \frac{d}{dT} \frac{ \braket{x}}{R_0} = \frac{3 g k_B}{4 c^2 R_0}  \quad ,
\end{equation}
mit dem mittleren Bindungsabstand $R_0$ bei $T=0$.

\begin{questions} 
\item Wie groß ist ein typischer Wärme-Ausdehnungskoeffizient $\alpha$? Wie könnte man damit die Koeffizienten $c$ und $g$ des Potentials vergleichen?
\item Warum nimmt man hier die Boltzmann-Verteilung, und nicht Bose-Einstein?
\end{questions}
 



% XXX TODO: Lösungen Anharm. Potential Kernwellenfunktionen und Verschiebung des Mittelwerts


% \begin{marginfigure}
%     \inputtikz{\currfiledir anharm_osc_wf}
%     \caption{Wellenfunktionen des anharmonischen Oszillators.}
% \end{marginfigure}



\section*{Phonon--Phonon-Wechselwirkung}

Die Anharmonizität des Potentials führt dazu, dass die Phononen miteinander wechselwirken. Die Einführung der Normalmoden in der Molekül- oder Festkörperphysik war möglich, weil dort das Potential als harmonisch angenommen wurde. Der $x^3$-Term führt dazu, dass die einzelnen Moden nicht mehr unabhängig voneinander sind, miteinander koppeln.\sidenote{Eine Rechnung findet sich in \cite{Gross_FK}.} Ein Molekül im Vakuum kann so Energie von einer hoch angeregten Schwingungsmode auf alle anderen Moden verteilen. Im Bild der quantisierten Schwingungen, wenn man also Phononen als Teilchen betrachtet, dann bedeutet dies, dass Phononen miteinander unter Beachtung der Energie- und Impulserhaltung wechselwirken, wie Billardkugeln.

Das kann man experimentell nachweisen. Zwei sich kreuzende Ultraschallwellen erzeugen eine dritte Welle in der durch die Impulserhaltung erwarteten Richtung (Abb.~\ref{fig:1_US_interaction}).

\begin{figure} 
    \input{\currfiledir fig_interaction.tikz.tex}
    \caption{Phonon--Phonon--Wechselwirkung in polykristallinem Magnesium (\cite{RollinsJr1964}). Zwei Ultraschallwellen kreuzen sich unter dem Winkel $\phi$. Man detektiert im Winkel der Impulserhaltung die resultierende Amplitude. Falls eine der Wellen transversal, die andere longitudinal ist, dann beobachtet man eine Auslöschung unter einem charakteristischen Winkel. }
    \label{fig:1_US_interaction}
\end{figure}

\begin{questions} 
    \item Falls Sie die 'Moderne Optik' besucht haben: Könnte man diesen Effekt auch im Wellen-Bild beschreiben?
\end{questions}
     

\section*{Wärmeleitfähigkeit}

 Bislang hatte der Festkörper überall die gleiche Temperatur. Nun betrachten wir beispielsweise einen Stab, der an beiden Enden durch ein Wärmebad auf eine zeitliche konstante aber verschiedene Temperatur gehalten wird. Makroskopisch ist der Fluss der thermischen Energie, also die Wärmestromdichte $\mathbf{j}_q$ abhängig von der Wärmeleitfähigkeit $K$ und dem Gradienten der Temperatur $T$, also
\begin{equation}
    \mathbf{j}_q = - K \, \nabla T \quad .
\end{equation}
Wir modellieren den Wärmestrom analog zur kinetischen Gastheorie  als Diffusion von Phononen. Wie auch schon bei der Wärmekapazität der Phononen steigt mit steigender Temperatur die Anzahl der Phononen bei den durch die Zustandsdichte erlaubten Frequenzen. Am warmen Ende des Stabes gibt es also mehr Phononen, die dann zum kalten Ende gelangen und so Energie transportieren. Dieser Transportprozess steckt in der Wärmeleitfähigkeit $K$. Die kinetische Gastheorie  ergibt
\begin{equation}
    K  = \frac{1}{3} \, C \, v \, \ell \quad , \label{eq:2_def_waermeleitf}
\end{equation}
mit der Wärmekapazität pro Volumen $C$, der Teilchengeschwindigkeit $v$ und der mittleren freien Weglänge $\ell$. Hier ist nun $C$ die Wärmekapazität der Phononen, $v$ deren Schallgeschwindigkeit und $\ell$ eine noch zu beschaffende mittlere freie Weglänge. Eigentlich sind alle drei Größen von der Frequenz und ggf. auch der Richtung abhängig. Dies ignorieren wir hier und verstehen sie als effektive Größen. Das ist die \emph{Dominante-Phononen-Näherung}, ähnlich wie beim Einstein-Modell die optischen Phononen als deltaförmige Zustandsdichte angenommen wurden.



\section*{Mittlere freie Weglänge}

\begin{marginfigure}
    \inputtikz{\currfiledir crosssection}
    \caption{Scheiben der Fläche $\sigma$ mit einer Anzahl-Dichte $n$ ergeben geometrisch die mittlere freie Weglänge $\ell$.}
     \label{fig:1_crosssection}
\end{marginfigure}

Die mittlere freie Weglänge kann rein geometrisch verstanden werden als Weglänge, bis zu der ein Strahl wieder auf eine Zielscheibe trifft. Die Fläche der Zielscheibe entspricht dabei dem Wechselwirkungsquerschnitt $\sigma$. Dazu benötigt man nur die Anzahl der Scheiben pro Volumen, also die Dichte $n$. Damit ist die mittlere freie Weglänge $\ell$
\begin{equation}
    \ell = \frac{1}{n \, \sigma} \quad .  \label{eq:1_def_weglaenge} 
\end{equation}
In der Sprache der Streutheorie, wie beispielsweise bei der Röntgenstreuung, ist der Wechselwirkungsquerschnitt $\sigma$ proportional zum Betragsquadrat $|\mathcal{A}|^2$ der Streuamplitude.

Dem Scheibchen-Bild nahe kommt die Streuung an Punktdefekten im Kristall. Wenn die Ausdehnung $a$ des Defekts viel kleiner als die Wellenlänge $\lambda$ des Phonons ist, dann ist die Physik völlig analog zur Rayleigh-Streuung, beispielsweise von Licht an Luft-Molekülen. Der Streuquerschnitt ist in diesem Fall
\begin{equation}
    \sigma \propto \frac{a^6}{\lambda^4} \quad \text{oder} \quad \propto a^6 \, \omega^4 \quad .
\end{equation}
Dieser Effekt ist nicht temperaturunabhängig, kann also nicht helfen, die Temperaturabhängigkeit der Wärmeleitung zu erklären. Er liefert vielmehr eine von der Qualität der Probe anhängen konstanten Beitrag.



\section{Phonon--Phonon--Streuung}


% Bei der Phonon--Phonon--Wechselwirkung sind \emph{drei} Phononen involviert. In den Ultraschall-Beispiel oben laufen 2 Phononen ein und ein drittes wird erzeugt und läuft aus, mit $\omega_\text{out}  = \omega_\text{in} + \omega_{US}$. Genauso ist auch möglich, dass ein Phonon einläuft, und zwei neue unter Energie- und Impulserhaltung auslaufen ($\omega_\text{in}  = \omega_\text{out} + \omega_{US}$). Beide Richtungen zusammen begrenzen die mittlere freie Weglänge eines Phonons. Es ist aber ein Wechselspiel zwischen Erzeugung und Vernichtung von Phononen. Wir nehmen nun an, dass sich die Energie nicht sehr ändert, also $\omega_\text{in}  = \omega_\text{out} = \omega$.

% Im Sinne von Gl. \ref{eq:1_def_weglaenge} ist die Dichte $n$ der Streuzentren also die Besetzung $\braket{n(\omega, T)}$ der Phononen-Zustände. Durch das Wechselspiel von Erzeugung und Vernichtung geht dabei nur der Unterschied, also die Ableitung der Besetzung ein, dekoriert mit der Zustandsdichte. Alles zusammen ist das\footnote{Hunklinger, eq. 7.15}
% \begin{equation}
%     \frac{1}{\ell} = n \sigma = \int \sigma(\omega) D(\omega) \frac{\partial \braket{n(\omega, T)} }{\partial \omega} d \omega  
% \end{equation}
% Das hat formal große Ähnlichkeit mit der Berechnung der Inneren Energie und der Wärmekapazität der Phononen. Wie dort verwenden wir die Abkürzungen $x = \hbar \omega / k_B T$ und 
% $x_D = \hbar \omega_D / k_B T = \Theta / T$ mit der Debye-Temperatur $\Theta$. Für den Wechselwirkungsquerschnitt $\sigma$ gilt hier, dass er proportional zum Produkt aller drei involvierten Frequenz ist, also $\sigma \propto \omega_{US} \, \omega^2$. Und die Zustandsdichte ist im Deybe-Modell proprtional zu $\omega^2$. Insgesamt ergibt das
% \begin{eqnarray}
%     \frac{1}{\ell} \propto & \omega_{US} \int \omega^4 \frac{\partial \braket{n(\omega, T)} }{\partial \omega} d \omega  \\
%     \propto & \omega_{US} \, T^4  \int_0^{x_D} \frac{x^4 e^4}{(e^x-1)^2} d \omega 
% \end{eqnarray}
% Für hohe Tempetratiren, also $x \rightarrow 0$, ist das Intgeral proportioanl zu $(\Theta / T)^3$, für tiefe Tempearturen ist es uanbhängig von $T$. Ingesamt haben wir damit
% \begin{eqnarray}
%     \frac{1}{\ell}    \propto & \omega_{US} \, T^4   \quad \text{für} \quad $T \ll \Theta \\
%                        \propto & \omega_{US} \, T   \quad \text{für} \quad $T \gg \Theta 
% \end{eqnarray}

% weil für tiefe Temperaturen das Integral konstant ist.


Auch die Phonon--Phonon--Wechselwirkung kann die freie Weglänge begrenzen, indem zwei einfallende Phononen in ein neues umgewandelt werden. Aus der Sicht eines der einfallenden Phononen ist die Streuwahrscheinlichkeit proportional zur Dichte $n(T)$ der anderen Phononen, also erwarten wir
\begin{equation}
    \ell \propto \frac{1}{n(T)} \quad .
\end{equation}



Bei der Streuung von Phononen in einem Kristall muss man allerdings den reziproken Gittervektor $\mathbf{G}$ berücksichtigen. In einem Kristall ist die Impulserhaltung
\begin{equation}
    \mathbf{k}_1 +  \mathbf{k}_2 =  \mathbf{k}_3 +  \mathbf{G}_{hkl}  \quad .
\end{equation}
Der reziproken Gittervektor $\mathbf{G}_{hkl}$ meint eine (unendliche) Menge von Vektoren, die sich aus den Linearkombinationen der primitiven Einheitsvektoren mit den Koeffizienten $h,k,l$ zusammensetzt. Damit unterscheiden wir den \emph{Normalprozess} ($\mathbf{G} = 0$) vom \emph{Umklappprozess} ($\mathbf{G} \neq 0$). Im Normalprozess gilt die Impulserhaltung in der strengen Form wie im Vakuum. Bei einem Gas von Phononen bleibt der Gesamtimpuls dann aber erhalten. Die Drift-Geschwindigkeit der Phononen kann sich nicht ändern und dieser Fall trägt nicht zu einem Wärmewiderstand bei.

\begin{marginfigure}
    \inputtikz{\currfiledir umklapp}
   \caption{Skizze zum Umklappprozess. Wenn die Summe von zwei reziproken Vektoren außerhalb der Brillouin-Zone liegt, dann führt die Addition von $\mathbf{G}$ zur Änderung der Richtung. }
   \label{fig:1_umklapp}
\end{marginfigure}

Beim Umklappprozess kann sich aber die Richtung ändern. Die Summe $ \mathbf{k}_1 +  \mathbf{k}_2 $ kann gerade über die erste Brillouin-Zone hinaus reichen, wird durch $\mathbf{G}$ zurück verschoben und kann dann entgegen der ursprünglichen Vektoren zeigen (siehe Abbildung~\ref{fig:1_umklapp}). Damit ändert sich der Gesamtimpuls des Phononen-Gases, was einem Wärmewiderstand entspricht.

\begin{questions} 
    \item Zeigt Abbildung \ref{fig:1_umklapp} den Realraum oder den reziproken Raum?  Wie sieht das im anderen Raum aus?
    \item Wie kommt es, dass hier die Impulserhaltung verletzt ist?
\end{questions}
     

\section{Temperaturabhängigkeit des Umklappprozesses}

Damit der Umklappprozess stattfindet, muss
\begin{equation}
    | \mathbf{k}_1 +  \mathbf{k}_2| \ge \frac{1}{2} | \mathbf{G} | \quad ,
\end{equation}
wobei $\mathbf{G}$ hier den kleinsten reziproken Gittervektor meint. Wir benötigen die Energie der Phononen mit solchen Impulsen $\mathbf{k}_i$. Dazu nehmen wir das Debye-Modell an, also einen linearen Zusammenhang zwischen dem Betrag des Impulses und der Frequenz des Phonons und eine Debye-Temperatur $\Theta$.  Am Rand der Brillouin-Zone haben die Phononen in diesem Modell die Energie $k_B \Theta$, so dass eine charakteristische Energie für den Einsatz des Umklappprozesses $k_B \Theta / 2$ ist. Die Besetzungsdichte bei dieser Energie ist in der Boltzmann-Verteilung
\begin{equation}
    \braket{n} \propto \frac{1}{e^{\Theta / 2T} -1}
\end{equation}
und die mittlere freie Weglänge ist somit
\begin{equation}
    \ell \propto e^{\Theta / 2T} -1 = 
    \left\{
    \begin{matrix*}
        \Theta / T             & \text{für} \quad T \gg \Theta  \\     
        e^{\Theta / 2T} \quad & \text{für} \quad T \ll \Theta 
    \end{matrix*}
    \right. \quad .
\end{equation}

Bei sehr tiefen Temperaturen ist also die mittlere freie Weglänge durch die Streuung an Punktdefekten begrenzt und fällt dann exponentiell mit der Temperatur ab, weil immer mehr Phononen als Streupartner hinzu kommen. Mit steigender Temperatur geht der exponentielle Abfall oberhalb der Debye-Temperatur $\Theta$ in einen $1/T$-Verlauf über.


Für die Wärmeleitfähigkeit benötigen wir noch die Temperaturabhängigkeit der Wärmekapazität $C$. Diese ist nach dem Debye-Modell proportional zu $T^3$ bei $T \ll \Theta$. Weit oberhalb $\Theta$ gilt das Dulong-Petit-Gesetz und die Wärmekapazität ist konstant. Insgesamt erhalten wir damit 
\begin{equation}
    K = \frac{1}{3} c v \ell  \propto 
    \left\{
    \begin{matrix*}[l]
        \Theta / T                    & \text{für} \quad T \gg \Theta    & \text{Phonon--Phonon}      \\
      T^n \,  e^{\Theta / 2T}        & \text{für} \quad T \ll \Theta    & \text{Phonon--Phonon}      \\
      T^3                            & \text{für} \quad T \lll \Theta   & \text{Phonon--Defekt}     
    \end{matrix*}
    \right. \quad .
\end{equation}
Der Exponent $n$ bei $ T \ll \Theta  $ soll die genaue Temperaturabhängigkeit offen lassen. Dazu müsste man das Integral im Debye-Modell der Wärmekapazität im Bereich $T \approx \Theta$ lösen. 

Für Silizium finden wir in den gemessenen Daten (Abb.~\ref{fig:1_WL_Si}) sowohl die $T^3$-Abhängigkeit bei tiefen Temperaturen, als auch die $T^{-1}$ oberhalb der Debye-Temperatur. Der Übergangsbereich ist aufwändiger zu modellieren.  Natriumfluorid (\ch{NaF}) verhält sich ähnlich (Abb.~\ref{fig:1_WL_NaF}).

\begin{figure}
    \inputtikz{\currfiledir fig_NaF}
    \caption{Wärmeleitfähigkeit $K$  von Natriumfluorid (\ch{NaF}) (\cite{Jackson1970}. Die Debye-Temperatur von \ch{NaF} beträgt 491~K.}
    \label{fig:1_WL_NaF}
\end{figure}



\begin{questions} 
\item Beschreiben Sie in Ihren Worten, wie es zur Temperaturabhängigkeit der Wärmeleitfähigkeit kommt, insbesondere bei hohen und niedrigen Temperaturen.
\end{questions}
 


% Das Pluto-Skript hydrogen\_wave\_functions\pluto{hydrogen_wave_functions} ermöglicht es Ihnen, mit verschiedenen Varianten der grafischen Darstellung zu experimentieren.

\newpage
\section{Zusammenfassung}

\textit{Schreiben Sie hier ihre persönliche Zusammenfassung des Kapitels auf. Konzentrieren Sie sich auf die wichtigsten Aspekte und die am Anfang genannten Ziele des Kapitels.}

\vspace*{10cm}

\printbibliography[segment=\therefsegment,heading=subbibliography]

%\renewcommand{\lastmod}{\today}
\renewcommand{\chapterauthors}{Markus Lippitz}
\renewcommand{\lastmod}{6. Mai 2025}

\chapter{Fermi-Gas}
\label{chap:fermi-gas}




\section{Ziele}
 


\begin{itemize}  
\item Sie können das Konzept der Fermi-Kugel benutzen, um den elektrischen Widerstand von Metallen zu erklären.
\item Sie können den mikroskopischen  Ursprung des Wiedemann-Franz-Gesetzes erklären, also warum die Temperaturabhängigkeit der elektrischen und  der thermischen Leitfähigkeit in Metallen so ähnlich bis identisch ist.

\end{itemize}

\begin{figure}
    \inputtikz{\currfiledir Cu_Lorenz}
    \caption{Temperaturabhängigkeit der thermischen ($K$) und elektrischen ($\sigma$) Leitfähigkeit von Kupfer und die daraus abgeleitet Lorenz-Zahl $L = K/ \sigma T$. Das Wiedemann-Franz-Gesetz besagt, dass diese konstant ist. 
        Daten aus \cite{Hust1984}. \label{fig:2_Cu_Lorenz}}
\end{figure} 

\section{Überblick}

Mit diesem Kapitel beginnen wir mit den elektronischen Eigenschaften der Festkörper. Bislang hatten wir diese dadurch ignoriert, dass die Beispiele immer als Isolatoren gewählt waren, Elektronen also keine Rolle gespielt haben. Von nun an stehen die Elektronen im Mittelpunkt. Wie in der Molekülphysik auch machen wir die adiabatische Näherung. Wir nehmen also an, dass die Elektronen viel schneller sind als die Kerne, die Kerne aber das Potential vorgeben, in dem sich die Elektronen bewegen. Weiterhin machen wir die Ein-Elektron-Näherung. Wir betrachten also nur ein Elektron. Die Anwesenheit aller anderen Elektronen beeinflusst nur das Potential, auch über das Pauli-Prinzip. Korrelationen zwischen Elektronen berücksichtigen wir  erst in den Kapiteln zur Supraleitung und zum Magnetismus.

In diesem Kapitel bildet der Kristall ein großes Kastenpotential, aber die Kerne selbst kommen nicht vor. Im folgenden Kapitel wird dann das Kristallgitter wichtig werden.




% \begin{questions} 
% \item Wie groß ist ein Molekül?
% \item Welche physikalische Eigenschaft eine Moleküls wird bei Röntgenstreuung, STM und AFM abgebildet?
% \end{questions}
 
% Das Pluto-Skript hydrogen\_wave\_functions\pluto{hydrogen_wave_functions} ermöglicht es Ihnen, mit verschiedenen Varianten der grafischen Darstellung zu experimentieren.


\section{Freies Elektronengas}

In beispielsweise Alkali-Metallen sind die meisten Elektronen an 'ihren' Atomkern gebunden und nur sehr wenige Elektronen pro Atomkern tragen zur Bindung bei. Diese Elektronen sehen nicht das vollständige Coulomb-Potential der stark positiv geladenen Atomrümpfe. Die gebundenen Elektronen schirmen dies ab, so dass nur ein schwaches und räumlich beinahe konstantes Potential verbleit. In diesem Potential bewegen sich die Valenz-Elektronen der Alkali-Atome wie ein Gas. Man spricht daher von freien Elektronengas oder auch Fermi-Gas.

Wir könnten die freien Elektronen als Teilchen im 3D-Kasten modellieren. Die Schrödingergleichung innerhalb des Kastens beinhaltet dann nur noch die kinetische Energie
\begin{equation}
    - \frac{\hbar^2}{2m} \nabla^2  \psi(\mathbf{r}) = E  \psi(\mathbf{r})
\end{equation}
und ihre Lösung sind ebene Wellen 
\begin{equation}
    \psi(\mathbf{r}) = \frac{1}{\sqrt{V}} e^{i \mathbf{k} \cdot \mathbf{r}}
\end{equation}
mit dem Wellenvektor $\mathbf{k}$ und der Normierung auf das Volumen $V$ des Kastens. Die Energie beträgt dann
\begin{equation}
    E = \frac{\hbar^2 }{2m} |\mathbf{k}|^2 = \frac{|\mathbf{p}|^2 }{2m} 
\end{equation}
und ist von der Richtung natürlich unabhängig.

Die Gleichungen werden angenehm, wenn wir (wie bei den Phononen) periodische Randbedingungen einführen: alle Eigenschaften sollen im Ort periodisch mit der Kasten-Größe $L$ sein, also 
\begin{equation}
    \psi(\mathbf{r}) =  \psi(\mathbf{r} + L \mathbf{\hat{e}_i}) \quad ,
\end{equation}
mit $\mathbf{\hat{e}_i}$ einem kartesischen Einheitsvektor. Damit umgehen wir auch das Problem, was außerhalb des Kastens eigentlich passiert.
Wie bei den Phononen sind die möglichen Werte des Wellenvektors  $\mathbf{k}$ diskret
\begin{equation}
    k_i = \frac{2 \pi}{L} \, m_i \quad \text{mit} \quad i = x,y,z \quad . \label{eq:2:k_randbed} 
\end{equation}




\section*{Zustandsdichte}

Die Zustandsdichte beschreibt die Anzahl der Zustände pro Energie- oder Wellenvektor-Intervall ($D(E) \, dE$ oder $D(k) \, dk$). Manchmal findet man auch zusätzlich eine Normierung auf das  Volumen, die wir hier aber nicht durchführen. Durch die Quantisierung der Wellenfunktion im Kasten sind nur noch diskrete Energien möglich. Je größer der Kasten wird, desto dichter liegen diese diskreten Niveaus beieinander. Man kann sie dann abzählen und eine Dichte (Anzahl pro Energieintervall) angeben. Der Weg zur Berechnung einer Zustandsdichte ist eigentlich immer derselbe, den Sie schon bei den Phononen gesehen hatten: Man beginnt im reziproken Raum, weil es dort einfach ist, und wandelt das dann in eine Energie um.

Die Zustandsdichte im reziproken Raum ist konstant:
\begin{equation}
    D(k) dk = \frac{V}{(2 \pi)^3} dk \quad .
\end{equation}
Um sie als Funktion der Energie zu erhalten benötigen wir wie bei den Phononen die Gruppengeschwindigkeit 
\begin{equation}
    v_g = \frac{\partial \omega}{ \partial k} = \frac{\partial E}{\partial (\hbar k)} = \frac{\hbar k}{m}
\end{equation}
und erhalten\sidenote{Für 3 Dimensionen. Niedrigdimensionale Strukturen kommen später.} 
\begin{equation}
    D'(E) dE = \frac{V}{\hbar (2 \pi)^3} \, dE \, \int_{E = \text{const.}} \frac{d S_E}{v_g}  
\end{equation}
wobei das Integral über eine Kugeloberfläche konstanter Energie im reziproken Raum läuft und $4 \pi k^2 / v_g$ ergibt.
Jetzt müssen wir noch berücksichtigen, dass wir jeden Zustand nach dem Pauli-Prinzip mit zwei Elektronen unterschiedlichen Spins besetzen können:
\begin{equation}
    D(E) dE = 2 D'(E) dE =  \frac{(2m)^{3/2}}{2 \pi^2 \hbar^3} \, V \,  \sqrt{E} \, dE \quad .
\end{equation}
Damit haben wir die Zustandsdichte von freien Elektronen.

\begin{marginfigure}
    \inputtikz{\currfiledir fermi-gas}
    \caption{Dispersionsrelation $E(k)$ und Zustandsdichte $D(E)$ eines Fermi-Gases in 3 Dimensionen.}
\end{marginfigure}


\begin{questions} 
    \item Wie kann man den wurzelförmigen Verlauf der Zustandsdichte $D(E)$ verstehen?
\end{questions}


\section*{Fermi-Energie und Fermi-Kugel}

In Metallen, insbesondere Alkali-Metallen sind die Eigenschaften maßgeblich durch die freien Elektronen bestimmt.
Elektronen sind Fermionen, haben einen halbzahlgen Spin und unterliegen  dem Pauli-Prinzip und der Fermi-Dirac-Statistik. Das macht die Besetzung der Zustände besonders. Im thermischen Gleichgewicht ist  jeder Zustand besetzt wie 
\begin{equation}
    f(E) = \frac{1}{e^{(E-\mu)/k_B T} + 1}
\end{equation}
mit dem chemischen Potential $\mu$. Die Fermi-Dirac-Verteilung ist (um das Pauli-Verbot zu erfüllen) maximal Eins. Wenn $f(E) \ll 1$, also $E- \mu \gg k_B T$, dann geht sie in die Boltzmann-Verteilung über. Dies wird später bei den Halbleitern wichtig werden, da dort die interessanten Energien $E$ relativ weit weg von $\mu$ sind.

\begin{marginfigure}
    \inputtikz{\currfiledir FD_stat}
    \caption{Fermi-Dirac-Statistik (fett) in Vergleich zur Bose-Einstein-Statistik (gestrichelt) und Boltzmann-Statistik (dünn).}
\end{marginfigure}

Das chemische Potential $\mu$ kommt aus der Ableitung beispielsweise der inneren Energie $U$ nach der Stoffmenge $n_i$. Bei mehreren Stoffen gibt es also mehrere $\mu_i$.
\begin{equation}
    \mu_i = \left( \frac{\partial U}{\partial n_i} \right)_{V,S,n_j \neq n_i}
\end{equation}
bzw.
\begin{equation}
    dU = T dS - p dV + \sum_i \mu_i \, d n_i \quad .
\end{equation}
Bei uns ist der Stoff natürlich die Elektronen, daher brauchen wir im folgenden kein Index an $\mu$.

Bei $E = \mu$ geht die Fermi-Dirac-Verteilungsfunktion immer durch $1/2$. Am absoluten Nullpunkt ($T=0$) ist sie konstant Eins für $E < \mu$ und konstant Null darüber. Wir bezeichnen als \emph{Fermi-Energie} $E_F$ die Energie, bis zu der alle Zustände lückenlos gefüllt sind. Das entspräche dem chemischen Potential, wenn letzteres nicht temperaturabhängig wäre. So definieren wir
\begin{equation}
    E_F = \mu (T = 0) \quad .
\end{equation}
Damit ist die Fermi-Energie \emph{nicht} temperaturabhängig. Später werden wir den Begriff 'Fermi-Niveau' einführen, der nur ein anderes Wort für chemisches Potential ist und damit temperaturabhängig.


Wir berechnen die Fermi-Energie $E_F$, indem wir bei $T=0$ nach und nach Elektronen in unseren Kasten einfüllen.
Bei $T=0$ füllen die Elektronen nach dem Pauli-Prinzip die tiefsten Zustände auf, so dass im Impulsraum eine gefüllte Kugel entsteht. Die Energie des höchsten gefüllten Zustandes ist die Fermi-Energie.
Wir integrieren also die Zustandsdichte $D(E)$ soweit auf, bis wir $N$ Elektronen untergebracht haben. Die Elektronendichte $n$ ist also\sidenote{Analog kann man ein temperaturabhängiges chemisches Potential ausrechnen. Siehe \cite{Hunklinger2014} oder \cite{Gross_FK}.  }
\begin{equation}
    n = \frac{N}{V} =\frac{1}{V} \, \int_0^{E_F} D(E) \, dE  =    \frac{(2m)^{3/2}}{2 \pi^2 \hbar^3} \,  \int_0^{E_F}  \sqrt{E} \, dE \quad .
\end{equation}
Damit erhalten wir
\begin{equation}
    E_F = \frac{\hbar^2}{2m} (3 \pi^2)^{2/3} \, n^{2/3} \quad .
\end{equation}
Alle Komposita mit 'Fermi-' sind entsprechend definiert. Der Fermi-Impuls $k_F$ ist einfach
\begin{equation}
    k_F = (3 \pi^2 \, n)^{1/3} \quad .
\end{equation}
Die \emph{Fermi-Kugel} ist die Kugel im reziproken Raum mit dem Radius $k_F$. Am absoluten Nullpunkt sind also alle Elektronen innerhalb dieser Kugel. Während die Fermi-Fläche im freien Elektronengas kugelförmig ist, wird sie in realen Materialien durch die Bandstruktur des Kristalls verzerrt. Dies führt zu komplexen \emph{Fermi-Flächen}, welche die elektronischen Transporteigenschaften bestimmen.


Bei Metallen trägt jedes Atom ein (oder mehr) Elektronen bei. Die Elektronendichte ist also etwa $n \approx 10^{28}$~m$^{-3}$.
Die Fermi-Energie von den hier betrachteten Metallen liegt typischerweise im Bereich von einigen Elektronvolt und die Fermi-Temperatur damit bei einigen 10~000~K, weit jenseits der Schmelztemperatur. Für Elektronen im Festkörper besteht also kein so großer Unterschied zum absoluten Nullpunkt. Die Stufenfunktion der Fermi-Dirac-Verteilung wird etwas abgerundet. Wenn man es maßstabsgerecht zeichnen würde, dann könnte man aber bei Raumtemperatur keinen Unterschied erkennen. In Halbleitern später ist die Ladungsträgerdichte viel kleiner.


\begin{questions} 
\item Woran liegt es, dass hier die Fermi-Fläche gerade eine Kugeloberfläche ist? Was ist notwendig, damit andere Formen  entstehen?
\item Wieviel Elektronen pro Atom muss ein Material ungefähr besitzen, damit die Fermi-Kugel den Rand der Brillouinzone berührt?
\item Was bedeutet 'Für Elektronen im Festkörper besteht also kein so großer Unterschied zum absoluten Nullpunkt' ?
\end{questions}



\section*{Wärmekapazität der freien Elektronen}

Neben den Gitterschwingungen (Phononen) tragen auch Elektronen zur Wärmekapazität von Metallen bei. Da die Elektronendichte in Metallen hoch ist, würde man erwarten, dass dieser Beitrag groß ist. Es zeigt sich jedoch, dass nur ein kleiner Teil der Elektronen thermisch angeregt werden kann -- eine direkte Folge der Fermi-Dirac-Statistik.

Analog zum Vorgehen bei den Phononen berechnen wir die Wärmekapazität der Elektronen als Ableitung der inneren Energie nach der Temperatur. Wir beginnen\sidenote{\cite{Hunklinger2014} folgend} mit der spezifischen inneren Energie $u$
\begin{equation}
    u = \frac{U}{V} = \frac{1}{V} \, \int_0^\infty E \, D(E) \, f(E) \,  dE \quad .
\end{equation}
Am absoluten Nullpunkt geht $f(E)$  in die Heavyside-Funktion über, so dass wir einfach das Integral nur bis $E_F$ laufen lassen
\begin{equation}
    u_0 = u(T=0) = \frac{1}{V} \, \int_0^{E_F} E \, D(E) \, dE = \frac{3n}{5} \, k_B \, T_F \quad .
\end{equation}
Bei einem klassischen freien Gas von Teilchen hätten wir 
\begin{equation}
    u_\text{klassisch} = \frac{3 n}{2} \, k_B \, T \quad \text{und} \quad c_\text{klassisch} = \frac{3 n}{2} \, k_B  \quad .
\end{equation}
Weil $T_F \gg T$ sind viele Elektronen schon bei Energie deutlich größer als $k_B T$. Dies ist eine Konsequenz des  Pauli-Verbots. Damit ist die  innere Energie eines freien Elektronengases sehr hoch. Wir müssen zu sehr hochenergetischen Zuständen ausweichen, um noch Elektronen zufügen zu können.

\begin{marginfigure}
    \inputtikz{\currfiledir fermi-dirac}
    \caption{Nur Zustände in der Nähe der Fermi-Energie tragen zur Wärmekapazität bei. Die graue Kurve ist um den Faktor 40 kühler und entspricht inm etwa den realen Verhältnissen bei Raumtemperatur.}
    \label{fig:2_fermi_dirac_T}
\end{marginfigure}


Die Ableitung $\partial u / \partial T$ ist aufwändig. Lehrbücher zeigen ein paar Schritte. Ich möchte das hier abkürzen und so argumentieren: ein freies Elektronengas ist quasi ein klassisches Gas, nur kann aufgrund der Fermi-Dirac-Statistik nur der Anteil $T/T_F$ weitere Energie aufnehmen und so zur Wärmekapazität beitragen. Zustände, die weiter von $E_F$ entfernt sind, sind entweder vollständig besetzt, so dass im Abstand $k_B T$ kein freier Zustand vorhanden ist, oder sie sind vollständig unbesetzt. Die Abschätzung ist also
\begin{equation}
    c_\text{geschätzt} = c_\text{klassisch}  \, \frac{T}{T_F} = \frac{3 n \, k_B}{2}  \, \frac{T}{T_F} \quad .
\end{equation}
Eine etwas bessere Rechnung\sidenote{Integration der Fermi-Dirac-Statistik und Taylor-Reihe um $E_F$} ergibt einen um den Faktor $\pi^2/3$ größeren Wert
\begin{equation}
    c_\text{el} = \frac{\pi^2 \, n \, k_B}{2}  \, \frac{T}{T_F} = \gamma T \label{eq:2_WK_elek}
\end{equation}
mit der Sommerfeld-Konstanten $\gamma$.

In Metallen tragen sowohl Phononen als auch Elektronen zur Wärmekapazität bei. Während die Phononen bei niedrigen Temperaturen mit dem Debye-Modell $T^3$-abhängig sind und bei hohen Temperaturen das Dulong-Petit-Gesetz erreichen, ist der Beitrag der Elektronen linear mit $T$. Die gesamte Wärmekapazität ist somit die Summe dieser beiden Terme:
\begin{equation}
    c_\text{ges} = \gamma \, T \, + \, 
        \left\{ 
        \begin{matrix}
            3 n_A k_B & \text{für} \quad T \gg \Theta \\
            \beta T^3 & \text{für} \quad T \ll \Theta 
        \end{matrix}
        \right. \label{eq:2_WK_Metall_ges}
\end{equation}
mit $\beta$ aus dem Debye-Modell und $n_A$ der Teilchenzahl-Dichte der Atom-Kerne.



\begin{marginfigure}
    \inputtikz{\currfiledir fig_Cu_WK}
    \caption{Wärmekapazität von Kupfer bei tiefen Temperaturen nach \cite{Rayne1956}. Elektronen und Phononen tragen bei. \label{fig:2_Cu_WK}}
\end{marginfigure}


Dieses Modell beschreibt die Wärmekapazität von Alkali-Metallen und anderen 'einfachen' Metallen gut (Abb.~\ref{fig:2_Cu_WK}). In anderen Fällen finden sich deutliche Abweichung, beispielsweise bei Nickel. Hier ist die gemessene Wärmekapazität um etwa den Faktor  15 höher als die wie oben berechnete. Bei den Alkali-Metallen ist die Annahme des freien Elektronengases gerechtfertigt. Bei Nickel tragen aber Elektronen zur Wärmekapazität bei, die aus atomaren d-Orbitalen stammen, daher eine Vorzugsrichtung haben und keine  isotrope Zustandsdichte im Kristall besitzen. Dies führt zu einer hohen Zustandsdichte an der Fermi-Energie und einem größeren  elektronischen Beitrag zur Wärmekapazität, da mehr Zustände in einem schmalen Energiebereich $k_B T$ thermisch angeregt werden können. \sidenote{siehe \cite{Hunklinger2014}, Abbildung 8.9}



\begin{questions} 
    \item Wo zeigt sich in Abb.~\ref{fig:2_fermi_dirac_T} die Wärmekapazität?
\end{questions}
    
    



\section{Drude-Modell}

Die herausragende Eigenschaft der Metalle ist ihre elektrische Leitfähigkeit. Die Leitfähigkeit ist hoch, aber nicht unendlich. Die Frage ist also, was die Bewegung der Elektronen so einschränkt, dass die Leitfähigkeit begrenzt bleibt. Dies wurde erstmals von Paul Drude um 1900 erklärt. Das Drude-Modell liefert das Ohmsche Gesetz, also ein richtiges Ergebnis, aber aus heutiger Sicht aus den falschen Gründen. Erst Arnold Sommerfeld und Hans Bethe brachten um 1933 die Quantenmechanik ins Spiel und lieferten die moderne Erklärung.

Die Annahmen des Drude-Modells sind ein freies Elektronengas, das nur mit den  Atomrümpfen, aber nicht miteinander stößt. Ohne externes Feld bewegen die Elektronen sich thermisch, aber ohne Vorzugsrichtung. Der Mittelwert über alle (vektoriellen) Geschwindigkeiten des Elektronengases ist also $\braket{\mathbf{v}} = 0$. Das externe elektrische Feld $\mathcal{E}$ überlagert der thermischen Bewegung eine Driftbewegung mit der Geschwindigkeit $\mathbf{v}_d = \braket{\mathbf{v}}$. Nach einem Abschalten des Feldes würde die Driftbewegung durch die Stöße langsam zum Erliegen kommen. Die Bewegungsgleichung ist also\footcite{Singleton_band_theory,Simon_solid_state_basics}
\begin{equation}
   m \frac{d \braket{\mathbf{v}} }{dt} = -e \, \mathcal{E} - m \frac{\braket{\mathbf{v}} }{\tau} \quad .
\end{equation}
$\tau$ ist dabei die mittlere Zeit zwischen zwei Stößen, also ein Maß für die Wechselwirkung der Elektronen mit dem Gitter. Sie hängt von der Temperatur, dem Material und seiner Reinheit ab. In Metallen beträgt $\tau \approx 10$~fs. Die Geschwindigkeit eines Elektrons bei $E_F$ beträgt etwa $10^6$~m/s, so dass die mittlere freie Weglänge einige 10~nm beträgt.

Im stationären Fall der Driftbewegung
($d \braket{\mathbf{v}} / dt = 0$) erhält man   
\begin{equation}
   \mathbf{v}_d =\braket{\mathbf{v}} =  - \frac{e \tau}{m} \mathcal{E} = - \mu \mathcal{E} 
   \quad \text{mit} \quad
    \mu = \frac{| \mathbf{v}_d |}{|\mathcal{E}|} = \frac{e \tau}{m} \label{eq:2_def_beweglichkeit}
\end{equation}
mit der Beweglichkeit $\mu$. Die Stromdichte ist dann
\begin{equation}
   \mathbf{j} = -e n  \mathbf{v}_d = n e \mu \mathcal{E} = \sigma \mathcal{E} 
   \quad \text{mit} \quad 
   \sigma = n e \mu  = \frac{n e^2 \tau}{m}
\end{equation}
mit der Elektronendichte $n$ und der Leitfähigkeit $\sigma$. Damit haben wir den linearen Zusammenhang zwischen Strom und Spannung des Ohm'schen Gesetzes erhalten. Der makroskopische Widerstand (bzw. dessen reziproker Wert, die Leitfähigkeit $\sigma$) ist verknüpft mit zwei mikroskopischen  Größen, der Elektronendichte $n$ und der mittleren Stoßzeit $\tau$. Erstere ergibt sich aus der Zahl der Valenz-Elektronen pro Atom und der Gitterkonstanten des Kristalls. Letzte liegt wie oben schon erwähnt bei etwa 10~fs.

Dieses Modell ignoriert völlig das Pauli-Prinzip und dass es eine Fermi-Dirac-Verteilung gibt, bei der quasi alle Zustände besetzt sind. Die allermeisten Elektronen können gar nicht streuen, weil sie dazu einen leeren Endzustand bräuchten, den es nicht gibt. Wir werden aber sehen, dass ein besseres Modell das gleiche Ergebnis liefert.

\section{Drude-Sommerfeld-Modell}

Arnold Sommerfeld und Hans Bethe entwickelten eine verbesserte Theorie. Die Elektronen sind quasi frei, es gilt die Schrödinger-Gleichung und das Pauli-Prinzip. Die Fermi-Fläche sei eine Kugel\sidenote{Was in diesem Kapitel trivial ist, später aber bedeutet, dass nur der Betrag des Wellenvektors $k$ in die Energie eingeht und die Fermi-Fläche den Rand der Brillouinzone nicht berührt.}.
Ohne äußeres Feld fließt im thermischen Gleichgewicht kein Strom, da die Fermikugel um $\mathbf{k} = 0$ zentriert ist. Da wir Isotropie angenommen haben gibt es für jedes Elektron mit $\mathbf{k}$ eines mit $-\mathbf{k}$.

Eine externes Feld $\bm{\mathcal{E}}$ beschleunigt jedes Elektron, ändert also zeitlich kontinuierlich den Impuls der Elektronen. Damit bewegt sich  die gesamte Fermikugel kontinuierlich immer weiter von der Gleichgewichtsposition weg:
\begin{equation}
   \hbar \frac{d \mathbf{k}}{dt} = -e \bm{\mathcal{E}} = \mathbf{F} \quad .
\end{equation}
Streuprozesse können dann aber Elektronen von 'vorne' an der Fermikugel in den frei werdenden Bereich 'hinter' der Kugel umlagern. Die Stöße wirken also rückstellend auf die Bewegung der Fermikugel. Im sich einstellenden Gleichgewicht wird die Fermikugel bei einer mittleren Stoßzeit $\tau$  um 
\begin{equation}
 \delta k = \frac{-e \tau \mathcal{E}}{\hbar}     
\end{equation}  
aus dem Ursprung verschoben sein. Nur der kleine Anteil $\delta k / k_F$ der Elektronen trägt zum Ladungstransport bei. Das sind aber die an der Fermi-Kante, also die schnellsten von allen. Für die Leitfähigkeit ergibt das Sommerfeld-Modell 
\begin{equation}
   \sigma = n e \mu  = \frac{n e^2 }{m} \, \tau(E_F)
\end{equation}
mit $\tau(E_F)$ der Stoßzeit der Elektronen an der Fermi-Kante. Der Unterschied zum Drude-Modell besteht nur in einer etwas anderen Bedeutung zweier Parameter. Damit ist nicht überraschend, dass auch das Drude-Modell die experimentellen Ergebnisse richtig wiedergibt.

Obwohl das Sommerfeld-Modell die gleiche Formel für die Leitfähigkeit liefert, ist die physikalische Interpretation völlig unterschiedlich: Während das Drude-Modell davon ausgeht, dass alle Elektronen gleichermaßen zur Stromleitung beitragen, zeigt das Sommerfeld-Modell, dass nur die Elektronen an der Fermikante tatsächlich relevant sind.

\begin{questions} 
    \item Warum ergeben das Drude-Modell und das Sommerfeld-Modell trotz völlig unterschiedlicher physikalischer Annahmen die gleiche Formel für die elektrische Leitfähigkeit?
\end{questions}



\section{Temperaturabhängigkeit der elektrischen Leitfähigkeit}

Die elektrische Leitfähigkeit $\sigma$ ist temperaturabhängig über die mittlere Stoßzeit $\tau$. Elektronen können mit verschiedenen anderen Partnern streuen (stoßen): mit Phononen, mit Defekten und mit der Probenoberfläche. Dabei addieren sich die Streu-Raten, also die reziproken Stoß-Zeiten. Nur die Streuung an Phononen ist temperaturabhängig. Die anderen Effekte führen zu einem konstanten Wert, der bei tiefen Temperaturen erreicht wird, wenn keine Phononen besetzt sind.

Bereits in Kapitel 1 hatten wir die mittlere freie Weglänge definiert als (Gl. \ref{eq:1_def_weglaenge})
\begin{equation}
   \ell = \frac{1}{n \, \sigma_{st}} = \tau \, v_F \quad ,
\end{equation}
wobei wir hier den Streuquerschnitt $\sigma_{st}$ genannt haben, um ihn von der Leitfähigkeit $\sigma$ zu unterscheiden. $\sigma_{st}$ ändert sich bei diesen Temperaturen kaum, ist also für unsere Zwecke konstant. Elektronen bewegen sich mit der Fermi-Geschwindigkeit $v_F$, die so den Zusammenhang zwischen Stoßzeit und Weglänge herstellt.

Bei einer Temperatur $T$ (viel) größer als der Debye-Temperatur $\Theta$ ändert sich die Dichte $n$ der Phononen wie $T/\Theta$, so dass wir für die Leitfähigkeit $\sigma$ erhalten
\begin{equation}
   \sigma \propto \left\{ 
      \begin{matrix}
         \text{const} & \text{für} \quad T \ll \Theta \\
   \frac{1}{T} &  \text{für} \quad T \gg \Theta \\
      \end{matrix}
   \right.  \quad .
\end{equation}
Bei hohen Temperaturen ist die Leitfähigkeit also völlig durch die Dichte $n$ der Phononen bestimmt.
Der Übergangsbereich ist wie immer aufwändig (Dichte der thermisch angeregten Phononen) und durch das Bloch-Grüneisen-Gesetz beschrieben, das einen $T^{-5}$-Term liefert.\sidenote{Siehe z.B. \cite{Hunklinger2014}}


\begin{questions}
    \item Warum nimmt die elektrische Leitfähigkeit $\sigma$ bei hohen Temperaturen ($T \gg \Theta$) mit $1/T$ ab, obwohl sich die Elektronendichte nicht ändert? Welche Rolle spielt die mittlere Stoßzeit $\tau$ dabei
\end{questions}

\section{Thermische Leitfähigkeit der Elektronen}

Abschließend wollen wir noch die thermische Leitfähigkeit der Elektronen diskutieren, nachdem die anderen Kombinationen aus Elektronen oder Phononen mit Wärmekapazität oder Wärmeleitfähigkeit schon früher besprochen wurden. Im täglichen Leben machen wir die Erfahrung, dass Metalle Wärme besser leiten als Isolatoren. Elektronen scheinen also einen hohen Beitrag zur Wärmeleitfähigkeit zu liefern.

Die  Wärmeleitfähigkeit der Phononen hatten wir in Gl. \ref{eq:2_def_waermeleitf} definiert. Hier gehen wir analog vor:
\begin{equation}
   K  = \frac{1}{3} \, C \, v \, \ell \quad , 
\end{equation}
wobei jetzt alle Größen als elektronische zu verstehen sind, also $K$ die elektronische Wärmeleitfähigkeit, $C$ deren Wärmekapazität, $v$ deren Geschwindigkeit und $\ell =  v_F \tau$ die mittlere freie Weglänge. Wir setzen Gl.~\ref{eq:2_WK_elek} für $C$ ein sowie die Fermi-Geschwindigkeit $v_F$ für $v$  und erhalten
\begin{equation}
   K  =  \frac{1}{3} \frac{\pi^2 \, n \, k_B}{2}  \, \frac{T}{T_F}  \, v_F \, \ell
   =     \frac{\pi^2 }{3} \frac{ n \, k_B^2 \tau}{m}  \, T 
\end{equation}
mit $T_F = m v_F^2 / (2 k_B)$. Die Temperaturabhängigkeit der Stoßzeit $\tau$ der Elektronen mit Phononen muss aber wie oben mit berücksichtigt werden. Abgesehen davon ist die Wärmeleitfähigkeit $K$ proportional zur Temperatur $T$.


\begin{questions}
    \item Warum tragen nur Elektronen nahe der Fermi-Energie $E_F$ zur Wärmeleitung bei, obwohl die Gesamtheit der Elektronen die Energie speichert? Welche Parallelen gibt es zur elektrischen Leitfähigkeit?

\end{questions}

\section{Wiedemann-Franz-Gesetz}

Da in der Temperaturabhängigkeit der elektrischen und auch der thermischen Leitfähigkeit die Temperaturabhängigkeit der Elektron-Phonon-Streuung in der Stoßzeit $\tau$ auftaucht, ist es nicht verwunderlich, dass beide Leitfähigkeiten miteinander in Beziehung stehen. Das ist das Wiedemann-Franz-Gesetz
\begin{equation}
   \frac{K_{el}}{\sigma} = \frac{\pi^2}{3} \, \left( \frac{k_B}{e} \right)^2 \, T = L \, T
\end{equation}
mit der universellen Lorenz-Zahl $L \approx 2.5 \cdot 10^{-8}$~$\Omega$WK$^{-2}$. Gute Wärmeleiter sind also auch gute elektrischer Leiter. In der Realität gewichten die beiden Transportprozesse die Streuung etwas unterschiedlich, so dass es zu Abweichungen bei mittleren Temperaturen kommt, siehe Abb.~\ref{fig:2_Cu_Lorenz}.

Das Drude-Modell sagt das Wiedemann-Franz-Gesetz richtig voraus. Dabei kompensieren sich allerdings der Fehler in der Wärmekapazität der Elektronen mit dem in ihrer Geschwindigkeit\sidenote{siehe \cite{Gross_FK}, Kap. 7.3.2.1}.

\begin{questions}
    \item  Warum bedeutet das Wiedemann-Franz-Gesetz, dass gute elektrische Leiter auch gute Wärmeleiter sind? 
\end{questions}


\newpage
\section{Zusammenfassung}

\textit{Schreiben Sie hier ihre persönliche Zusammenfassung des Kapitels auf. Konzentrieren Sie sich auf die wichtigsten Aspekte und die am Anfang genannten Ziele des Kapitels.}

 \vspace*{10cm}

\printbibliography[segment=\therefsegment,heading=subbibliography]

%\renewcommand{\lastmod}{\today}
\renewcommand{\chapterauthors}{Markus Lippitz}
\renewcommand{\lastmod}{11. Mai  2023}

\chapter{Elektronen in Festkörpern}
\label{chap:bandstruktur}




\section{Ziele}
 


\begin{itemize}  
\item Sie können eine Bandstruktur wie die unten gezeigte erklären und die Schritte zu ihrer Berechnung darstellen. 
\item Sie können die Konzepte 'effektive Masse' und 'Loch' erklären.
\end{itemize}


\begin{figure}
    \inputtikz{\currfiledir Alu_empty_lattice}
    \caption{Die Bandstruktur von Aluminium in der Näherung quasi-freier Elektronen. Nur die Fourier-Koeffizienten $V_g$ wurden als von Null verschieden angenommen.  Das kommt der vollständigen Rechnung (\cite{Segall1961}) schon ziemlich nahe. Wenn das Potential vollständig ignoriert und nur die Periodizität des Kristalls berücksichtigt wird, dann erhält man den gestrichelten Verlauf. Rechnung basierend auf \cite{Polakovic_cmpm3}. \label{fig:3_al_empty_lattice}}
\end{figure} 
  

\section{Überblick}

Die zentrale Frage dieses Kapitels ist, wie das Coulomb-Potential der Atomkerne die Wellenfunktion und insbesondere die Energie-Eigenwerte der Elektronen beeinflusst. Im letzten Kapitel hatten wir schon das freie Elektronengas behandelt. Der Kristall bildet dabei ein großes Kastenpotential, aber die Kerne selbst kommen nicht vor
Wir betrachten nun zwei weitere Fälle: 
\begin{itemize}
    \item die Näherung des (beinahe) leeren Gitters (\textit{empty lattice approximation}): die Kerne liefern ein periodisches, aber ansonsten sehr schwaches Potential
    \item die Näherung stark gebundener Elektronen (\textit{tight binding}): die Kerne liefern ein quasi atomares Potential, mit einer schwachen Möglichkeit, doch ans benachbarte Atom zu wechseln
\end{itemize}

% \begin{questions} 
% \item Wie groß ist ein Molekül?
% \item Welche physikalische Eigenschaft eine Moleküls wird bei Röntgenstreuung, STM und AFM abgebildet?
% \end{questions}
 
% Das Pluto-Skript hydrogen\_wave\_functions\pluto{hydrogen_wave_functions} ermöglicht es Ihnen, mit verschiedenen Varianten der grafischen Darstellung zu experimentieren.



\section*{Schrödinger-Gleichung im reziproken Raum}

Wir wollen  dem Potential der Elektronen etwas mehr Struktur als nur einen Kasten geben. Imn letzten Kapitel hatten wir $V(\mathbf{r}) = \text{const.}$ angenommen. Nun soll das Potential gitterperiodisch sein, also
\begin{equation}
    V(\mathbf{r}) = V(\mathbf{r} + \mathbf{R})
\end{equation} 
mit $\mathbf{R}$ einem Gittervektor im Realraum, also einer ganzzahligen Linearkombination der primitiven Gittervektoren. Dieses Potential können wir  über seine Fourier-Transformation beschreiben
\begin{equation}
    V(\mathbf{r}) = \sum_{\mathbf{G}} V_{\mathbf{G}} \, e^{i \mathbf{G} \cdot \mathbf{r} }
\end{equation} 
mit $\mathbf{G} = \mathbf{G}_{hkl}$ einem reziproken Gittervektor mit den ganzzahligen Koeffizienten $h,k,l$.


Wir machen  für die Wellenfunktion den Ansatz einer Linearkombination von ebenen Wellen mit dem Wellenvektor $\mathbf{k}$
\begin{equation}
    \psi(\mathbf{r}) = \sum_{\mathbf{k}} C_{\mathbf{k}} \, e^{i \mathbf{k} \cdot \mathbf{r} } \quad . \label{eq:2_psi_allg}
\end{equation}
Der Wellenvektor  $\mathbf{k}$ kann wie immer (siehe beispielsweise  Gl.~\ref{eq:2:k_randbed}) nur diskrete Werte annehmen. Anders als bei den Phononen kann er aber auch außerhalb der ersten Brillouin-Zone liegen, da die Elektronen-Wellenfunktion  nicht nur an Gitterpunkten physikalische Bedeutung hat. Elektronen können an 'verschiedeneren' Orten des Kristalls sein als Kerne.

Nun setzen alles in die Schrödingergleichung ein
\begin{eqnarray}
    \hat{H}  \psi(\mathbf{r}) = \left( - \frac{\hbar^2}{2m} \nabla^2 + V(r) \right) \psi(\mathbf{r}) =  & E  \psi(\mathbf{r}) \\
     \left( - \frac{\hbar^2}{2m} \nabla^2 + \sum_{\mathbf{G}} V_{\mathbf{G}} \, e^{i \mathbf{G}  \cdot \mathbf{r} } \right) \sum_{\mathbf{k}} C_{\mathbf{k}} \, e^{i \mathbf{k} \cdot \mathbf{r} } =  & E  \sum_{\mathbf{k}} C_{\mathbf{k}} \, e^{i \mathbf{k} \cdot \mathbf{r} }
\end{eqnarray}
Wir multiplizieren die Klammer aus, leiten dabei die Wellenfunktion ab (was ein $- | \mathbf{k}|^2$ liefert), und benennen übergangsweise den Summationsindex nach $\mathbf{k}'$ um:
\begin{eqnarray}
    \sum_{\mathbf{k}}  \frac{\hbar^2 | \mathbf{k}|^2}{2m} C_{\mathbf{k}} \, e^{i \mathbf{k} \cdot \mathbf{r} } 
    + \sum_{\mathbf{G}, \mathbf{k}'} V_{\mathbf{G}} \, C_{\mathbf{k}'} \, e^{i (\mathbf{G}+\mathbf{k}')  \cdot \mathbf{r} }  
    =  & E  \sum_{\mathbf{k}} C_{\mathbf{k}} \,       e^{i \mathbf{k} \cdot \mathbf{r} } 
\end{eqnarray}
Mit $\mathbf{k} = \mathbf{k}' + \mathbf{G}$ erhalten wir 
\begin{equation}
    \sum_{\mathbf{k}}   e^{i \mathbf{k} \cdot \mathbf{r} } 
    \left[
        \left( \frac{\hbar^2 | \mathbf{k}|^2}{2m} - E \right) C_{\mathbf{k}}
    + \sum_{\mathbf{G}} V_{\mathbf{G}} \, C_{\mathbf{k}- \mathbf{G}} 
    \right]
    = 0  \quad .
\end{equation}
Diese Gleichung muss für jedes $\mathbf{r}$ gelten, also muss der Inhalt der Klammer für jedes $\mathbf{k}$  Null sein
\begin{equation}
        \left( \frac{\hbar^2 | \mathbf{k}|^2}{2m} - E \right) C_{\mathbf{k}}
    + \sum_{\mathbf{G}} V_{\mathbf{G}} \, C_{\mathbf{k}- \mathbf{G}} 
    = 0  \quad . \label{eq:3_SG_rezi}
\end{equation}
Dieser Satz von Gleichung ist die \emph{Schrödinger-Gleichung im reziproken Raum}, für Elektronen in einem gitterperiodischen Potential. Dies ist ein unendlich großes Gleichungssystem, weil die Summe über $\mathbf{G}$ zunächst nicht beschränkt ist. In der Praxis zeigt sich aber, dass bei Coulomb-artigen Potentialen die Fourier-Koeffizienten  $ V_{\mathbf{G}} $ des Potentials schnell mit $|\mathbf{G}|$ abfallen, so dass letztendlich doch nicht so viele Gleichungen gekoppelt sind. 
Auch koppelt das Gleichungssystem nicht alle Koeffizienten $C_i$ miteinander, sondern nur die, die  um einen reziproken Gittervektor $\mathbf{G}$ auseinander liegen, also $C_\mathbf{k}$ mit $C_\mathbf{k \pm G'}$, $C_\mathbf{k \pm G''}$, etc. 

Oben in Gl.~\ref{eq:2_psi_allg} hatten wir die Wellenfunktion $\psi$ als Linearkombination von ebenen Wellen geschrieben. Die Summe lief dabei über den vollständigen  reziproken Raum.
Nun hat sich herausgestellt, dass nur manche Koeffizienten vorkommen, nämlich gerade die, die um $\mathbf{G}$ auseinander liegen.  Wir schreiben stattdessen mit weniger Summanden die Wellenfunktion $\psi$ 
\begin{equation}
    \psi_\mathbf{k}(\mathbf{r}) =  \sum_{\mathbf{G}} C_{\mathbf{k}-\mathbf{G}} \, e^{i (\mathbf{k}- \mathbf{G}) \cdot \mathbf{r} } 
\end{equation}
wobei der Wellenvektor $\mathbf{k}$ jetzt als Quantenzahl gesehen wird. Wir können ihn immer so wählen, dass er in der ersten Brillouinzone liegt.\sidenote{Den Rest kann man immer in die Summe über $\mathbf{G}$ verschieben.} Ebenso bekommt die Energie $E$ aus Gl.~\ref{eq:3_SG_rezi} ein   $\mathbf{k}$ als Index
\begin{equation}
    E_\mathbf{k} = E(\mathbf{k}) \quad .
\end{equation}
Für Werte von  $\mathbf{k}$ außerhalb der ersten Brillouinzone braucht es eine gesonderte Behandlung. Für diese gibt es immer einen  reziproken Gittervektor $\mathbf{G}_n$ so, dass damit $\mathbf{k}$ wieder in die erste Brillouinzone verschoben würde. Wir definieren damit
\begin{equation}
    E_n(\mathbf{k}) = E(\mathbf{k} + \mathbf{G}_n)
\end{equation}
mit $\mathbf{k}$ innerhalb der ersten Brillouinzone.
Der Index  $n$ nennt sich Band-Index und ist so sortiert, dass die Energien $E_n$ aufsteigend sind.

\begin{questions}
    \item Wie kommt es, dass aus einer Schrödingergleichung im Realraum jetzt ein ganzes Gleichungssystem (\ref{eq:3_SG_rezi}) in reziproken Raum wird?
    \item Skizzieren Sie eine Elektronen-Wellenfunktion mit $k$ innerhalb und außerhalb der ersten Brillouinzone.
\end{questions}


\section{Bloch-Theorem}

\begin{marginfigure}
    \inputtikz{\currfiledir fig_bloch_example}
    \caption{Die Wellenfunktion $\psi$ ist das Produkt einer ebenen Welle $e^{i k x}$ mit einer gitterperiodischen Funktion $u_k(x)$. Dargestellt ist jeweils der Realteil. Graue Kreise symbolisieren die Atomkerne. }
\end{marginfigure}


Wir schreiben die Wellenfunktion noch einmal um
\begin{eqnarray}
    \psi_\mathbf{k}(\mathbf{r}) = & \sum_{\mathbf{G}} C_{\mathbf{k}-\mathbf{G}} \, e^{i (\mathbf{k}- \mathbf{G}) \cdot \mathbf{r} }
    = e^{i \mathbf{k} \cdot \mathbf{r} } \,
    \sum_{\mathbf{G}} C_{\mathbf{k}-\mathbf{G}} \, e^{-i \mathbf{G} \cdot \mathbf{r} } \\
    = & u_\mathbf{k}(\mathbf{r}) \; e^{i \mathbf{k} \cdot \mathbf{r} } 
\end{eqnarray}
mit $\mathbf{k}$ aus der ersten Brillouinzone und  einer gitterperiodischen Funktion $u_\mathbf{k}(\mathbf{r})$, die eine ebene Welle $e^{i \mathbf{k} \cdot \mathbf{r} }$ räumlich moduliert. Diese Art der Aufteilung nennt man \emph{Bloch-Wellen}. Das \emph{Bloch-Theorem} besagt, dass Lösungen der Schrödingergleichung in einem periodischen Potential von dieser Form sein müssen. Unsere Rechnung oben ist eine Herleitung des Bloch-Theorems.\sidenote{Für Theoretiker siehe z.B. \cite{Czycholl_theo_FK1}.} Die Aufteilung zwischen ebener Welle und $u(\mathbf{r})$ ist dabei nicht eindeutig, weil immer ein Phasenfaktor zwischen beiden verschoben werden kann.




Bloch-Wellen sind periodisch im reziproken Raum:
\begin{eqnarray}
    \psi_{\mathbf{k} + \mathbf{G}}(\mathbf{r}) = & 
     e^{i (\mathbf{k} + \mathbf{G}) \cdot \mathbf{r} } \, \sum_{\mathbf{G}'} C_{\mathbf{k} +  \mathbf{G} -\mathbf{G}'} \, e^{-i \mathbf{G}' \cdot \mathbf{r} } \\
     = &  e^{i \mathbf{k}  \cdot \mathbf{r} } \, \sum_{\mathbf{G}'} C_{\mathbf{k} - (\mathbf{G}' -\mathbf{G})} \, e^{-i (\mathbf{G}' - \mathbf{G}) \cdot \mathbf{r} } \\
     = &  e^{i \mathbf{k}  \cdot \mathbf{r} } \, \sum_{\mathbf{G}''} C_{\mathbf{k} - \mathbf{G}''} \, e^{-i \mathbf{G}''   \cdot \mathbf{r} } \\
 = &  \psi_{\mathbf{k}}(\mathbf{r})  \quad .
    \end{eqnarray}
Das bedeutet auch, dass $\hbar \mathbf{k}$ nicht als Impuls verstanden werden kann. Noch deutlicher: Bloch-Wellen sind keine Eigenfunktionen des Impuls-Operators. Manchmal bezeichnet man  $\mathbf{k}$ als \emph{Kristallimpuls}, eine Art verallgemeinertem Impuls. Da ein Kristall keine vollständige Translationsinvarianz mehr besitzt, gilt eben auch die Impulserhaltung nur noch eingeschränkt. Das ist hier völlig analog zu der Diskussion bei Phononen. 



\section{Reduziertes Zonenschema}

Zu Demonstrationszwecken machen wir nun eine sehr weitgehende Näherung: wir nehmen wie beim freien Elektronengas an, dass gar kein Potential vorhanden ist, aber die Periodizität des Raums weiterhin erhalten bleibt (ohne zu sagen, was da nun noch periodisch sein soll). Wir wenden also den Formalismus auf $V=0$ an. Damit verschwinden alle Fourier-Koeffizienten $V_\mathbf{G}$ und das Gleichungssystem Gl. \ref{eq:3_SG_rezi} zerfällt in einzelne Gleichungen mit der Lösung
\begin{equation}
    E(\mathbf{k}) = \frac{\hbar^2}{2m} \left| \mathbf{k}  \right|^2 
    \quad \text{und} \quad 
    \psi = e^{i \mathbf{k} \cdot \mathbf{r}} \quad .
\end{equation}
Im Eindimensionalen ist das  wie erwartet ein parabelförmiger Zusammenhang zwischen Wellenvektor und Energie. Wenn wir $\mathbf{k}$ über den ganzen reziproken Raum laufen lassen, nicht nur die erste Brillouinzone, dann nennt man diese Darstellung \emph{ausgedehntes Zonenschema}.

Jetzt nehmen wir die Periodizität hinzu:
\begin{equation}
    E_n(\mathbf{k}) = E(\mathbf{k} + \mathbf{G}_n) 
     = \frac{\hbar^2}{2m} \left| \mathbf{k} + \mathbf{G}_n \right|^2 \quad \text{und} \quad \psi_n = e^{i (\mathbf{k} + \mathbf{G}_n) \cdot \mathbf{r}} \quad .
\end{equation}
Im Eindimensionalen sind das Parabeln, die von jedem Punkt  $\mathbf{G}_n $ im reziproken Raum starten. Diese Darstellung nennt man \emph{periodisches Zonenschema}. Aufgrund der Periodizität ist aber alle Information schon in der ersten Brillouinzone enthalten. Man kann die Darstellung also auf diese beschränken. Dies nennt man  \emph{reduziertes Zonenschema}. Man kann sich das so vorstellen, dass die Parabel ausgehend von  $\mathbf{k} = 0$  an den Grenzen der Brillouinzone zurückgefaltet wird. Wenn sie dann später nochmals auf eine Grenze trifft, wird sie wieder gefaltet.


Im Dreidimensionalen zeichnet man die Energie entlang eines Pfades durch den reziproken Raum, beispielsweise entlang der $x$-Komponente des Wellenvektors. Dann 
kommt zu den Rück-Faltungen noch hinzu, dass es einen Offset aufgrund der $y$ und $z$-Komponenten in $\mathbf{G}_n $ gibt. Dies allein erklärt schon einen großen Teil der Dispersionsrelation, nämlich die gestrichelten Linien in Abb.~\ref{fig:3_al_empty_lattice}.

\begin{figure}
    \inputtikz{\currfiledir fig_zone_scheme}
   \caption{Zonenschemata. Die vertikalen Linien geben die Grenzen der Brillouinzonen wieder (bei ganzzahligen Vielfachen von $\pi/a$, aber nicht $k=0$). \label{fig:3_zone_scheme}}
\end{figure}

\begin{questions}
    \item Skizzieren Sie für ein dreidimensionales kubisch-primitives Gitter die Dispersionsrelation entlang $k_x$ in der Näherung $V \approx 0$.
\end{questions}



\section{Näherung der beinahe freien Elektronen}



Nun wollen wir das Potential hinzunehmen, aber weiterhin als 'schwach' betrachten, eben 'beinahe freie' Elektronen. Im Englischen  nennt man das \emph{empty lattice approximation}. In der Schrödingergleichung \ref{eq:3_SG_rezi} kommen die Fourier-Koeffizienten $V_\mathbf{G}$ des Potentials vor. Im Eindimensionalen ist $G_n = 2 \pi n / a$, mit der Gitterkonstante $a$. Das betragsmäßig kleinste $G$ ist also $g = 2 \pi / a$. Wir nehmen also an, dass nur $V_{\pm g}$ von Null verschieden ist. $V_0$ beschreibt einen konstanten Offset, den wir bei einer Energie-Achse immer zu Null wählen können. Alle höheren $V_G$ sollen der Einfachheit halber ebenfalls Null sein.  Wenn die $V_{\pm g}$ klein sind, dann wir die Lösung ähnlich dem vorangegangenen Abschnitt sein, also zurückgefaltete Parabeln innerhalb der ersten Brillouinzone:
\begin{equation}
    E_k^0 =  \frac{\hbar^2 k^2}{2m} \quad \text{bzw.} \quad    E_{k \pm g}^0 =  \frac{\hbar^2 (k  \pm g)^2}{2m}  \quad .
\end{equation}

Damit wird die Schrödingergleichung \ref{eq:3_SG_rezi} zu
\begin{equation}
   \begin{pmatrix}
    E_k^0 - E &  V_{g} &  V_{-g} \\
    V_{-g}  &   E_{k - g}^0 - E & 0  \\
    V_{g}  &  0 &     E_{k + g}^0 - E  
\end{pmatrix}
  \cdot
  \begin{pmatrix}
    C_{k} \\  C_{k - g} \\  C_{k + g}
  \end{pmatrix}
 = 0  \quad . 
 \label{eq:3_SG_empty_lattice} 
\end{equation}
Die dreidimensionale Lösung dieses Gleichungssystems ist in Abb.~\ref{fig:3_al_empty_lattice} dargestellt und kommt der vollständigen Rechnung schon sehr nahe. Das Pluto-Skript\pluto{al_dispersion} für diese Abbildung ist inspiriert von \cite{Polakovic_cmpm3}.

Interessant wird es an der Grenze der Brillouinzone, also bei $k = \pm g/2$. Dort schneiden sich zwei Parabeln. Die Zustände sind ohne Potential also energetisch entartet. Lassen Sie uns diese Stelle im Fall von  $V_{\pm g} \neq 0$ genauer betrachten. Bei beispielsweise $k = + g/2$  sind die ersten beiden Diagonalelemente von Gl. \ref{eq:3_SG_empty_lattice} nahe Null, das dritte betragsmäßig deutlich größer. Damit die dritte Zeile sich trotzdem zu Null summiert, muss also hier $C_{k+g} \approx 0$ sein. Andersherum: An der rechten Grenze der Brillouinzone schneiden sich die Parabeln, die von $G =0$ und von $G=+g$ ausgehen. Nur diese beiden Koeffizienten $C_{k-G}$ bzw. diese ebenen Wellen tragen bei. Das Gleichungssystem wird noch einfacher 
\begin{equation}
    \begin{pmatrix}
        E_k^0 - E &  V_{g}  \\
     V_{-g}  &   E_{k - g}^0 - E   \\
 \end{pmatrix}
   \cdot
   \begin{pmatrix}
     C_{k} \\  C_{k - g} 
   \end{pmatrix}
  = 0 \quad . \label{eq:3_SG_empty_lattice_2}
 \end{equation}
Wir nehmen weiterhin an, dass das Potential inversionssymmetrisch ist, also $V_g = V_{-g}$. Damit finden wir die Energie-Eigenwerte
\begin{equation}
    E_\pm = \frac{ E_k^0 +   E_{k-g}^0}{2} \mp \sqrt{  \left( \frac{ E_k^0 -   E_{k-g}^0}{2} \right)^2 + V_g^2 }  \quad .
\end{equation}

\begin{marginfigure}
    \inputtikz{\currfiledir bandgap_1d}
    \caption{Das Potential $V_g$ bewirkt eine Bandlücke an der Grenze der Brillouinzone. In der Nähe der Bandlücke verläuft die Dispersionsrelation parabelförmig.}
\end{marginfigure}

An der Grenze der Brillouinzone bildet sich eine \emph{Bandlücke}: während vorher alle Energiewerte (bei entsprechendem $k$) möglich waren, so gibt es nun keine Zustände mehr mit Energien  im Bereich $E_k^0   \pm  V_g$. Die Größe der Bandlücke ist also $2 V_g$ und ihr Auftreten direkt verbunden mit den Fourier-Koeffizienten $V_{\pm g}$ des Potentials.
Die Dispersionsrelation $E(k)$ nähert sich mit horizontaler Asymptote der Grenze der Brillouinzone, also ist die Gruppengeschwindigkeit Null, was einer stehenden Welle entspricht. In erster Näherung ist der Bandverlauf in der Nähe der Bandlücke parabelförmig. 

\begin{questions}
    \item Lesen Sie den Blog-Artikel \cite{Polakovic_cmpm3}. Das sollten Sie jetzt alles gut verstehen.
\end{questions}


\section*{Anschauliche Interpretation I}


Für die  Koeffizienten  $C_{k}$ und $C_{k - g}$ findet man durch Einsetzen der $E_\pm$
\begin{equation}
    \frac{ C_{k-g} }{ C_{k}} = \frac{E - \frac{\hbar^2 k^2}{2m}  }{V_g} \quad .
\end{equation}
Bei $k = g/2$ wird der Zähler also $\pm |V_g|$ und damit die beiden Koeffizienten betragsmäßig gleich.
An dieser Stelle sind die Eigenfunktionen  also
\begin{eqnarray}
    \psi_+ & \propto e^{i g x /2} +  e^{-i g x /2} \propto \cos (gx /2) \\
    \psi_- &\propto e^{i g x /2} -  e^{-i g x /2} \propto \sin (gx /2) \quad .
\end{eqnarray}
Die Ladungsdichte ist das Betragsquadrat der Wellenfunktion. Bei einer ebenen Welle ist die Ladungsdichte räumlich konstant. Durch die Überlagerung zweier ebener Wellen ergibt sich ein Interferenzmuster in der Ladungsdichte
\begin{equation}
    |\psi_+|^2 \propto \cos^2 \left( \frac{\pi x}{a}  \right) \quad \text{und} \quad  |\psi_-|^2 \propto \sin^2 \left( \frac{\pi x}{a}  \right)  \quad .
\end{equation}
Die Atomkerne mit dem zugehörigen attraktiven Coulomb-Potential sitzen bei $x = n a$. Die symmetrische Wellenfunktion $\psi_+$ hat also eine erhöhte Aufenthaltswahrscheinlichkeit in der Nähe der Kerne, und damit eine reduzierte Energie $E_+$. Bei der antisymmetrischen Wellenfunktion $\psi_-$ ist es gerade andersherum. Der symmetrische Fall erinnert an s-Orbitale des Wasserstoff-Atoms, der antisymmetrische an die p-Orbitale.

An der Grenze der Brillouinzone können zwei ebene Wellen also so hybridisieren, dass dadurch für eine der beiden neuen Eigenfunktionen die Energie abgesenkt wird. Im Gegenzug erhöht sich die Eigenenergie der anderen. Dies ist sehr ähnlich dem bindenden und anti-bindenden Potential bei \ch{H2+} in der Molekülphysik. Fern der Grenzen der Brillouinzone sind die Energien der beiden Zustände so unterschiedlich, dass die Beschreibung durch die alten, ungekoppelten Zustände gültig bleibt: Der Term $ E_k^0 -   E_{k-g}^0$ ist viel größer als $V_g$, so dass der Effekt von $V_g$ vernachlässigt werden kann.

Eigenwert-Gleichungen der Form \ref{eq:3_SG_empty_lattice_2} sind ein häufig vorkommendes Motiv in der Physik. Zwei Zustände geringfügig unterschiedlicher Energie koppeln dann zu neuen, hybridisierten Zuständen, wenn ihre Energiedifferenz kleiner als die Kopplungsenergie ist. Wenn man die Energiedifferenz variiert (hier durch Ändern von $k$) dann werden die Energien sich kreuzen (wie im letzten Abschnitt). Die Kopplungsenergie $V_g$ führt zu einer Vermeidung der Kreuzung (engl. \emph{avoided crossing}). Das ist kein quantenmechanischer Effekt. Auch bei den gekoppelten Pendeln sieht man das, wenn man die Eigenfrequenz der Pendel verstimmt, um so die Eigenenergien unterschiedlich zu machen.


\section*{Anschauliche Interpretation II}

Man kann auch aus einem anderen Blickwinkel darauf schauen. Eine ebene Welle $e^{i k x}$, die in positive $x$-Richtung läuft, wird an dem Gitter der Atome mit der Gitterkonstante $a$ gestreut.  In der Laue-Streutheorie entspricht das der Addition eines reziproken Gittervektors $G$, hier der kleinste $g = \pm 2 \pi /a$. Dadurch entsteht eine auslaufende Welle mit dem Wellenvektor $k - g$, also in negative $x$-Richtung laufend.  An der Stelle $k = \pm g/2$, also am Rand der Brillouinzone, hat diese nach links laufende Welle genau den gleichen Wellenvektor, wie die Welle, die zur Parabel bei $G=+g$ gehört. Diese beiden interferieren also konstruktiv miteinander. Anders gesprochen: die Streuung führt bei Erfüllen der Laue-Bedingung $\Delta k = G$ zur Interferenz der beiden Wellen. Dies entspricht dem Koppeln der Zustände wie oben besprochen.


\section*{Näherung stark gebundener Elektronen}

In den letzten Abschnitten haben wir die Bandstruktur, insbesondere die Existenz einer Bandlücke, hergeleitet unter der Annahme, dass die Elektronen quasi frei sind und nur ein schwaches, periodisches Potential wirkt. Nun machen wir genau das Gegenteil und kommen zum gleichen Ergebnis. Wir nehmen an, dass die Elektronen stark gebunden sind, wie in einem Atom, und sie mit geringer Wahrscheinlichkeit zum Nachbaratom tunneln können. Das Modell nennt sich \emph{tight binding} und ist analog zur \emph{linear combination of atomic orbitals} (LCAO) in der Molekülphysik.

Im Folgenden bezeichnet die Tilde Variablen, die für ein einzelnes Atom gelten. Sei also $\tilde{V}$ das Coulomb-artige Potential eines Atoms. Wir kennen die Lösungen $\tilde{\psi}$ der Schrödingergleichung
\begin{equation}
    H_A \tilde{\psi}  = \left( - \frac{\hbar^2}{2m} \nabla^2 + \tilde{V} \right) \tilde{\psi} = \tilde{E} \tilde{\psi} \quad .
\end{equation}
Im Kristall gibt es nun an den Gitterpunkten $\mathbf{R}_m$ Atomkerne, die einen zusätzlichen Störterm im Hamilton-Operator bewirken
\begin{equation}
    H_S = \sum_{m \neq n} \tilde{V}(\mathbf{r} - \mathbf{R}_m) \quad ,
\end{equation}
wobei das Atom bei $\mathbf{R}_n$ schon im ungestörten Operator berücksichtigt ist. Als Ansatz für die Wellenfunktion wählen wir wie in LCAO eine Superposition von Atom-Eigenfunktionen an den Orten $\mathbf{R}_m$, also
\begin{equation}
    \psi = \sum_m a_m \tilde{\psi} (\mathbf{r} - \mathbf{R}_m) \quad .
\end{equation}

Jetzt benutzen wir das Bloch-Theorem\sidenote{Es würde auch ohne gehen, macht es aber hier einfacher.}: der gitter-periodische Anteil $u_\mathbf{k}$ wird durch die Wellenfunktionen $\tilde{\psi} (\mathbf{r} - \mathbf{R}_m)$ geliefert, also muss die ebene Welle in den Koeffizienten $a_m$ stecken, also
\begin{equation}
    a_m \propto e^{i \mathbf{k} \cdot \mathbf{R}_m}
\end{equation}
bei passender Normierung. Somit wird also die Wellenfunktion 
\begin{equation}
    \psi = \frac{1}{\sqrt{N}} \sum_m  \tilde{\psi} (\mathbf{r} - \mathbf{R}_m) \, e^{i \mathbf{k} \cdot \mathbf{R}_m} \quad .
\end{equation}
Die Eigen-Energie ist wie immer
\begin{equation}
    E = \frac{\int \psi^\star \, H \, \psi \, dV }{\int \psi^\star \, \psi \, dV} \quad .
\end{equation}
Der Nenner ist nahe bei Eins, weil die Atom-Wellenfunktionen sich nur sehr wenig überlappen. Der Zähler ist interessanter. Die Summen über die Atom-Positionen können wir vor das Integral ziehen 
\begin{equation}
    E = \frac{1}{N} \sum_{m,n} \, e^{i \mathbf{k} \cdot (\mathbf{R}_m  - \mathbf{R}_n) }\,
    \int \tilde{\psi}^\star (\mathbf{r} - \mathbf{R}_n) \left[ H_A + H_S(\mathbf{r} - \mathbf{R}_m) \right] \tilde{\psi} (\mathbf{r} - \mathbf{R}_m) \quad .
\end{equation}
Wir können drei Beiträge unterscheiden
\begin{itemize}
\item Integranden der Form $\tilde{\psi}^\star_n \, H_A  \, \tilde{\psi}_n$. Das sind die Eigenenergien der Atome, die wir schon kennen.
\item  Integranden der Form $\tilde{\psi}^\star_n \, H_S  \, \tilde{\psi}_n$. Das ist der Einfluss der Potentiale der anderen Atome (in $H_S$) auf 'unser' Atom $n$. Wir kürzen dieses Coulomb-Integral mit $-\alpha$ ab.
\item Integranden der Form $\tilde{\psi}^\star_n \, H_S  \, \tilde{\psi}_m$. Das ist der Einfluss des Überlapps mit den  anderen Wellenfunktionen. 
 Wir kürzen dieses Transfer-Integral mit $-\beta_m$ ab.
\end{itemize}    
Insgesamt haben wir damit\sidenote{Die Summen über $\tilde{E}$ und $\alpha$ liefern ein $N$, was sich mit der Normierung kürzt.}
\begin{equation}
    E = \tilde{E} - \alpha - \sum_m \beta_m  \, e^{i \mathbf{k} \cdot (\mathbf{R}_m  - \mathbf{R}_n) }\, \quad .
\end{equation}
Das Coulomb-Integral $\alpha$ bewirkt einer Energie-Absenkung, weil auch die Nachbar-Atome etwas attraktives Coulomb-Potential beisteuern. Das Transfer-Integral $\beta$ kann sowohl positiv als auch negativ sein, und auch richtungsabhängig, wie bei der kovalenten Bindung in der Molekülphysik. Auch da hatten wir gesehen, dass s- und p-Orbitale je nach Anordnung anziehende oder abstoßende Energiebeiträge liefern können. Genau das gleiche passiert hier. Dieses Integral liefert die Abhängigkeit von Wellenvektor $\mathbf{k}$ und damit die Dispersionsrelation.


\section{Beispiel: kubisch-primitives Gitter}

Als Beispiel betrachten wir ein kubisch-primitives Gitter, erlauben nur Wechselwirkung zwischen nächsten Nachbarn, und nehmen die Wechselwirkung als richtungsunabhängig an, wie sie bei atomaren s-Orbitalen wäre. Für den Term $\mathbf{R}_m  - \mathbf{R}_n$ kommen damit nur die drei (kartesischen) Gittervektoren der Länge $a$ jeweils mit beiden Vorzeichen in Frage. Das Skalarprodukt ausmultipliziert ergibt
\begin{equation}
    E = \tilde{E} - \alpha - 2 \beta \left[ \cos( k_x a ) +  \cos( k_y a ) +  \cos( k_z a ) \right] \quad .
\end{equation}
Insgesamt wird damit ein Band der Breite $\tilde{E} - \alpha \pm  6 \beta $ abgedeckt. Das Band hat einen Kosinus-förmigen Verlauf. Am $\Gamma$-Punkt und an der Grenze der Brillouinzone stimmt es somit mit dem parabelförmigen Verlauf der Näherung quasi-feier Elektronen überein. 


% \section{Beispiel: Diamant und Kaliumchlorid (\ch{KCl}) }

% Sowohl in Diamant als auch in Kaliumchlorid (\ch{KCl}) werden die Bänder durch atomare s- und p-Orbitale gebildet. Diamant ist aber kovalent gebunden, wohingegen Kaliumchlorid ein Ionenkristall ist. Das findet sich auch in der Bandstruktur wieder.

% In Diamant hybridisierend die s- und p-Orbitale zu sp$^3$-Hybrid-Orbitalen. Die ursprünglich $2+4$ Zustände in den s- und p-Orbitalen verteilen sich bei Annäherung der Atome zu einem Kristall auf zwei Bänder mit jeweils 4 Zustände pro Atom. Da Kohlenstoff genau 4 Valenz-Elektronen mitbringt, ist somit das eine Band gefüllt und das andere leer. Der Überlapp der elektronischen Wellenfunktion ist groß, und so sind die Bänder breit (entlang der Energie-Achse).

% In Kaliumchlorid ist der Überlapp der Wellefunktionen kleiner. Die Bänder sind viel schmaler (entlang der Energie-Achse).

% XXX Hunklinger Fig 8.24 und 25


\section{Metalle, Halbleiter und Isolatoren}

Sowohl in der Näherung der schwach gebundenen Elektronen (empty lattice approximation), als auch in der der stark gebunden Elektronen (tight binding) haben wir eine Dispersionsrelation für die Elektronen\sidenote{Evl. besser \emph{das Elektron}, weil wir ja in der Ein-Elektron-Näherung sind.} gefunden, die \emph{Bandlücken} besitzt. Durch diese Lücken entstehen Bänder. Die entscheidende Frage für sehr viele Eigenschaften der Materialien ist nun, bis wohin Elektronen eingefüllt sind. Wo liegt die Fermi-Energie $E_F$, also welches ist das höchste besetzte Niveau bei $T=0$? Wie die Lage von $E_F$ welche Eigenschaften beeinflusst werden wir im nächsten Kapitel besprechen.

Hier klassifizieren wir Materialien nach der Lage der Fermi-Energie $E_F$ relativ zur Bandstruktur. Bei einem Metall liegt $E_F$ in einem Band, bei einem Isolator in einer Bandlücke. Das Band, das vollständig unterhalb von $E_F$ liegt, nennt man Valenzband, das (teilweise) über $E_F$ Leitungsband. Als Halb-Metall bezeichnet man ein Material, bei dem $E_F$ mit einem geringfügigen Überlapp von Bändern zusammenfällt. Ein Halbleiter unterscheidet sich eigentlich nicht von einem Isolator. Im Allgemeinen ist bei Halbleitern die Bandlücke kleiner, so dass Elektronen thermisch ins Leitungsband angeregt werden können. Der Übergang ist aber fließend.





\section{Elektron als Wellenpaket}

Wir hatten oben die Wellenfunktion des Elektrons als (ebene) Bloch-Welle geschrieben
\begin{equation}
   \phi_\mathbf{k}(\mathbf{r}) =  u_\mathbf{k}(\mathbf{r}) \; e^{i \mathbf{k} \cdot \mathbf{r} } \quad ,
\end{equation}
die zwar eine gitterperiodische Modulation $u_\mathbf{k}(\mathbf{r})$ besitzt, aber ansonsten räumlich unendlich ausgedehnt ist. Das ist für Transportprozesse ungeeignet. Das Teilchen-Bild ist besser an die Fragestellung angepasst. Wir bauen also aus verschiedenen Bloch-Wellen ein Wellenpaket, das sich dann wie ein Teilchen benimmt:
\begin{eqnarray}
   \psi_{n, \mathbf{k}}(\mathbf{r},t) =  & \sum_{\mathbf{k}'}  a(\mathbf{k}') \, \phi_\mathbf{k'}(\mathbf{r}) \, e^{-i E_n(\mathbf{k}') t / \hbar} \\
   = & \sum_{\mathbf{k}'}  a(\mathbf{k}') \,  u_\mathbf{k}(\mathbf{r}) \, e^{i (\mathbf{k}' \cdot \mathbf{r}  - E_n(\mathbf{k}') t / \hbar )} \quad ,
\end{eqnarray}
wobei die Summe in $\mathbf{k}'$ über das Intervall $\mathbf{k} \pm \delta \mathbf{k} / 2$ läuft. $E_n(\mathbf{k}')$ ist die Dispersionsrelation des $n$-ten Bandes.
Die Idee ist, dem Wellenpaket des Elektrons eine Gruppengeschwindigkeit $\mathbf{v}_n$ zuzuordnen, über
\begin{equation}
   \mathbf{v}_n = \frac{1}{\hbar} \frac{\partial E_n(\mathbf{k})}{\partial \mathbf{k}} \quad . \label{eq:3_v_gruppe}
\end{equation}
Dazu muss aber $\mathbf{k}$ gut genug definiert sein, also $\delta \mathbf{k}$ klein genug, insbesondere gegenüber der Größe der Brillouinzone. Das geht dann, wenn der Gitterabstand im Realraum viel kleiner ist als die Ausdehnung des Wellenpakets, die wiederum viel kleiner sein muss als die Wellenlänge des externen elektrischen Feldes. Dies nennt man ein semi-klassisches Modell. Da sich der Bandindex $n$ in unserem Modell nicht ändern kann lassen wir ihn zukünftig weg.


\section{Effektive Masse}

Die Bewegungsgleichung ist klassisch $\mathbf{F} = \dot{\mathbf{p}}$. Analog gilt hier
\begin{equation}
   \hbar \frac{d \mathbf{k}}{dt} = \mathbf{F}(\mathbf{r}, t) = q \left[ \bm{\mathcal{E}}(\mathbf{r}, t) +   \mathbf{v}(\mathbf{k}) \times \mathbf{B}(\mathbf{r}, t)\right]
   \label{eq:2_Bewegungsgleichung}
\end{equation}
wobei $\hbar \mathbf{k}$ als Quasi-Impuls zu verstehen ist, also jederzeit ein reziproker Gittervektor $\mathbf{G}$ addiert werden kann. Die Ladung des Elektrons ist $q = -e$.
Eine externe Kraft führt also zu einer Bewegung des Elektrons im reziproken Raum.


Nun beschaffen wir uns einen Term für die Masse des so beschriebenen Elektrons. Wir starten mit der zeitlichen Ableitung von Gl.~\ref{eq:3_v_gruppe}
\begin{equation}
  \frac{\partial \mathbf{v}}{\partial t} = \frac{1}{\hbar}   \frac{\partial^2 E(\mathbf{k})}{\partial t \partial \mathbf{k}} 
  = \frac{1}{\hbar^2}   \frac{\partial^2 E_n(\mathbf{k})}{\partial \mathbf{k} \partial \mathbf{k}} \, \hbar \dot{\mathbf{k}}
  = \frac{1}{\hbar^2}   \frac{\partial^2 E_n(\mathbf{k})}{\partial \mathbf{k} \partial \mathbf{k}} \, \mathbf{F}
\end{equation}
wobei wir im zweiten Schritt Zähler und Nenner mit $\partial \mathbf{k}$ erweitert und dann die Ableitungen umsortiert haben. Das können wir in den cartesischen Komponenten schreiben
\begin{equation}
   \dot{v}_i =  \frac{1}{\hbar^2} \sum_j  \frac{\partial^2 E(\mathbf{k})}{\partial k_i \partial k_j} \, F_j
= \sum_j \left( \frac{1}{m^\star} \right)_{ij} \, F_j
\quad \text{mit} \quad 
\left( \frac{1}{m^\star} \right)_{ij}  =  \frac{1}{\hbar^2} \, \frac{\partial^2 E(\mathbf{k})}{\partial k_i \partial k_j}
\end{equation}
Den Tensor $m^\star$ nennt man \emph{effektive Masse}. Er beschreibt die dynamischen Eigenschaften des Elektrons im Kristall und ist gegeben durch die Krümmung (also zweite Ableitung) der Dispersionsrelation. Der Tensor ist symmetrisch und lässt sich auf drei Hauptachsen transformieren. Wenn die alle gleich groß sind, dann vereinfacht sich alles zu
\begin{equation}
   m^\star(k) = \frac{\hbar^2}{\partial ^2 E(k)/ \partial k^2} \quad .
\end{equation}
Das ist beispielsweise in der Nähe der Ober- und Unterkanten eines Bandes der Fall, wo wir die Dispersionsrelation nähern können als
\begin{equation}
   E_\mathbf{k} = E_0 \pm \frac{\hbar^2}{2 m^\star} \left(k_x^2 + k_y^2 + k_z^2 \right) \quad . \label{eq:3_parabel_band}
\end{equation}
Die effektive Masse ist also der Parameter, der die (inverse) Breite eines parabelförmigen Bandes beschreibt. Stark gekrümmte Bänder haben eine geringe effektive Masse, und andersherum.

Was ist hier passiert? 
Wir haben den Einfluss der vielen Coulomb-Potentiale um die Atomkerne im Kristall zusammengefasst und in eine Eigenschaft des Elektrons gesteckt. Die (effektive) Masse des Elektrons hängt nun vom Kristall ab, in dem es steckt. Das erlaubt es uns aber, das komplizierte Problem 'Elektron zwischen vielen Kernen' in ein einfaches Teilchen-Modell zu überführen. Nur müssen wir uns von der klassischen Vorstellung einer Masse lösen. Die effektive Masse hängt von der Krümmung des Bandes ab. Die kann aber Null werden (die Masse dadurch unendlich), oder auch negativ. Im 'tight binding' Modell ist die Dispersionsrelation
\begin{equation}
   E(k) = - \cos ( k a ) \quad \text{und somit} \quad    m^\star(k) = \frac{\hbar^2}{a^2 \cos ( k a )}
\end{equation}
an der Grenze der Brillouinzone ($k = \pi / a$) ist also $m^\star = - \hbar^2 / a^2$. Die Masse ist nur der Proportionalitätsfaktor zwischen Kraft und Beschleunigung. Hier wirkt die Kraft in eine Richtung, die Beschleunigung geht in die entgegengesetzte.


\section{Nebenbemerkung: Bloch-Oszillationen}

Das sich ändernde Vorzeichen der Masse führt zu Bloch-Oszillationen, zumindest in idealen Systemen. Ein konstantes äußeres elektrisches Feld führt via  \ref{eq:2_Bewegungsgleichung} zu einer gleichmäßigen Bewegung des Elektrons im reziproken Raum. Dabei werden auch Grenzen der Brillouinzonen überschritten. Alles ändert sich also periodisch, und damit auch die Geschwindigkeit und Bewegungsrichtung der Elektronen. Das sind die Bloch-Oszillationen. Die Periodendauer $T_B$ lässt sich ausrechnen als die Zeit, die das Elektron braucht, einmal die Brillouinzone zu durchqueren, also
\begin{equation}
   T_B = \frac{2\pi / a}{e \mathcal{E} / \hbar} = \frac{h}{a e  \mathcal{E}} \quad .
\end{equation}
Bei einem elektrischen Feld $ \mathcal{E} =1 $~kV/m und einer Gitterkonstante $a= 2$~\AA\ ergibt sich eine Periodendauer $T_B = 20$~ms. Bei einer Fermi-Geschwindigkeit $v =10^6~$m/s beträgt die räumliche Amplitude der Bewegung etwa 5~mm. 

Dieser Effekt tritt so nicht auf. Elektronen stoßen selbst in reinen Kristallen typischerweise alle 10~fs oder 10~nm Weglänge. Selbst in idealen Kristallen würde das Elektron noch mit anderen Elektronen stoßen. Man kann aber Modellsysteme herstellen, in denen Bloch-Oszillationen gezeigt wurden. Dazu baut man die Potentiale der Atome durch Kastenpotentiale in Halbleiter-Heterostrukturen auf einer größeren Längenskala nach. Damit wird die Gitterkonstante $a$ viel größer und die Periodendauer $T_B$ und somit auch die Auslenkung viel kleiner.



\section{Löcher}

Wir hatten schon den Effekt von Coulomb-Potentialen in die Masse der Elektronen verschoben. Jetzt werden wir fehlende Elektronen mit einer positiven Ladung versehen und auch als Teilchen behandeln. Unser Ziel ist es, den Ladungstransport in einem Material zu beschrieben, dessen Bandstruktur wir kennen. Jedes Elektron trägt dazu mit seiner Geschwindigkeit (Gl.~\ref{eq:3_v_gruppe}) bei. Wir integrieren einfach über alle Elektronen, um die Stromdichte\sidenote{Strom pro Fläche} $\mathbf{j}$ zu bekommen
\begin{equation}
   \mathbf{j} = \frac{-e}{V} \int_\text{1.BZ} \frac{2V}{(2\pi)^3} \, \mathbf{v}(\mathbf{k}) \, f(E,T) \, d\mathbf{k}
   =  \frac{-e}{4 \pi^3} \int_\text{1.BZ}  \mathbf{v}(\mathbf{k}) \, f(E,T) \, d\mathbf{k}
\end{equation}
mit dem Kristall-Volumen $V$ und der Fermi-Dirac-Verteilung $f(E,T)$. Der erste Term im ersten Integral ist die Zustandsdichte im reziproken Raum. Wir vereinfachen die Gleichung weiter, indem wir $T=0$ annehmen, wodurch die Fermi-Verteilung eine Stufenfunktion wird
\begin{equation}
   \mathbf{j} =  \frac{-e}{4 \pi^3} \int_\text{besetzt}   \mathbf{v}(\mathbf{k}) \, d\mathbf{k} 
   =  \frac{-e}{4 \pi^3 \hbar} \int_\text{besetzt}  \nabla_\mathbf{k} \, E(\mathbf{k}) \, d\mathbf{k}  \quad .
\end{equation}
Dieses Integral wird Null, wenn das Band vollständig besetzt ist. Dann gibt es aufgrund der Punktsymmetrie des reziproken Gitters für jedes Elektron mit $\mathbf{k}$ eines mit $-\mathbf{k}$, das also bei ansonsten gleicher Energie und Geschwindigkeit in die entgegengesetzte Richtung läuft. Ein voll besetztes Band trägt nicht zum Ladungstransport bei! Damit können wir aber schreiben
\begin{eqnarray}
   \mathbf{j} = &  \frac{-e}{4 \pi^3} \int_\text{besetzt}   \mathbf{v}(\mathbf{k}) \, d\mathbf{k}  \\
  =  &  \frac{-e}{4 \pi^3} \left[  \int_\text{1.BZ}   \mathbf{v}(\mathbf{k}) \, d\mathbf{k} -  \int_\text{leer}   \mathbf{v}(\mathbf{k}) \, d\mathbf{k} \right]
  =  &  \frac{+e}{4 \pi^3}  \int_\text{leer}   \mathbf{v}(\mathbf{k}) \, d\mathbf{k} 
\end{eqnarray}
weil eben das Integral über die komplette erste Brillouinzone Null ergibt. Anstatt über alle besetzten Zustände zu integrieren und mit einer negativen Ladung zu multiplizieren können wir auch über alle unbesetzten Zustände integrieren und mit einer positiven Ladung multiplizieren. Wir nennen die unbesetzten Zustände für Elektronen \emph{Löcher} und weisen ihnen eine positive Elementarladung zu.

Das hat große Vorteile, weil wir oft über nahezu volle Bänder integrieren müssen. Dann ist es einfacher, nur über den leeren Anteil der Löcher zu integrieren, insbesondere auch, weil wir dann in der Nähe des Maximums des Bandes\sidenote{Löcher sind wie Luftblasen und steigen nach oben} das Band wie in GL.~\ref{eq:3_parabel_band} parabolisch nähern können.

Um Elektronen von Löchern zu unterscheiden kennzeichnen wir erstere mit dem Index 'n', letztere mit einem 'p'. Weil $\int_\text{voll} \mathbf{k} d\mathbf{k} = 0$ ist ein Loch immer das Gegenteil des Elektrons, dessen Stelle es einnimmt, also
\begin{equation}
   \mathbf{k}_p = - \mathbf{k}_n \quad \quad E_p(\mathbf{k}) = - E_n(\mathbf{k}) \quad \quad m^\star_p = - m^\star_n
   \quad \quad q_p = - q_n = + e
\end{equation} 
aber
\begin{equation}
   \mathbf{v}_p (\mathbf{k}) = \mathbf{v}_n (\mathbf{k}) 
\end{equation}
weil dabei zweimal das Vorzeichen gewechselt wird. Ein Loch wirkt wie ein Teilchen mit positiver Ladung. Bei einem angelegten elektrischen Feld haben wir also Ladungstransport durch Elektronen, die sich in die eine Richtung bewegen, und Löcher, die sich in die andere Richtung bewegen. Die Summe aus beiden bildet den beobachteten Strom.


\newpage
\section{Zusammenfassung}

\textit{Schreiben Sie hier ihre persönliche Zusammenfassung des Kapitels auf. Konzentrieren Sie sich auf die wichtigsten Aspekte und die am Anfang genannten Ziele des Kapitels.}

 \vspace*{10cm}

\printbibliography[segment=\therefsegment,heading=subbibliography]

%\renewcommand{\lastmod}{\today}
\renewcommand{\chapterauthors}{Markus Lippitz}
\renewcommand{\lastmod}{19. Juli 2023} 

\chapter{Kristall-Elektronen im Magnetfeld}

\label{chap:magnetic_field}


\section{Ziele}
 
\begin{itemize}
\item Sie können  Methoden zur Vermessung der Fermi-Oberfläche wie beispielsweise die Zyklotron-Resonanz oder die De Haas-van Alphén-Oszillationen erklären und auf einfache Beispiele anwenden.
\item Sie können das Konzept der Landau-Niveaus benutzen, um den hier dargestellten Quanten-Hall-Effekt phänomenologisch zu erklären.
\end{itemize}


\begin{figure}
    \inputtikz{\currfiledir qhe_gaas}
    \caption{Quanten-Hall-Effekt in einem effektiv zweidimensionalen Elektronengas (\cite{Klitzing1984}). Der Querwiderstand $\rho_{xy}$ nimmt nur diskrete Werte 25812~$\Omega / p$ an, wobei $p$ die Anzahl der voll besetzen Landau-Niveaus ist. An diesen Plateaus verschwindet der Längswiderstand $\rho_{xx}$.}
    \label{fig:4_qhe_gaas}
\end{figure}
 

\section{Überblick}

Im letzten Kapitel haben wir die Bandstruktur diskutiert, also die möglichen Zustände eines Elektrons als Funktion der Quantenzahlen Wellenvektor $\mathbf{k}$ und Band-Index $n$. Ein externes Magnetfeld erlaubt einerseits, bestimmte  Eigenschaften der Bandstruktur zu bestimmen. Daher können wir dieses Kapitel auch sehen als experimentelle Überprüfung des vorangegangenen. Andererseits führt ein starkes Magnetfeld aber auch zu einer zusätzlichen Quantisierung der Elektronen-Zustände.

Eigentlich bräuchte man für diese Phänomene die zeitabhängige Schrödingergleichung. Dies wird aber zu aufwändig. Daher machen wir semi-klassische Modelle und beschreiben die Zustände der Elektronen im Rahmen der Quantenmechanik, den Einfluss äußerer Kräfte aber klassisch.

% \begin{questions} 
% \item Wie groß ist ein Molekül?
% \item Welche physikalische Eigenschaft eine Moleküls wird bei Röntgenstreuung, STM und AFM abgebildet?
% \end{questions}
 


% Das Pluto-Skript hydrogen\_wave\_functions\pluto{hydrogen_wave_functions} ermöglicht es Ihnen, mit verschiedenen Varianten der grafischen Darstellung zu experimentieren.




\section{Fermi-Flächen}


Bevor wir zur Bewegung der Elektronen unter Einfluss eines magnetischen Felds übergehen, müssen wir  die Form der Fermi-Flächen diskutieren, die dabei eine Rolle spielen werden. Wir hatten im vorletzten Kapitel gesehen, dass bei einem freien Elektronengas und $T=0$ alle Zustände unterhalb einer charakteristischen Energie $E_F$ und damit auch unterhalb eines charakteristischen Wellenvektors $k_F = (3 \pi^2 n)^{1/3}$ vollständig besetzt sind. Im reziproken Raum entspricht dies einer Kugel mit dem Radius $k_F$, der Fermi-Kugel.

Die Bandstruktur führt dazu, dass die Form von einer Kugel abweichen wird. Wir bezeichnen dann als \emph{Fermi-Fläche} die Fläche im reziproken Raum, die genau alle besetzten Zustände bei $T=0$ einschließt. Die möglichen Werte des Wellenvektors $\mathbf{k}$ sind diskret. Die Gesamtzahl der  $\mathbf{k}$-Werte in der ersten Brillouinzone entspricht der Anzahl primitiver Einheitszellen im Kristall. Weil ein Elektron zwei verschiedene Spin-Eigenwerte annehmen kann, gibt es in der ersten Brillouinzone also doppelt so viele Zustände für Elektronen wie Atome im Kristall, falls nur ein Atom in der Einheitszelle ist.

Falls jedes Atom ein Elektron zum quasi-freien Elektronengas beisteuert, dann ist das Band nur halb gefüllt und damit $k_F$ knapp  unterhalb der Grenze der Brillouinzone bei $\pi / a$. Wenn die Bandstruktur isotrop ist, dann nimmt die Fermi-Fläche auch hier die Form einer Kugel an. Ansonsten 'verbeult' die Form etwas. Beispiele sind die Alkali-Metalle.

Bei zwei Elektronen pro Atom würden zwar genügend Zustände innerhalb der ersten Brillouinzone existieren. Bei einer isotropen Bandstruktur ist der Kugel-Radius $k_F$ aber größer als $\pi /a$. Eine Kugel reicht etwas über den Quader gleichen Volumens hinaus. Im reduzieren Zonenschema findet man also freie Zustände in den Ecken des ersten Bandes und besetzte Zustände in Taschen an den Kanten des zweiten Bandes (siehe Abbildung~\ref{fig:4_fermi_sphere_BZ_sketch}), die man sich durch Rückfaltung entstanden vorstellen kann.

\begin{marginfigure}
   \inputtikz{\currfiledir fermi_surface_2d}
   \caption{Wenn die Fermi-Kugel die Grenze der ersten Brillouinzone erreicht entstehen freie Zustände in den Ecken des ersten Bandes und besetzte Zustände in Taschen an den Kanten des zweiten Bandes. \label{fig:4_fermi_sphere_BZ_sketch}}
\end{marginfigure}


An der Grenze der Brillouinzone, wo diese Rückfaltung stattfindet, entsteht durch das periodische Potential aber auch die Bandlücke. Dort ist die Gruppengeschwindigkeit Null, also 
\begin{equation}
 \mathbf{v}_g = \frac{\partial \omega}{\partial \mathbf{k}} = \frac{1}{\hbar} \, \nabla_\mathbf{k} E = 0  \quad .
\end{equation}
Der Gradient der Energie verläuft also parallel zur Grenze der Brillouinzone und damit die Linien konstanter Energie und so die Fermi-Fläche senkrecht auf die Grenze der Brillouinzone. Dies führt lokal zu einer Abweichung von der (zurückgefalteten) Kugelform. 

Im ausgedehnten Zonenschema findet man so durch 'Hälse' verbundene Flächen, die sich periodisch wiederholen.


\begin{figure}
   \includegraphics*[width=30mm]{\currfiledir fermi-surfaces/K.jpg}
  \hspace*{4mm}
   \includegraphics*[width=30mm]{\currfiledir fermi-surfaces/Ca.jpg}
   \hspace*{4mm}
   \includegraphics*[width=30mm]{\currfiledir fermi-surfaces/Cu.jpg}

   \caption{Fermi-Oberflächen von \ch{K}, \ch{Ca} und \ch{Cu} (von links). \cite{Choy00_fermi_surfaces} }
\end{figure}
    

\begin{questions}
   \item Machen Sie sich klar welche Formen von Fermi-Flächen durch den Kontakt mit der Grenze der Brillouinzone entstehen können, wenn man das periodische Zonenschema benutzt.
\end{questions}

\section{Zyklotron-Resonanz}

Die Zyklotron-Resonanz ist eine experimentelle Methode, mit der die Form der Fermi-Fläche bestimmt werden kann. Man legt ein statisches Magnetfeld $\mathbf{B}$ an und bestimmt die Absorption einer Radiowelle variabler Frequenz.\sidenote{Eigentlich macht man es genau andersherum: konstante Frequenz, variables Feld. Aber sorum erklärt es sich einfacher.} Man findet charakteristische Frequenzen, bei denen die Absorption besonders hoch ist. Aus diesen Zyklotron-Resonanzfrequenzen kann man den Umfang der Fermi-Fläche geschnitten mit einer Ebene senkrecht zum Magnetfeld bestimmen.


Betrachten wir zunächst klassische freie Elektronen im Magnetfeld. Die Bewegungsgleichung ist 
\begin{equation}
   \hbar \frac{d \mathbf{k}}{dt} = \mathbf{F}(\mathbf{r}, t) = q \left[ \bm{\mathcal{E}}(\mathbf{r}, t) +   \mathbf{v}(\mathbf{k}) \times \mathbf{B}(\mathbf{r}, t)\right]  \quad .
\end{equation}
Wir haben hier kein elektrisches Feld und setzten $\mathbf{B} = B_0 \hat{\mathbf{e}}_z$. Damit ist die Bewegung in $z$-Richtung frei und wir erhalten ein Gleichungssystem für die Bewegung in der $xy$-Ebene
\begin{eqnarray}
   m \dot{v}_x & = -e B_0 \, v_y \\
   m \dot{v}_y & = e B_0 \, v_x  \quad .
\end{eqnarray}
Die Lösung ist eine kreisförmige Bahn mit der Zyklotron-Frequenz $\omega_c$
\begin{equation}
   \omega_c = \frac{e \, B }{m} \quad .
\end{equation}

Im Kristall wird nun aus der Geschwindigkeit $\mathbf{v}$ die Gruppengeschwindigkeit $\mathbf{v}_g$ und die Bewegungsgleichung
\begin{equation}
   d\mathbf{k} = - \frac{e}{\hbar^2} \left[ \nabla_\mathbf{k}\, E(\mathbf{k}) \times \mathbf{B} \right] \, dt  \quad .
\end{equation}
Die Änderung des Wellenvektors steht also senkrecht auf dem Magnetfeld und dem Gradienten der Energie. Das Elektron bewegt sich auf einer Bahn, die senkrecht auf dem Magnetfeld steht und entlang konstanter Energie $E(\mathbf{k}) = \text{const.}$ geht. Wir führen einen Vektor $d\mathbf{k}_\perp$ ein, der senkrecht auf $\mathbf{B}$ und  $d\mathbf{k}$ steht. Seine Länge ist $dk_\perp$. Damit erhalten wir wir 
\begin{equation}
   |d\mathbf{k}|  =  \frac{e}{\hbar^2} \, B \, |d E /dk_\perp | \, dt  \quad .
\end{equation}
%
\begin{marginfigure}
   \inputtikz{\currfiledir dsde_sketch}
   \caption{Die Fläche $dS$ ist ein Kreisintegral über $|d\mathbf{k}|$ mit der 'Dicke' $d\mathbf{k}_\perp$.}
\end{marginfigure}

Die Umlaufzeit $T$ ist
\begin{equation}
   T = \oint dt =  \frac{\hbar^2}{e B } \oint \frac{ |d\mathbf{k}|  }{|d E /dk_\perp |}
   =  \frac{\hbar^2}{e B } \oint \frac{dk_\perp  |d\mathbf{k}| }{d E } 
   = \frac{\hbar^2}{e B } \, \frac{dS}{dE} \label{eq:4_t_zykl}
\end{equation}
mit der Änderung der umschlossenen Fläche im reziproken Raum 
\begin{equation}
   dS = \oint dk_\perp  |d\mathbf{k}|  \quad .
\end{equation}


Beim freien Elektron ändert sich die Länge des Wellenvektors $k$ nicht. Die umschlossene Fläche ist also $S = \pi k^2$ und die Energie $E = (\hbar k)^2 / 2m$. Damit wird 
\begin{equation}
   \frac{dS}{dE} = \frac{2 \pi m}{\hbar^2} \quad \text{und} \quad  \omega_c = \frac{2\pi}{T} =  \frac{e \, B }{\tilde{m}}  \quad .
 \end{equation}
Nur im Fall freier Elektronen ist die Masse $\tilde{m}$ die Ruhemasse des Elektrons. Ansonsten behält man aber diesen Zusammenhang bei und steckt das Ergebnis von Gl.~\ref{eq:4_t_zykl} in die 
 \emph{Zyklotron-Masse}, die sich von der effektivem Masse $m^\star$ unterscheidet.

Bislang haben wir zwar Kreisfrequenzen, aber noch keine Absorption. Alle Elektronen bewegen sich auf geschlossenen Bahnen, nehmen dabei aber keine Energie auf, weil ja gerade $E (\mathbf{k}) = \text{const.}$ gilt. Nun strahlen wir Mikrowellen der Frequenz $\omega_{RF} = \omega_c$ ein. Nur Elektronen in der Nähe der Fermi-Energie also der Fermi-Fläche können Energie aufnehmen. Die Zyklotron-Resonanz misst also die Länge der Bahnen auf der Fermi-Fläche; je nach Orientierung des Magenfeldes laufen die Bahnen in unterschiedlichen Flächen. Scharfe Resonanzen bekommt man nur, wenn die mittlere Stoßzeit groß gegenüber der Umlaufzeit ist, also mehrere Umläufe ungestört absolviert werden können. Man benutzt also hohe Magnetfelder und damit hohe Umlauf-Frequenzen sowie tiefe Temperaturen und reine Substanzen.

Weiterhin tragen nur \emph{extremale Bahnen} bei. Das sind Bahnen, bei denen es viele Bahnen ähnlicher Frequenz gibt,
oder andersherum sich die Frequenz nur geringfügig ändert, wenn die die Komponenten des Wellenvektors parallel zum Magnetfeld sich ändert, die Bahn also stabil gegen Störungen ist. 

\begin{marginfigure}
   \includegraphics*[width=49mm]{\currfiledir sketches/cyclotron_res.png}
   \caption{Nur extremale Bahnen entlang der Fermi-Oberfläche tragen zur Zyklotron-Resonanz bei.}
\end{marginfigure}


\section{Beispiel: Germanium}


Germanium (\ch{Ge}) ist ein indirekter Halbleiter mit kubisch-flächenzentrierter  (fcc) Kristallstruktur, mehr dazu im nächsten Kapitel.  Die Fermi-Energie liegt bei einem Halbleiter in der Bandlücke und eine Fermi-Oberfläche lässt sich so nicht definieren. Wir betrachten darum hier eine Iso-Energie-Fläche für angeregte Elektronen im Leitungsband. Von der Idee her ist das aber identisch mit der Spektroskopie der Fermi-Flächen. Die Bandstruktur von Germanium ist etwas speziell. Insbesondere liegt das Minimum des Leitungsbandes nicht im $\Gamma$-Punkt, sondern an den L-Punkten am Rand der Brillouinzone. Die Iso-Energie-Fläche nehmen die Form von Ellipsoiden an, deren Mittelpunkt an den L-Punkten liegt und deren lange Achse in Richtung $\Gamma$-Punkt zeigt. Es gibt zwar 8 L-Punkte, aber die Hälfte der Ellipsoide liegt außerhalb der ersten Brillouinzone, so dass effektiv nur 4 vollständige Ellipsoide betrachtet werden müssen.

\begin{marginfigure}[-40mm]
   \inputtikz{\currfiledir fcc-3d_2x}
   \caption{Brillouinzone eines fcc Kristalls. }
\end{marginfigure}


\begin{figure}
   \begin{tabular}{ll}
   \inputtikz{\currfiledir germanium_spec}&
   \inputtikz{\currfiledir germanium_mass}
\end{tabular}
\caption{links: Zyklotron-Resonanz-Spektrum von Germanium bei  einem Winkel $\theta = 60^\circ$ zur [001]-Achse. rechts:  Resonanz-Feldstärke $B$ als Funktion von  $\Theta$. \label{fig:4_Germanium_res}}
\end{figure}


Die Form dieser Iso-Energie-Fläche haben \cite{Dresselhaus1955} durch Zyklotron-Resonanz bestimmt. Es ist im Experiment viel einfacher, den Mikrowellengenerator bei konstanter Frequenz zu betreiben und die Amplitude und Richtung des Magnetfelds zu variieren. Es ist also immer 
$\omega_c = 2 \pi \cdot 24$~GHz. Abbildung \ref{fig:4_Germanium_res}(links) zeigt das Resonanz-Spektrum für eine bestimmtem Orientierung des Magnetfelds relativ zur Probe. Man findet fünf Resonanzen, von denen sich die mittleren drei den Elektronen im Leitungsband zuordnen lassen. In der gewählten Anordnung von Probe und Magnetfeld sind zwei der vier Ellipsoide äquivalent, so dass effektiv drei verschiedene Ellipsoide sichtbar sind, die jeder eine Resonanz im Spektrum ergeben.

Dann wird die Richtung des Magnetfelds in der (110)-Ebene variiert.\sidenote{Siehe Festkörperphysik I zur Definiton von Ebenen und Richtungen. Insbesondere benennt man Richtungen in fcc-Kristallen  mit sc-Koordinaten.}  Angegeben ist jeweils der Winkel $\theta$ zur [001]-Achse. Mit dem Winkel ändert sich die Schnittebene, auf der die Zyklotron-Bahnen laufen und somit deren Fläche. Man kann die Ellipsoide durch zwei effektive Massen $m_l$, $m_t$ in longitudinaler und transversaler Richtung beschreiben. Bei einem Winkel $\alpha$ zwischen Magnetfeld und langer Achse des Ellipsoiden ergibt sich eine effektive Masse $m_{ell}$
\begin{equation}
   \left( \frac{1}{m_{ell}} \right)^2 = \frac{\cos^2 \alpha}{m_t^2} + \frac{\sin^2 \alpha}{m_t \, m_l} \quad .
\end{equation}
Diese effektive Masse bestimmt die Lage der Zyklotron-Resonanz. Der gesamte Datensatz in Abb.~\ref{fig:4_Germanium_res}(rechts) kann daher beschrieben werden durch $m_l = 1.58 \, m_e$ und $m_t =  0.082 \, m_e$, wobei $m_e$ die freie Elektronenmasse ist.



\begin{questions}
   \item Wie hängt die effektive Masse mit der  Form der Ellipsoiden zusammen?
\end{questions}




\section{Landau-Niveaus}

Wir haben bei der Zyklotron-Resonanz die Elektronenbahnen klassisch behandelt. In der Quantenmechanik würde man eigentlich eine einzige Wellenfunktion pro Bahn erwarten, also insbesondere, dass sich die Phase nur um Vielfache von $2\pi$ pro Umlauf ändert. Dies führt zu quantisierten Bahnen und einer Aufspaltung der Zustände in Landau-Niveaus. Den Energiebeitrag der Elektronenpins im Magnetfeld vernachlässigen wir.

Wir müssen also das Magnetfeld $\mathbf{B}$ mit in die Schrödingergleichung aufnehmen\sidenote{siehe Kap. 9.3.2 in \cite{Hunklinger2014} und  Kap. 9.6.1 in \cite{yu_cardona}}. Die geschieht über das zugehörige Vektorpotential $\mathbf{A}$ 
\begin{equation}
   \frac{1}{2m} \left( \frac{\hbar}{i} \nabla - q \mathbf{A}  \right)^2 \, \psi = E \psi \quad .
\end{equation}
Wir wählen die Eichung
\begin{equation}
   \mathbf{A} = \begin{pmatrix}
      0 & B x & 0
   \end{pmatrix}
   \quad \text{und somit} \quad
   \mathbf{B} = \begin{pmatrix}
      0 & 0 & B
   \end{pmatrix} \quad .
\end{equation}
Damit ist die Schrödingergleichung separierbar in einen $z$-Anteil und einen für $xy$. Der $z$-Anteil entspricht einem freien Teilchen, also 
\begin{equation}
    \psi(z) \propto e^{ \pm i \, k_z \, z}    \quad \text{und} \quad
    E_z = \frac{\hbar^2}{2m} \, k_z^2 \quad .
\end{equation}
Für den $xy$-Anteil machen wir den Ansatz
\begin{equation}
   \psi(x,y) \propto u(x) e^{i \, k_y \, y}
\end{equation}
und erhalten eine eindimensionale Gleichung für $u(x)$
\begin{equation}
   - \frac{\hbar^2}{2m} \, \frac{\partial^2 u }{\partial x^2} + 
   \frac{m}{2} \left(
 \frac{eB}{m} x - \frac{\hbar k_y}{m}
   \right)^2 \, u = 
   (E - E_z) u \quad .
\end{equation}
Das ist ein eindimensionaler harmonischer Oszillator in einem parabolischen Potential,  dessen Gleichgewichtslage bei 
\begin{equation}
   x_0 = \frac{\hbar^2 k_y}{m \omega_c}
\end{equation}
liegt. Die Eigenfrequenz ist $\omega = \omega_c = e B / m$, mit $m$ wieder der Zyklotron-Masse. 
Die Gesamtenergie hat die Eigenwerte
\begin{equation}
   E = \left( n + \frac{1}{2} \right) \hbar \omega_c + \frac{\hbar^2}{2m} k_z^2 \quad .
\end{equation}
Insgesamt gibt es eine freie Bewegung in $z$-Richtung, also die des Magnetfelds, in eine quantisierte geschlossene Kreisbahn in der $xy$-Ebene. Abbildung \ref{fig:4_dispersion_3d_b} zeigt die Dispersionsrelation entlang $k_z$. In Metallen liegen die Zustände allerdings sehr viel dichter: Bei einem Magnetfeld von 1~T beträgt $\hbar \omega_c$ etwa 0.1~meV, verglichen mit $E_F \approx 10$~eV.

\begin{marginfigure}
   \inputtikz{\currfiledir dispersion_3d_b}
   \caption{Dispersionsrelation entlang $k_z$ ohne Magnetfeld (fett) und nach Quantisierung der Kreisbahnen (dünn). \label{fig:4_dispersion_3d_b}}
\end{marginfigure}

Die $x$ und $y$-Komponente des Wellenvektors sind keine guten Quantenzahlen mehr, also nicht mehr zeitlich konstant, also kein Eigenwert des Hamilton-Operators. Die Energie ist gegeben durch die Komponente $k_\perp$ senkrecht zum Magnetfeld
\begin{equation}
   k_{\perp, n} = \sqrt{ \frac{2m}{\hbar^2} \left( n + \frac{1}{2}\right) \hbar \omega_c} \quad .
\end{equation}
Damit wird die Energie 
\begin{equation}
   E  = \frac{\hbar^2}{2m}  \left( k_{\perp, n}^2 + k_z^2 \right)  \quad .
\end{equation}
%
\begin{marginfigure}
   \inputtikz{\currfiledir landau_sketch_2d}
   \caption{Durch das Magnetfeld in $z$-Richtung ändert sich die Anordnung der Zustände im reziproken Raum $k_x$--$k_y$. \label{fig:4_landau_states_2d}}
\end{marginfigure}
%
In der $xy$-Ebene des reziproken Raums sind die Zustände also konzentrische Kreise, deren Durchmesser proportional zu $\sqrt{n +1/2}$ ansteigt (Abb.~\ref{fig:4_landau_states_2d}). Der Kreis mit der Quantenzahl $n$ hat im reziproken Raum die Fläche
\begin{equation}
S_n = \pi  k_{\perp, n}^2 = \frac{2 \pi e B}{\hbar} \left( n + \frac{1}{2}\right) \quad .  \label{eq:4_Sn}
\end{equation}
Die Fläche zwischen zwei Kreisen ist im reziproken Raum für alle Kreise gleich
\begin{equation}
   \Delta S = S_{n+1} - S_n = \frac{2 \pi e B}{\hbar} \quad . 
\end{equation}
Ohne Magnetfeld ist sind Zustände im reziproken Raum homogen verteilt, die Zustandsdichte konstant. Mit Magnetfeld werden die Zustände so umverteilt, dass sie in der $xy$-Ebene auf Kreisen liegen, und zwar auf jedem Kreis gleich viele. Das sind gerade die, die innerhalb des Kreises bis zum vorangegangen liegen. Der Entartungsgrad $g_e$ lässt sich schreiben als
\begin{equation}
   g_e =  \Delta S \frac{L^2}{(2\pi)^2} = \frac{e B L^2}{h} = \frac{\Phi}{2 \Phi_0} \label{eq:4_landau_entartung}
\end{equation}
mit der Fläche der Probe $L^2$ senkrecht zum Magnetfeld, dem magnetischen Fluss $\Phi = L^2 B$ und dem \emph{magnetischem Flussquant} $\Phi_0$
\begin{equation}
   \Phi_0 = \frac{h}{2 e} \quad .
\end{equation}
Der Begriff des \emph{Flussquants} greift dem Kapitel zur Supraleitung vorweg. Dort wird auch der Faktor 2 klar, weil dort Elektronen gepaart auftreten werden. Man kann die Stärke des Magnetfelds in Anzahl Flussquanten messen (bei gegebener Probengröße). Genau diese Anzahl definiert die Anzahl der Zustände auf den Landau-Kreisen, also deren Entartung.




In $z$-Richtung hat das Magnetfeld keinen Einfluss. Hier liegen die Zustände weiterhin homogen. Es bilden sich  konzentrische Zylinder, die bis zur Fermi-Fläche mit Elektronen besetzt sind. Die Landau-Kreise sind also eigentlich Landau-Zylinder.


\begin{questions}
   \item Skizzieren Sie, wie in 3 Dimensionen die Landau-Zylinder in der Fermi-Kugel liegen, oder suche Sie solch ein Bild im Internet.
\end{questions}

\section{Dimensionalität und Zustandsdichte}


Lassen Sie uns zunächst die Zustandsdichte ohne Magnetfeld aber bei reduzierter Dimensionalität betrachten. Wir nehmen ein freies Elektronengas an. Die Gruppengeschwindigkeit ist $v_g = \hbar k / m$. Damit wird die  Zustandsdichte in $n$ Dimensionen 
\begin{equation}
    D(E)^{(n)} dE = 2 \, \frac{L^n}{\hbar (2 \pi)^n} \, \frac{m}{\hbar k} \,  dE \, \int_{E = \text{const.}} d S_E 
\end{equation}
und somit
\begin{eqnarray}
   D(E)^{(3)} = &  \frac{(2m)^{3/2}}{2 \pi^2 \hbar^3} \, L^3 \,  E^{1/2}  \\
   D(E)^{(2)} = &  \frac{m}{\pi \hbar^2} \, L^2   = \text{const.} \\
   D(E)^{(1)} = &  \frac{1}{\pi \hbar} \, L \, E^{-1/2}    \quad .
\end{eqnarray}
Für punktförmige Objekte, also null Dimensionen erhält man wie für Atome diskrete Zustände
\begin{equation}
   D(E)^{(0)} =  2 \, \delta(E-E_0) \quad .
\end{equation}

Ein angelegtes Magnetfeld lässt eine freie Bewegung nur noch in der $z$-Richtung zu. Damit ändert sich effektiv die Dimensionalität der Probe. Nicht mehr alle drei Dimensionen sind relevant, sondern nur noch eine. Dies sieht man auch in der Zustandsdichte. Die Gesamtenergie jedes Zustands lässt sich aufspalten in die  Energie der Kreisbewegung und die der freien Bewegung
\begin{equation}
   E = E_n + E_z =  \left(n + \frac{1}{2} \right) \hbar \omega_c +  \frac{\hbar^2}{2m} k_z^2 \quad .
\end{equation}
Dann verschiebt $E_n$ die Energieskala, die in $ D(E)^{(1)}$ eingeht. Die Gesamt-Zustandsdichte ist also\footcite{Czycholl_theo_FK1}
\begin{equation}
   D(E)^{(3d + B)} = 
    \frac{(2m)^{3/2}}{4 \pi^2 \hbar^3}
    \, L^3  \, \sum_n \frac{\hbar \omega_c}{ \sqrt{E - E_n}} \, \Theta(E-E_n)
\end{equation} 
wobei durch die Stufenfunktion $\Theta$ nur Summanden mit $E \ge E_n$ beitragen. Abbildung~\ref{fig:4_dos_3d_b} zeigt diese Zustandsdichte im Vergleich zum Fall $B=0$. Die Flächen unter den Kurven sind identisch. Bei kleiner werdendem Magnetfeld wird die Zyklotron-Frequenz $\omega_c$ kleiner und damit der Abstand der Peaks kleiner. Wenn man dann über ein kleines Energieintervall mittelt\sidenote{beispielsweise durch thermische Einflüsse via $k_B T$} gehen beide Fälle ineinander über.

\begin{marginfigure}
   \inputtikz{\currfiledir dos_3d_b}
   \caption{Zustandsdichte in 2 und 3 Dimensionen ohne (fett) und mit (dünn) angelegtem Magnetfeld. \label{fig:4_dos_3d_b}}
\end{marginfigure}


Wenn wir von einer zweidimensionalen Probe ausgehen, dann ist ohne Magnetfeld die Zustandsdichte konstant. Mit Magnetfeld bleibt nur eine äquidistante Reihe von Delta-Funktionen $\delta(E-E_n)$ bei den Zyklotron-Energien $E_n$, da keine freie Bewegung mit variabler Energie mehr möglich ist. 


\section{Innere Energie} 

Die Änderung der Zustandsdichte durch das Magnetfeld hat einen überraschenden Einfluss auf die innere Energie. Wir diskutieren dies  für eine zweidimensionale Probe. Man kann sich den dreidimensionalen Zustandsdichte aber als etwas ausgeschmierte 2d Dichte vorstellen und erwartet somit ähnliche Ergebnisse.

Wenn kein Magnetfeld anliegt, sind alle Zustände bis zur Fermi-Energie $E_F$ gefüllt (bei $T=0$), alle darüber leer. Jetzt schalten wir das Magnetfeld ein und wählen seine Stärke gerade so, dass die Entartung $g_e$ (Gl.~\ref{eq:4_landau_entartung}) passend ist (siehe Abb.~\ref{fig:4_de_haas}), nämlich  die Gesamtzahl $N$ der Elektronen ein ganzzahliges Vielfaches des Entartungsgrads ist
\begin{equation}
   N = p \, g_e \quad \text{mit} \quad p \in \mathcal{N} \quad .
\end{equation}
In diesem Fall ändert sich die innere Energie nicht. Die Zustände ändern zwar ihre Energie, weil alle auf die Landau-Niveaus verschoben  werden, aber gleich viele werden angehoben wie abgesenkt: die bis zum Abstand $\hbar \omega_c/2$ unterhalb des Niveaus werden angehoben; die bis zu diesem Abstand darüber werden abgesenkt.

\begin{figure}
   \includegraphics*[width=100mm]{\currfiledir sketches/de_haas.png}
   \caption{Zustandsdichte in zwei Dimensionen. Bei gewissen Feldstärken $B$ ist die innere Energie größer als ohne Feld. \label{fig:4_de_haas} }
\end{figure}

Falls die Feldstärke $B$ aber anders ist, die Entartung $g_e$ also nicht gerade passt, dann gibt es  ein nur teilweise besetztes Landau-Niveau. Bei schrittweise befüllen dieses letzten Niveaus werden aber zunächst nur Energien angehoben. Erst nach der Halbbesetzung werden Energie abgesenkt. In Summe ist bleibt unabhängig von $B$ immer eine Anhebung übrig.

Die innere Energie oszilliert also mit der Zahl der besetzten Niveaus $p$, wobei ein nicht-ganzzahliges $p$ nun eine teilweise Besetzung bedeutet. Das Minimum der inneren Energie ist bei $E_F(B=0)$.
Weil
\begin{equation}
   E_F \approx p \hbar \omega_c = p \frac{e \hbar B}{m} 
\end{equation}
treten die Minima mit einer Periode des reziproken Feldes   
\begin{equation}
   \delta \left(\frac{1}{B}  \right) = \frac{e \hbar }{m \, E_F}
\end{equation}
auf.

Interessanter ist aber die Größe $S_F$ des höchsten besetzten Landau-Niveaus im reziproken Raum. Dies ist der Schnitt der Fermi-Fläche in  einer Ebene senkrecht zum Magnetfeld und erlaubt und so, die Fermi-Fläche auszumessen. Sei bei einem Feld $B_n$ gerade $n$ Landau-Niveaus voll besetzt. Die Größe $S_F$ des obersten Niveaus im reziproken Raums ist durch Gl. \ref{eq:4_Sn} gegeben:
\begin{equation}
   S_F =  \frac{2 \pi e B_n}{\hbar} \left( n + \frac{1}{2}\right)  \overset{!}{=} \frac{2 \pi e B_{n+1}}{\hbar} \left( n + 1+ \frac{1}{2}\right)  
\end{equation}
so dass
\begin{equation}
   \delta \left(\frac{1}{B}  \right) = \frac{2 \pi e }{\hbar \, S_F} \quad .
\end{equation}

\begin{questions}
   \item Machen Sie sich klar, was in Abb.~\ref{fig:4_de_haas} passiert. Wie ändert sich die Besetzung der Landau-Niveaus und die innere Energie, wenn man die magnetische Feldstärke variiert? Das ist der zentrale Punkt für den Rest des Kapitels.
\end{questions}

\section{De Haas-van Alphén-Effekt}

Viele messbaren Größen hängen mit der  inneren Energie zusammen. Man findet diese Oszillationen mit $1/B$ beispielsweise in der Wärmekapazität  und in der Magnetisierung. Die Wärmekapazität ist
\begin{equation}
   C_v =  \left(\frac{\partial U}{ \partial T} \right)_{V} 
\end{equation}
und die Magnetisierung $M$ ist die Ableitung der freien Energie $F$ bzw. (bei $T=0$) der inneren Energie $U$ nach $B$:
\begin{equation}
   M = - \frac{1}{V} \left(\frac{\partial F}{ \partial B} \right)_{T,V} 
     = - \frac{1}{V} \left(\frac{\partial U}{ \partial B} \right)_{T=0,V} \quad .
\end{equation}
Die Oszillation von $M$ mit $1/B$ nennt man De Haas-van Alphén-Effekt nach W.J. de Haas und P.M. van Alphén.

\begin{figure}
   \inputtikz{\currfiledir beryllium}
   \caption{Relative Änderung der Wärmekapazität von Beryllium (\ch{Be}) mit dem Magnetfeld (\cite{Sullivan68_beryllium}). Das Feld ist parallel zur kristallographischen c-Achse. \label{fig:4_beryllium}}
\end{figure}

Im Fall von Wärmekapazität Beryllium (Abb. \ref{fig:4_beryllium}) misst man bei passender Orientierung des Magnetfeldes die Querschnittstfläche der zigarrenförmigen Fermi-Flächen am Rand der Brillouinzone. 

\begin{marginfigure}
   \includegraphics*[width=40mm]{\currfiledir fermi-surfaces/Be.jpg}
   \caption{Fermi-Flächen von  Beryllium (\cite{Choy00_fermi_surfaces}).}
\end{marginfigure}



\section{Hall-Effekt}

Wir verlassen für einen Augenblick die quantenmechanische Beschreibung und betrachten den klassischen Hall-Effekt. Er erlaubt es, die Ladungsträgerkonzentration zu bestimmen und zwischen Elektronen und Löchern zu unterscheiden. Er dient uns aber auch als Basis für den Quanten-Hall-Effekt im darauffolgenden Abschnitt.

\begin{marginfigure}
   \includegraphics*[width=49mm]{\currfiledir sketches/hall.png}
   \caption{Geometrie zum Hall-Effekt.}
\end{marginfigure}

Wir betrachten die Bewegung eines quasi-freien Elektrons ($q=-e$) unter elektrischen und magnetischen Feldern und berücksichtigen Stöße wie im Drude-Modell. Relevant ist die mittlere Geschwindigkeit $\mathbf{v}_d = \braket{\mathbf{v}}$, bei der also die thermische Bewegung herausgemittelt ist. Für diese gilt
\begin{equation}
   m^\star \frac{d \braket{\mathbf{v} }} {dt} = 
   -e \left[ \bm{\mathcal{E}} + \braket{\mathbf{v}} \times \bm{B} \right] 
   - m^\star \frac{\braket{\mathbf{v}}}{\tau} \quad .
\end{equation}
Das Magnetfeld zeigt in $z$-Richtung und wir sind nur am stationären Fall interessiert ($d \braket{\mathbf{v} } / dt = 0$). Dann erhalten wir
\begin{eqnarray}
   v_{d,x} =& - \frac{e \tau}{m^\star} ( \mathcal{E}_x + v_{d,y} B ) \\
   v_{d,y} =& - \frac{e \tau}{m^\star} ( \mathcal{E}_y - v_{d,x} B ) \\
   v_{d,z} =& - \frac{e \tau}{m^\star} \mathcal{E}_z  \quad .
\end{eqnarray}
Mit der Definition der Stromdichte $\mathbf{j} = - e n \mathbf{v}_d = \bm{\sigma} \bm{\mathcal{E}}$ 
 und der Zyklotron-Frequenz $\omega_c = e B /m^\star$ 
 bekommen wir den Leitfähigkeitstensor $\bm{\sigma}$
\begin{equation}
   \bm{\sigma} = \frac{\sigma_0}{ 1+ \omega_c^2 \tau^2}
   \begin{pmatrix}
      1  & - \omega_c \tau & 0 \\
      \omega_c \tau & 1 & 0  \\
      0 & 0 & 1+ \omega_c^2 \tau^2 \\
   \end{pmatrix} \label{eq:4_sigma_tensor}
\end{equation}
wobei gerade $\sigma_{zz} = \sigma_0 = n e^2 \tau / m^\star $ die Leitfähigkeit bei $B=0$ ist. Man benutzt immer flache Proben, ohne elektrisches Feld oder Stromfluss in $z$-Richtung. Aus Symmetriegründen reicht es aus $\sigma_{xx} = \sigma_{yy}$ und $\sigma_{xy} = - \sigma_{yx}$ zu betrachten. 

Wir machen das Experiment so, dass nur Strom in $x$-Richtung fliesst, also $j_y = 0$. Damit wird 
\begin{equation}
   \mathcal{E}_y = - \omega_c \tau \mathcal{E}_x =  - \omega_c \tau \frac{j_x}{\sigma_0} = R_H \, B \, j_x \quad .
\end{equation}
Das Magnetfeld bewirkt also ein elektrisches Hall-Feld in $y$-Richtung, obwohl der Strom in $x$-Richtung fliesst. Die Hall-Konstante $R_H$ ist also
\begin{equation}
   R_H = \frac{ \mathcal{E}_y }{j_x B} = - \frac{1}{n \, e} \quad .
\end{equation}
Da sich $\mathcal{E}_y$, $j_x$ und $B$ messen lassen, kann man so die Ladungsträgerdichte $n$ bestimmen und auch ihr Vorzeichen. 

\begin{margintable}
   \begin{tabular}[pos]{lrrrr}
      Element  & \ch{Li} & \ch{Na} & \ch{Be} & \ch{Al} \\
      Wertigkeit & 1    &    1    & 2         & 3 \\
      $n$/Atom & 0.8     & 1.2  & -0.4   & -0.9 \\
   \end{tabular}
   \caption{Ladungträgerdichte $n$ bestimmt aus der Hall-Konstanten $R_H$ (aus \cite{Hunklinger2014}). Ein negatives Vorzeichen bedeutet Löcherleitung.}
\end{margintable}

Falls sowohl Elektronen als auch Löcher zum Ladungstransport beitragen, dann werden deren Konzentrationen $n$ und $p$ mit der Beweglichkeit\sidenote{$\mu = e \tau / m$, siehe Gl.~\ref{eq:2_def_beweglichkeit}.} $\mu_n$ und $\mu_p$ gewichtet
\begin{equation}
   R_H =  \frac{p \mu_p - n \mu_n}{e (p \mu_p + n \mu_n)} \quad .
\end{equation}


Analog zum Querwiderstand $\rho_{xy} = R_H \, B$ ist später auch  der Längswiderstand
\begin{equation}
   \rho_{xx} = \frac{\mathcal{E}_x}{j_x} = \frac{B}{n e} \, \frac{1}{\omega_c \tau } = \frac{m^\star}{n e^2 \tau}
\end{equation}
interessant.  


\section{Quanten-Hall-Effekt}

Beim klassischen Hall-Effekt hatten wir gefordert, dass die  Probe flach  ist, also kein Feld oder Stromfluss in $z$-Richtung geschieht. Das verschärfen wir jetzt und fordern eine quantenmechanisch-zweidimensionale Probe, also eine, bei der in $z$-Richtung keine Quantenzahlen mehr relevant sind, insbesondere kein $k_z$. Solche Strukturen nennt man zweidimensionales Elektronengas (2DEG) und kann sie an Grenzflächen zwischen zwei Halbleitern erzeugen.

Wenn wir ein starkes Magnetfeld anlegen und die Proben rein genug sind, so dass $\omega_c \tau \gg 1$, und die Temperatur $T$ kalt genug, dann erhalten wir wie oben besprochen als Zustandsdichte eine äquidistante Reihe von Delta-Funktionen $\delta(E-E_n)$.  Bei manchen Magnetfeldern $B$ sind dann alle Landau-Niveaus vollständig gefüllt und das nächste $\hbar \omega_c$ in der Energie 
  entfernt. Damit sind keine Streuprozesse möglich und $\tau$ wird unendlich. Damit werden im Leitfähigkeitstensor  
  Gl. \ref{eq:4_sigma_tensor} die Einträge $\sigma_{xx} = \sigma_{yy} = 0$. Der Einfachheit halber in zwei Dimensionen also
  \begin{equation}
   \bm{\sigma}_{\tau \rightarrow \infty} =
   \begin{pmatrix}
      0  & -\sigma_\infty  \\
     \sigma_\infty  & 0   \\
   \end{pmatrix} \quad .
\end{equation} 
  Die Kreuz-Terme $\sigma_{xy}$ und $\sigma_{yx}$ bleiben von Null  verschieden, weil auch $\sigma_0 \propto \tau$. Der Widerstandstensor ist das Inverse\sidenote{Man invertiert den Tensor (die Matrix), nicht jedes Element!} des Leitfähigkeitstensors
  \begin{equation}
   \bm{\rho}_{\tau \rightarrow \infty} = 
   \bm{\sigma}_{\tau \rightarrow \infty}^{-1} =
   \begin{pmatrix}
      0  & 1/\sigma_\infty  \\
     -1/\sigma_\infty  & 0   \\
   \end{pmatrix} \quad .
\end{equation}
Das Längsfeld $\mathcal{E}_x$ verschwindet, weil zwar $j_x \neq 0$ aber  $\sigma_{xx} = 0$ und wie immer  $j_y = 0$. Damit wird auch der Längs-Widerstand $\rho_{xx} = 0$.
Das Hall-Feld $\mathcal{E}_y$ bewirkt den Stromfluss in $x$-Richtung.

Das hat Klaus von Klitzing 1980 auch genau  so gefunden\sidenote{siehe \cite{Klitzing1980}. Nobelpreis K. von Klitzing 1985}. Etwas neuere Daten sind in Abbildung~\ref{fig:4_qhe_gaas} gezeigt. An den Stellen, an denen der Längs-Widerstand $\rho_{xx}$ verschwindet, also gerade alle Landau-Niveaus vollständig gefüllt sind, ist die Elektronendichte
\begin{equation}
   n = \frac{N}{L^2} = \frac{p g_e}{L^2} = \frac{p e B}{h}
\end{equation}
mit ganzzahligem $p$. Der Quer-Widerstand wird somit
\begin{equation}
  \rho_{xy} = R_H B =  - \frac{B}{n \, e} = \frac{1}{p} \, \frac{h}{e^2} = \frac{1}{p} \, R_K
\end{equation} 
mit der Naturkonstanten $R_K = 25812,8\dots$~$\Omega$. Dies wird heute als Widerstandsnormal genutzt, und auch im die Sommerfeld'sche Feinstrukturkonstante $\alpha$ zu bestimmen
\begin{equation}
   \alpha = \frac{\mu_0 \, c}{2} \, \frac{e^2}{h} = \frac{\mu_0 \, c}{2} \, \frac{1}{R_K} \approx \frac{1}{137} \quad .
\end{equation}



Überraschend war jedoch, dass sich \emph{Plateaus} ausbilden, also  $\rho_{xx}$  und $\rho_{xy}$ über ein gewisses Intervall von $B$ einen konstanten Wert annehmen. Eigentlich sollte ja nur für genau einen Wert von  $B$ alle Landau-Niveaus gefüllt sein. Der Längswiderstand $\rho_{xx}$ ist in Abbildung~\ref{fig:4_qhe_gaas} aber beinahe durchgängig Null.  Eine einfache phänomenologische Erklärung findet sich in der Zustandsdichte einer realen Probe. In Wirklichkeit wird die nicht allein eine Sequenz von Delta-Funktionen sein. Defekte und Verunreinigungen führen zu einer kontinuierlichen Verteilung von Zuständen auf der Energieskala. Diese Defekt-Zustände sind aber im Gegensatz zu den Landau-Niveaus im Raum lokalisiert und tragen nicht zur Leitfähigkeit bei, aber zur Fermi-Energie. Wenn das Magnetfeld  etwas erhöht wird, sollte in einer idealen Probe sofort das nächst tiefer  Landau-Niveau depopuliert werden, weil die Fermi-Energie direkt zu diesem Niveau springt. In einer realen Probe werden aber zunächst alle Defekt-Zustände dazwischen depopuliert und das Landau-Niveau bleibt vollständig besetzt. Solange es also noch besetzte Defekt-Zustände zwischen zwei landau-Niveaus gibt, bleibt der Quanten-Hall-Effekt bestehen.

Tiefergehende Erklärungen benutzen 'skipping orbits' in Randkanälen (am Rand der Probe). Heutzutage wird der Quanten-Hall-Effekt als ein topologischer Effekt gesehen, der also unter stetigen Verformungen erhalten  bleibt. Er wird auch als ganzzahliger Quanten-Hall-Effekt bezeichnet, weil es auch einen 'fractional quantum Hall effect' gibt, bei dem $p$ rationale Zahlen annimmt. Für diesen haben Robert Laughlin, Horst Störmer und Daniel Tsui 1998 den Nobelpreis bekommen.





\newpage

\section{Zusammenfassung}

\textit{Schreiben Sie hier ihre persönliche Zusammenfassung des Kapitels auf. Konzentrieren Sie sich auf die wichtigsten Aspekte und die am Anfang genannten Ziele des Kapitels.}

\vspace*{10cm}
\printbibliography[segment=\therefsegment,heading=subbibliography]

%\renewcommand{\lastmod}{\today}
\renewcommand{\chapterauthors}{Markus Lippitz}
\renewcommand{\lastmod}{21. März 2025}

\chapter{Halbleiter}

\label{chap:halbleiter}



\section{Ziele}
 


\begin{itemize}
\item Sie können die Ladungsträgerdichten in dotierten und undotierten Halbleitern im Zusammenspiel von Zustandsdichte und Fermi-Dirac-Verteilung erklären und so beispielsweise die unten gezeigten Messungen an Germanium interpretieren.
\item Sie können die Strom-Spannungs-Kennlinie eines p-n-Kontakts und die zugrundeliegende Bandstruktur erklären.
\end{itemize}



\begin{figure}
    \inputtikz{\currfiledir germanium_electron_density}
     \caption{
        Dichte der Elektronen im Leitungsband von Germanium, gemessen mit dem Hall-Effekt, als Funktion der Konzentration der Arsen-Donatoren. Daten aus \cite{Conwell1952}. Zum Vergleich ist $p_i(T) = n_i(T)$ (gestrichelt) und $p(T,n_D)$ (grau) eingezeichnet.
    \label{fig:5_Ge_n_density} 
     }
\end{figure}
 


% \begin{questions} 
% \item Wie groß ist ein Molekül?
% \item Welche physikalische Eigenschaft eine Moleküls wird bei Röntgenstreuung, STM und AFM abgebildet?
% \end{questions}
 


% Das Pluto-Skript hydrogen\_wave\_functions\pluto{hydrogen_wave_functions} ermöglicht es Ihnen, mit verschiedenen Varianten der grafischen Darstellung zu experimentieren.


\section{Überblick}


Wir haben bereits im Kapitel zur Bandstruktur gesehen, dass der Übergang zwischen Isolator und Halbleiter fließend ist. Beide haben ein voll besetztes Valenzband und ein unbesetztes Leitungsband mit der Bandlücke $E_g$. Thermische Anregung führt zu einer Besetzung des Leitungsbandes proportional zu $\exp(E_g / 2 k_b T)$, wie wir weiter unten sehen werden. Beispielsweise werden bei Raumtemperatur je nach Bandlücke $10^{-5}$ (bei $E_g$ = 0.5 eV) oder $10^{-26}$ (bei $E_g$ = 3 eV) aller Ladungsträger angeregt sein. Dies führt zu einer von Null verschiedenen Leitfähigkeit. Als Halbleiter bezeichnet man ein Material, dessen spezifische Leitfähigkeit bei Raumtemperatur zwischen $10^{-2}$ und $10^9$ $\Omega$m liegt. Halbleiter können  nach verschiedenen Eigenschaften klassifiziert werden:


\paragraph*{intrinsisch oder dotiert} Wenn man ein Material gezielt mit einer Verunreinigung versieht, also dotiert, dann kann dies zu mehr freien Ladungsträgern führen, weil diese dann vom Fremdatom stammen. Bei intrinsischen Halbleitern stammen die Elektronen im Leitungsband immer aus dem Valenzband.

\paragraph*{direkt oder indirekt} In einem direkten Halbleiter befindet sich das Maximum des Valenzbandes an der gleichen Stelle im reziproken Raum wie das Minimum des Leitungsbandes. Dies ermöglicht optische  Übergänge ohne Änderung des Wellenvektors bei niedrigen Energien. Bei indirekten Halbleitern ist dies nicht der Fall. Optische Übergänge benötigen hier die Beteiligung eines Phonons.

\paragraph*{kristallin oder amorph} Eine perfekte periodische Kristallstruktur ist nicht unbedingt erforderlich ($\alpha$-Silizium ist für Solarzellen relevant), aber wir betrachten hier nur kristalline Materialien.

\paragraph*{organisch oder anorganisch} Wir betrachten hier anorganische Halbleiter, die aus einem oder zwei Elementen bestehen. Organische Halbleiter auf der Basis von Kohlenstoffverbindungen organischer Moleküle sind ein Thema für sich.


\section{Intrinsische kristalline Halbleiter}

Die Bandstruktur und die Lage der Fermi-Energie hängt sowohl in der 'empty lattice approximation' als auch im 'tight binding' Modell von der Elektronenkonfiguration der beteiligten Atome ab. Es ist daher nicht verwunderlich, dass Halbleiter aus im Periodensystem benachbarten Atomen gebildet werden. Dies sind zum einen die Elemente der Gruppe IV, also Kohlenstoff (\ch{C}), Silizium (\ch{Si}) und Germanium (\ch{Ge}). Die Bandlücken betragen 5.47~eV, 1.12~eV und 0.66~eV und sind jeweils indirekt. Diamant ist also ein Isolator.

Verbindungshalbleiter bestehen aus zwei Elementen, deren Elektronenzahl sich gerade auf die Gruppe IV mittelt. III-V-Halbleiter sind beispielsweise \ch{GaP} (2.26 eV, indirekt), \ch{GaAs} (1.43 eV, direkt), \ch{InSb} (0.18 eV, direkt). II-VI-Halbleiter sind beispielsweise \ch{ZnS} (sc: 3.54 eV) und \ch{CdSe} (1.74 eV).



\begin{questions}
    \item Warum ist die Bandlücke von Silizium größer als die von Germanium, obwohl beide Elemente in der gleichen Gruppe des Periodensystems stehen?
\end{questions}


\section{Optische Übergänge}

Als Beispiel für den Unterschied zwischen direkten und indirekten Halbleitern möchte ich hier dem späteren Kapitel über optische Eigenschaften vorgreifen und optische Übergänge diskutieren. Ein Photon wird absorbiert und transportiert ein Elektron vom Valenzband in das Leitungsband. Dies wird als Interbandübergang bezeichnet. Gleichzeitig kann ein Phonon absorbiert oder emittiert werden. Die Energie- und Impulserhaltung lautet dann
\begin{align}
    E_g = & \hbar \omega_\gamma(\mathbf{k}_\gamma ) \pm \hbar \Omega (\mathbf{q}) \\
    \hbar \Delta \mathbf{k} = &\hbar \mathbf{k}_\gamma \pm \hbar \mathbf{q} \quad .
\end{align}
Größen des Photons sind mit $\gamma$ gekennzeichnet. Das eventuell beteiligte Phonon hat die Frequenz $\Omega$ und den Impuls $\mathbf{q}$. Das Elektron ändert seine Energie um $E_g$ und seinen Impuls um $\Delta \mathbf{k}$.



Betrachten wir zunächst \emph{direkte Übergänge}, also \emph{Prozesse ohne Beteiligung eines Phonons}. Der Impuls eines Photons ist viel kleiner als relevante Impulse von Elektronen in der Brillouinzone:
\begin{equation}
    k_\gamma = \frac{2 \pi }{\lambda} \approx \frac{2 \pi }{500\text{ nm}}
    \ll k_{BZ} = \frac{\pi}{a} \approx \frac{\pi}{0.5 \text{ nm}} \quad .
\end{equation}
Photonen ermöglichen also nur senkrechte Übergänge in der Elektronen-Dispersionsrelation bei quasi konstantem Elektronen-Impuls $\mathbf{k}$ ($\hbar \Delta \mathbf{k} \approx 0$). Diese direkten Prozesse dominieren in direkten Halbleitern. Bei gegebenem $\mathbf{k}$ ist die Energiedifferenz zwischen Valenz- und Leitungsband in der Nähe der Extrema
\begin{equation}
    \Delta E = \hbar \omega_\gamma = E_c(\mathbf{k})-  E_v(\mathbf{k}) = 
    \left( \frac{1}{m_e^\star} + \frac{1}{m_h^\star} \right) \frac{\hbar^2 k^2}{2} =  \frac{\hbar^2 k^2}{2 m^\star_\text{komb}}
\end{equation}
mit der kombinierten effektiven Masse $m^\star_\text{komb}$ einer kombinierten Bandstruktur $ \Delta E(\mathbf{k})$. Für diese können wir auch eine kombinierte Zustandsdichte angeben
\begin{equation}
    D_\text{komb}( \Delta E) = \frac{V}{2 \pi^2} \left( \frac{2 m^\star_\text{komb}}{\hbar^2} \right)^{3/2} \sqrt{\Delta E - E_g} \quad .
\end{equation}
Diese Zustandsdichte bestimmt über  Fermis  Goldener Regel
\begin{equation}
            \Gamma_{if} = \frac{2 \pi}{\hbar} \, \braket{f | \hat{H}' | i} \, D( \Delta E)
\end{equation}
 die Übergangsrate $\Gamma_{if}$ und damit den Absorptionskoeffizienten $\alpha \propto \Gamma_{if}$. Dieser beschreibt die verbleibende Intensität $I$, nachdem ein Lichtstrahl gegebener Frequenz eine Probe der Dicke $L$ durchquert hat
\begin{equation}
    I(\omega_\gamma) = I_0 e^{- \alpha(\omega_\gamma) L} \quad .
\end{equation}
Insgesamt bekommen wir also mit
\begin{equation}
    \alpha(\omega_\gamma) \propto ( m^\star_\text{komb})^{3/2}  \sqrt{\hbar \omega_\gamma - E_g} \label{eq:5_absorption_direct_HL}
\end{equation}
einen wurzelförmigen Verlauf des Absorptionsspektrums, der bei der Energie der Bandlücke einsetzt. Dabei haben wir die Annahme gemacht, dass  das Übergangs-Dipolmoment $\braket{f | \hat{H}' | i} $  im relevanten Frequenzbereich konstant ist.

\begin{marginfigure}[-170mm]
    \inputtikz{\currfiledir InSb_absorption}
    \caption{Absorption von Indiumantimonid (\ch{InSb}), einem direkten Halbleiter, in der Nähe der Bandkante (\cite{Johnson1967}).}
\end{marginfigure}

\begin{marginfigure}[-50mm]
    \inputtikz{\currfiledir Si_absorption}
    \caption{Absorption von Silizium (\ch{Si}), einem indirekten Halbleiter, in der Nähe der Bandkante (\cite{Macfarlane1955}).}
\end{marginfigure}


Bei \emph{indirekten Übergängen} unter Beteiligung eines Phonons wird die Rechnung aufwändiger. Man muss über alle Ausgangs- und Endzustände der Elektronen integrieren, die über irgendein Phonon verbunden sind.  Außerdem kann die Reihenfolge der Wechselwirkung mit Photon und Phonon getauscht werden und beide Möglichkeiten interferieren in der Quantenmechanik.\sidenote{siehe zB \cite{yu_cardona}} Das Ergebnis der Rechnung ist 
\begin{equation}
    \alpha(\omega_\gamma) \propto \left( \hbar \omega_\gamma -( E_g  \pm \hbar \Omega) \right)^2 \quad \text{für} \quad 
    \hbar \omega_\gamma >  E_g  \pm \hbar \Omega \quad ,
\end{equation}
je nachdem, ob ein Phonon absorbiert oder emittiert wird.
Die Absorption setzt also mit einem quadratischen Verlauf bei der indirekten Bandlücke ein. Der Anstieg ist mit steigender Frequenz also viel flacher als bei einer direkten Bandlücke. Auch bei indirekten Halbleitern gibt es ab einer gewissen Energie direkte Übergänge. Diese benötigen keine Phononen und sind somit wahrscheinlicher. Ab der direkten Bandlücke dominiert dann also auch der wurzelförmige Verlauf.



\begin{questions}
    \item Warum ist die Absorption bei direkten Halbleitern bei der Bandkante so viel stärker als bei indirekten Halbleitern?
\end{questions}


\section{Thermische Besetzung der Bänder}

Am absoluten Nullpunkt der Temperatur ist das Valenzband vollständig gefüllt, das Leitungsband vollständig leer. Durch thermische Anregung gelangen aber Elektronen von Valenz- ins Leitungsband. Wir berechnen nun, wie viele das sind. Dabei behandeln wir Elektronen und Löcher immer parallel zu einander.

Die Dichte $n$ der Elektronen im Leitungsband bzw. die Dichte $p$ der Löcher im Valenzband können wir einfach durch Integration über die Zustandsdichte $D$ (im jeweiligen Band) und die Fermi-Dirac-Funktion $f_{FD}$ bestimmen
\begin{eqnarray}
    n &= & \int_{E_c}^\infty f_{FD}(E,T) \, D_c(E) \, dE \\
    p &= & \int_{-\infty}^{E_v} (1 - f_{FD}(E,T)) \, D_v(E) \, dE  \quad .
\end{eqnarray}
Die Energien $E_{v,c}$ bezeichnen die jeweiligen Bandkanten ($E_g = E_c - E_v$). Bei den Löchern integrieren wir über $1-f_{FD}$, weil diese ja gerade unbesetzte Elektronenzustände sind. Die Zustandsdichte ist jeweils
\begin{eqnarray}
    D_c(E) &= & \frac{ (2 m^\star_n)^{3/2} } {2 \pi^2 \hbar^3} \, \sqrt{E - E_c}   \quad \text{für} \quad E > E_c \\
    D_v(E) &= & \frac{ (2 m^\star_p)^{3/2} } {2 \pi^2 \hbar^3} \, \sqrt{E_v - E} \quad \text{für} \quad E < E_v \quad ,
\end{eqnarray}
wobei die Massen hier die jeweiligen effektiven Massen an der Bandkante sind.

Nun müssen wir noch die Fermi-Dirac-Funktion $f_{FD}$ geeignet nähern und dazu zunächst die Unterscheidung zwischen Fermi-Energie und Fermi-Niveau einführen. In der Fermi-Dirac-Funktion steht eigentlich das chemische Potential $\mu$. Die Fermi-Energie hatten wir als $E_F = \mu(T=0)$ definiert. Hier interessieren uns aber höhere Temperaturen. \cite{Gross_FK} verwendet direkt das chemische Potential $\mu(T)$, \cite{Hunklinger2014} führt den Begriff des Ferminiveaus als Äquivalent zum chemischen Potential ein. Um Verwechslungen mit der Beweglichkeit zu vermeiden, wird dort $E_F(T)$ statt $\mu$ verwendet. Ich folge hier Hunklinger.

Wir machen die Annahme, dass das Fermi-Niveau $E_F(T)$ weit\sidenote{auf einer $k_b T$-Skala} von den Bandkanten entfernt ist. Dies ist die \emph{Näherung der Nichtentartung} und gilt bei intrinsischen oder nur schwach dotierten Halbleitern. Sie gilt bei stark dotierten (entarteten)  Halbleitern nicht. Dazu mehr unten. Wenn diese Näherung also gemacht werden kann, also $|E - E_F| \gg k_b T$ für die relevanten Werte von $E$ ist, dann können wir die Fermi-Dirac-Verteilungen als Boltzmann-Verteilungen nähern:
\begin{eqnarray}
    f_{FD} = \frac{1}{e^{ (E- E_F) / k_B T} +1 } \approx e^{ -(E- E_F) / k_B T}  \quad \text{für} \quad E > E_F \\
  1-  f_{FD} = \frac{1}{e^{ (E_F - E) / k_B T} +1 } \approx e^{ -(E_F - E) / k_B T}  \quad \text{für} \quad E < E_F  \quad .
\end{eqnarray}

Damit können wir nun alles einsetzen und die Integrale berechnen:\sidenote{Man integriert über $x = (E-E_F)/k_bT$ und benutzt $\int \sqrt{y} e^{-y} = \sqrt{\pi}/2$.}
\begin{eqnarray}
    n &=& 2 \left( \frac{ k_b T \,  m^\star_n } {2 \pi \hbar^2} \right)^{3/2} \, e^{- (E_c - E_F) / k_b T} = \mathcal{N} \, e^{- (E_c - E_F) / k_b T} \\*  
    p &=& 2 \left( \frac{ k_b T \,  m^\star_p } {2 \pi \hbar^2} \right)^{3/2} \, e^{+ (E_v - E_F) / k_b T} = \mathcal{P} \, e^{+ (E_v - E_F) / k_b T}  \quad . \label{eq:5_konz_zugaenglich}
\end{eqnarray}
Wir haben die effektiven ('zugänglichen') Zustandsdichten $\mathcal{N}$ und $\mathcal{P}$ eingeführt, die nur schwach von der Temperatur abhängen. Wenn wir sie als konstant annehmen, haben wir die beiden Bänder zu zwei diskreten Zuständen vereinfacht, die bei den Energien $E_c$ bzw. $E_v$ liegen. Dies wird im Folgenden vieles vereinfachen.

\begin{figure}
    \inputtikz{\currfiledir thermal_occupation}
    \caption{Thermische Besetzung der Zustände eines intrinsischen Halbleiters mit 
    $m^\star_p = 1.5 m^\star_n$ und $E_g = k_b T$. 
    Die besetzten Zustände sind grau unterlegt. Man erkennt $n=p$. }
\end{figure}

\begin{questions}
    \item Wie beeinflusst das Verhältnis der effektiven Massen  $m^\star_p / m^\star_n$ die Lage des Fermi-Niveaus $E_F(T)$ im undotierten Halbleiter?
\end{questions}



\section{Massenwirkungsgesetz}

Interessant ist, dass sowohl $n$ als auch $p$ von der Lage des Fermi-Niveaus $E_F$ abhängt, deren Produkt aber nicht
\begin{equation}
    n \, p = \mathcal{N} \mathcal{P} \, e^{- E_g / k_b T} =   \mathcal{W} \, T^3 \, e^{- E_g / k_b T}  \quad . \label{eq_5_mwg}
\end{equation}
Das Produkt $n p$ hängt nur von Materialparametern und der Temperatur ab. Die Beziehung  $n p = $~const. wird in Analogie zu einer chemischen Reaktion \emph{Massenwirkungsgesetz} genannt. Bei einer  Reaktion der Form
\begin{equation}
    \ch{A} +  \ch{B} \rightleftharpoons  \ch{C} +  \ch{D} 
\end{equation}
ergibt sich in stark verdünnter Lösung eine Gleichgewichtskonstante $K$ mit
\begin{equation}
    K = \frac{ [\ch{C}] \, [\ch{D}] }{ [\ch{A}] \, [\ch{B}]} \quad ,
\end{equation}
wobei eckige Klammern hier die Konzentration der jeweiligen Stoffe bezeichnen. Man kann die Existenz von Elektronen im Leitungsband und Löchern im Valenzband als chemische Reaktion der Art
\begin{equation}
    \ch{<nichts>} \rightleftharpoons  \ch{Elektronen} +  \ch{Löcher} 
\end{equation}
sehen. Durch thermische Anregung entstehen Elektronen (im Leitungsband) und Löcher (im Valenzband) aus dem Nichts. Die rechte Seite von Gl. \ref{eq_5_mwg} entspricht also der Gleichgewichtskonstante $K$.

Man beachte, dass zwar  $n p = $~const., aber bislang noch nicht $n=p$ gefordert wurde. Das Massenwirkungsgesetz  $n p = $~const gilt also auch für dotierte Halbleiter.



\section{Intrinsische Ladungsträgerdichte}

Bei dotierten Halbleitern (s.u.) kann ein Elektron nicht nur aus dem Valenzband, sondern auch von einem Fremdatom stammen. 
Für intrinsische, d.h. undotierte Halbleiter gilt jedoch, dass für jedes Elektron im Leitungsband ein Loch im Valenzband entstehen muss, d.h. $n=p$. Die Dichte der so paarweise erzeugten Ladungsträger wird als intrinsische Ladungsträgerdichte $n_i = p_i$ bezeichnet. Sie ergibt sich aus dem Massenwirkungsgesetz
\begin{equation}
    n_i = p_i = \sqrt{n p} = \sqrt{\mathcal{N}\mathcal{P}} \, e^{- E_g / 2 k_b T} \quad .
\end{equation}
Diese Gleichung hatten wir in der Einführung schon verwendet, um Ladungsträgerdichten abzuschätzen. Man beachte hier die Zwei im Exponenten. Die Bandlücke erscheint nur halb so groß, weil wir gleichzeitig ein Elektron und ein Loch aus dem Nichts erzeugen.



\section{Temperaturabhängigkeit des Fermi-Niveaus}

Wir können uns einen Zusammenhang für die Temperaturabhängigkeit des Fermi-Niveaus   undotierter Halbleiter beschaffen, in dem wir $n_i = p_i$ umformen zu
\begin{equation}
    \frac{\mathcal{P}}{\mathcal{N}} = \left( \frac{m^\star_n}{m^\star_p} \right)^{3/2} = e^{(2 E_F - E_c - E_v)/(k_b T)}
\end{equation}
und dann nach $E_F$ auflösen
\begin{equation}
    E_F(T) = \frac{E_c + E_v}{2} + \frac{3}{4} k_B T \ln \left( \frac{m^\star_p}{m^\star_n} \right) \quad . \label{eq:5_Efermi_intr}
\end{equation}
Bei $T=0$ liegt das Fermi-Niveau also in der Mitte der Bandlücke. Falls Leitungs- und Valenzband gleich stark gekrümmt sind, also Elektronen- und Loch-Masse identisch, dann bleibt es dort auch bei steigender Temperatur. Unterscheiden sich die Massen, dann bewegt sich das Fermi-Niveau leicht in Richtung der leichteren Ladungsträger. Da $k_B T \ll E_g$ ist dies aber ein kleiner Effekt.


\section{Dotierung}

Kein Halbleiterkristall ist wirklich rein, sondern enthält immer Fremdatome. Diese Fremdatome verursachen zusätzliche Ladungsträger, Elektronen oder Löcher und beeinflussen damit die Leitfähigkeit. In einem idealen \ch{GaAs} beträgt die intrinsische Ladungsträgerdichte bei Raumtemperatur beispielsweise $n_i \approx 10^7$~cm$^{-3}$. Aber selbst in den besten \ch{GaAs}-Kristallen findet man $n \approx 10^{16}$~cm$^{-3}$. Daher kann man einen Kristall gezielt dotieren (eigentlich 'verunreinigen'), um die Ladungsträgerdichte einzustellen. Man unterscheidet zwischen n- und p-Dotierung, da man ein Atom der Valenz $V$ im Ausgangsmaterial durch ein Fremdatom der Valenz $V+1$ oder $V-1$ ersetzen kann. Im Fall $V+1$ spricht man von Donatoren, also Atomen, die Elektronen abgeben, oder von n-Dotierung. Im Fall von $V-1$ spricht man von Akzeptoren (Atome, die Elektronen aufnehmen) oder p-Dotierung.

Ein Donator-Atom ist für 'sein' Elektron ähnlich einem Proton für das Elektron im Wasserstoff-Atom, und analog ein Akzeptor für das Loch. Wir können also die Beschreibung des Wasserstoff-Atoms benutzen und müssen nur die effektive Masse des Ladungsträgers und die relative Permittivität $\epsilon_r$ des Materials anpassen. Für einen Donator sind die Energie-Eigenwerte also 
\begin{equation}
    E_n = - \frac{1}{2} \, \frac{m_e^\star e^4}{(4 \pi \epsilon_r \epsilon_0)^2 \hbar^2}
\frac{1}{n^2} =     
   - \frac{m_e^\star}{m_\text{frei}} \, \frac{1}{\epsilon_r^2} \, \frac{13.6 \text{ eV}}{n^2}
\end{equation}
mit $n$ hier der Quantenzahl und nicht der Ladungsträgerdichte. Der Bohr-Radius ist
\begin{equation}
    r_\text{Bohr} = \frac{\epsilon_r}{m_e^\star / m_\text{frei}} \, 0.53 \text{ \AA} \quad .
\end{equation}
Für Silizium findet man mit $m_e^\star = 0.1 m_\text{frei}$ und $\epsilon_r = 11.7$ die Werte $E_1 = -10$~meV und $ r_\text{Bohr} = 62$~\AA. Der Betrag der Bindungsenergie ist also viel kleiner als die Bandlücke (1.14~eV) und kleiner als $k_b T$ (25~meV) bei Raumtemperatur. Der Bohr-Radius ist viel größer als die Gitterkonstante (5.4~\AA). Es liegen also sehr viele Silizium-Atome innerhalb der Bahn des Elektrons um das eine Donator-Fremdatom.

Welche Bedeutung hat die Bindungsenergie des Donator-Elektrons für die Bandstruktur? Die Ionisation des Wasserstoffatoms entspricht dem Übergang des gebundenen Donator-Elektrons in das Leitungsband. Dort kann es sich frei bewegen. Die Donatorzustände liegen daher im Abstand $E_n$ unterhalb der Unterkante des Leitungsbandes. Analog dazu liegen die Akzeptorzustände etwas oberhalb der Oberkante des Valenzbandes. In unserem Wasserstoffmodell sind die Bindungsenergien (Abstände der Zustände von der Bandkante) unabhängig vom Fremdatom, es gehen nur die Eigenschaften des Ausgangsmaterials ein. Dies ist in der Realität auch fast so, da der Bohrradius so groß ist, dass Details der Elektronenkonfiguration des Fremdatoms nur sehr schwach eingehen.

Bei Raumtemperatur sind die Störstellen mit großer Wahrscheinlichkeit ionisiert. Bei tiefen Temperaturen kann man durch Infrarotabsorption die Lage der Donator- und Akzeptor-Niveaus bestimmen.

%XXXX Ende Revision March 2025
\section{Temperaturabhängigkeit der Ladungsträgerdichte}

Durch die Dotierung lässt sich die Ladungsträgerdichte einstellen, die dann aber natürlich stark temperaturabhängig ist, weil die Bindungsenergie der Störstellen relativ klein ist. Dies werden wir nun etwas genauer betrachten. Wir machen weiterhin die Näherung der Nichtentartung, also dass das Fermi-Niveau\sidenote{=chemisches Potential $\mu$} weit genug von der Bandkante entfernt ist, so dass wir die Fermi-Dirac-Verteilung mit einer Boltzmann-Verteilung nähern können. Die Ladungsträgerdichten im Leitungsband ($n$) und Valenzband ($p$) sind also wie oben
\begin{equation}
    n  = \mathcal{N} \, e^{- (E_c - E_F) / k_b T} \quad \text{und} \quad   
    p =  \mathcal{P} \, e^{+ (E_v - E_F) / k_b T} \quad , \label{eq:5_np_eff}
\end{equation}
unabhängig davon, ob die LLadungsträger aus dem gegenüberliegen Band oder aus Störstellen stammen. Die intrinsischen Dichten hatten wir erst später eingeführt.

Jedes Donator- und Akzeptor-Niveau kann nur mit einem Ladungsträger besetzt werden. Ein unterschiedlicher Spin reicht nicht aus, weil die Coulomb-Abstoßung der Ladungsträger größer ist also die Bindungsenergie.\sidenote{Man darf nicht die Lage der Zustände im \ch{H}-Atom vergleichen, sondern die zwischen \ch{H} und \ch{H-} oder \ch{He} und \ch{He+}.} Sei $n_D$ also die Dichte der Donatoren, die entweder besetzt (neutral) sein können ($n_D^0$) oder ionisiert  ($n_D^+$) und damit natürlich
\begin{equation}
    n_D = n_D^0 + n_D^+ \quad \text{und} \quad   n_A = n_A^0 + n_A^-   \quad .
\end{equation}
Für deren Temperaturabhängigkeit kann man die Fermi-Dirac-Verteilung nicht mehr durch eine Boltzmann-Verteilung nähern, sondern muss schreiben
\begin{equation}
    \frac{n_D^0}{n_D} = \frac{1}{e^{(E_D - E_F)/ k_b T } +1}
    \quad \text{und} \quad 
    \frac{n_A^0}{n_A} = \frac{1}{e^{(E_F - E_A)/ k_b T } +1} \quad .
\end{equation}

Alle Ladungen im Halbleiter müssen sich jederzeit neutralisieren, also 
\begin{equation}
    n + n_A^- = p + n_D^+ \quad .
\end{equation}

Diese vier Gleichungen beschrieben zusammen die Temperaturabhängigkeit. Wir nehmen nun  einen n-Halbleiter an, also dass viel mehr Donatoren als Akzeptoren vorhanden sind, also $n_D \gg n_A$. In diesem Fall findet sich sicherlich für jeden Akzeptor ein Elektron aus einem Donator, so dass alle Akzeptoren negativ geladen sind ($n_A \approx n_A^-$).
Und wir nehmen an, dass die Dotierung so stark ist, dass sie die Leitfähigkeit dominiert (\emph{Störstellenleitung}). Dies ist dann der Fall, wenn $n_D^+ \gg p$, also viel mehr Donatoren ionisiert sind als Löcher vorhanden, oder die Mehrzahl der Elektronen im Leitungsband von Donatoren stammt, also die intrinsische Dichte vernachlässigt werden kann ($n_i \ll n_D^+$). Damit wird die Dichte $n$ der Elektronen im Leitungsband
\begin{eqnarray}
    n  &= & p + n_D^+ -  n_A^- \approx n_D^+ -  n_A = (n_D - n_D^0) - n_A \\
    & =& n_D \left( 1- \frac{1}{e^{(E_D - E_F)/ k_b T } +1} \right) - n_A \quad .
\end{eqnarray}
Das Fermi-Niveau $E_F$ können wir durch Umformen mit Gl.~\ref{eq:5_np_eff} entfernen
\begin{equation}
    \frac{n (n_A + n)}{n_D - n_A - n} = \mathcal{N} \, e^{- E_d / k_b T} \quad , \label{eq:5_n_of_T}
\end{equation}
wobei $E_d = E_c - E_D$ der energetische Abstand der Donator-Niveaus von der Unterkante der Leitungsbandes ist. Aus dieser Gleichung kann man bei gegebener Fremdatom-Konzentrationen  $n_A$, $n_D$ und Energien die Ladungsträgerdichte $n$ berechnen. Zusammen mit  
Gl.~\ref{eq:5_np_eff} erhält man dann auch die Lage der Fermi-Niveaus $E_F$. Dies ist in Abb.~\ref{fig:5_dpoing_temp} als Funktion der reziproken Temperatur $T$ gezeigt und in der Abb.~\ref{fig:5_Ge_n_density} am Anfang des Kapitels linear. Man findet vier Temperaturbereiche, die im Folgenden mit steigender Temperatur diskutiert werden.

\begin{figure} 
    \inputtikz{\currfiledir doping_temp}
    \caption{Temperaturabhängigkeit der Ladungsträgerdichte $n$. In dieser Darstellung ist $E_d = E_g / 10$ gewählt. \label{fig:5_dpoing_temp}}
\end{figure}

\paragraph*{Kompensationsbereich} Bei sehr tiefen Temperaturen ist $k_b T \lll E_d$ und nur sehr wenige Ladungsträger sind im Leitungsband, also $n \ll n_A \ll n_D$ und Gl. \ref{eq:5_n_of_T} wird
\begin{equation}
    n  \approx \frac{n_D \, \mathcal{N}}{n_A}  \, e^{- E_d / k_b T}  \quad .
\end{equation}
Das Fermi-Niveau wird mit Gl.~\ref{eq:5_np_eff}
\begin{equation}
    E_F \approx E_c - E_d + k_b T \ln \left( \frac{n_D}{n_A} \right) \quad .
\end{equation}
Wenn es nur kalt genug ist, liegt das Fermi-Niveau auf den Donator-Zuständen. Die Donatoren sind teilweise geladen, aber nicht so sehr, weil deren Ladungsträger im Leitungsband sind, sondern weil sie die Akzeptoren besetzt haben und diese keine Löcher mehr liefern können. Die Akzeptoren werden also kompensiert. Wenn $T$ größer wird, dann verschiebt sich $E_F$ in Richtung Bandkante und mehr Ladungsträger gelangen ins Leitungsband.

\paragraph*{Störstellenreserve} Mit steigender Temperatur ist dann der Punkt erreicht, dass $n_A \ll n \ll n_D$. Damit wird   Gl. \ref{eq:5_n_of_T} 
\begin{equation}
    n \approx \sqrt{n_D \, \mathcal{N}} \,  \, e^{- E_d / 2 k_b T} 
\end{equation}
(man beachte die 2 im Exponenten) und das Fermi-Niveau wird 
\begin{equation}
    E_F \approx E_c - \frac{E_d}{2} - \frac{k_b T}{2} \ln \left( \frac{\mathcal{N}}{n_D} \right)  \quad .
\end{equation}
Das Fermi-Niveau ist als in etwa in der Mitte zwischen Leitungsband-Unterkante und Donator-Niveau. Es sind noch nicht alle Donatoren ionisiert. Die Donatoren übernehmen die Rolle des Valenzbandes bei intrinsischen Halbelitern und $E_d$ die Rolle der Bandlücke.

\paragraph*{Störstellenerschöpfung} Jetzt erreichen wir Raumtemperatur und $k_b T \approx E_d$. Damit wird Gl. \ref{eq:5_n_of_T} 
\begin{equation}
    \frac{n^2}{n_D - n} \approx \mathcal{N}  \quad \text{bzw.} \quad n \approx n_D = \text{const.}
\end{equation}
weil $n \ll  \mathcal{N}$ und 
\begin{equation}
    E_F \approx E_c - k_b T \ln \left( \frac{\mathcal{N}}{n_D} \right) \quad .  \label{eq:5_ef_stoerstellenerschoepfung}
\end{equation}
Alle Störstellen sind ionisiert, aber die direkte Anregung von Elektronen aus dem Valenz- ins Leitungsband spielt noch keine Rolle.

\paragraph*{Eigenleitung} Die Temperatur ist schließlich so hoch, dass $k_b T \gg E_d$. Ladungsträger werden vom Valenz- ins Leitungsband angeregt und die Annahme $p \ll n_D$ gilt nicht mehr. Die Dotierung spielt keine Rolle mehr und der Halbleiter benimmt sich wie ein intrinsischer Halbleiter mit den Gleichungen  \ref{eq:5_konz_zugaenglich} und  \ref{eq:5_Efermi_intr} für $n$ und $E_F$.


\section{p-n-Übergang}

Die Dotierung eines Halbleiter-Kristalls kann räumlich variieren. Dazu kann man beispielsweise durch UV-Lithographie eine Maske erzeugen, die das Eindringen von Fremdatomen an manchen Stellen verhindert. Wir betrachten hier einen Kristall, bei dem sich auf einer Längenskala von wenigen Nanometern die Dotierung von p nach n ändert.  Seien die Konzentration der Störstellen beispielsweise
\begin{equation}
    n_A(x) = n_A \Theta(x) \quad \text{und}  \quad  n_D(x) = n_D \Theta(-x) 
\end{equation}
mit der Stufenfunktion $\Theta$. Der dabei entstehende p-n-Übergang besitzt einen charakteristischen Verlauf der Bandkanten und dient technologisch beispielsweise als Diode.


\begin{marginfigure}[-110mm]
  %  \includegraphics*[width=49mm]{\currfiledir sketches/pn-sketch.png}
    \inputtikz{\currfiledir pn_sketch}
    \caption{Räumliche Verteilung der festen und beweglichen Ladungen an einem p-n-Übergang und das sich daraus ergebende Potential.}
\end{marginfigure}

\begin{marginfigure}[-10mm]
    %\includegraphics*[width=49mm]{\currfiledir sketches/pn-level.png}
  \inputtikz{\currfiledir pn_level}
    \caption{Bandstruktur und Lage des Fermi-Niveaus vor und nach dem Verbinden  von zwei unterschiedlich dotierten Halbleitern.}
\end{marginfigure}

Stellen wir uns zunächst vor, dass die beiden Bereiche nicht miteinander in Verbindung stünden. Im p-dotierten Bereich gibt es bei Raumtemperatur viel mehr  Löcher im Valenzband  als Elektronen im Leitungsband, im n-dotierten Bereich ist es gerade anders herum. Wenn man die beiden Bereiche miteinander verbindet, dann diffundieren die Ladungsträger jeweils in den anderen Bereich, um die Konzentration auszugleichen. Dort rekombinieren aber Elektronen und Löcher und vernichten sich so gegenseitig. In den Ausgangsbereichen bleiben die gegenteilig geladenen immobilen Störstellen zurück. Dadurch bildet sich im p-dotierten Bereich eine negative Raumladung, im n-dotierten eine positive. Diese wirkt der Diffusion entgegen und ein Gleichgewicht stellt sich ein. Die dabei entstehende Potentialdifferenz nennt man \emph{Diffusionsspannung} $V_D$. Diese verschiebt die Gesamtenergie der Elektronen um den Beitrag  $-e \, V_D$ und verbiegt so die Bänder, im n-dotierten Bereich hin zu niedrigeren Energien. Die Raumladungszone nennt man auch Verarmungszone, weil dort fast keine mobilen Ladungsträger mehr vorhanden sind.\sidenote{Das Massenwirkungsgesetz fordert $n \cdot p =$const, aber $n+p$ kann stark variieren. } 

Man kommt zum gleichen Ergebnis auch mit einer anderen Argumentation. Vor dem Zusammenführen hat man im p-dotierten Bereich bei Raumtemperatur ein Fermi-Niveau etwas oberhalb des Akzeptor-Niveaus, weil wir im Bereich der Störstellenerschöpfung sind. Im n-dotierten Teil ist die Bandlücke identisch, weil es dasselbe Halbleiter-Material ist. Nur liegt dort das Fermi-Niveau  etwas unterhalb der Donator-Niveaus. Im thermischen Gleichgewicht kann es aber nur ein Fermi-Niveau (= chemisches Potential) geben.\sidenote{In der Chemie pro Stoff, hier aber alles Elektronen.} Es bildet sich also ein Makropotential $\phi(x)$, das zur potentiellen Energie $q \phi(x)$ der Ladungsträger ($q = \pm e$) beiträgt, so dass das Fermi-Niveau wieder räumlich konstant ist.\sidenote{Wenn man eine externe Spannung anlegt, so wird dieses externe Potential typischerweise als separat vom Fermi-Niveau angesehen. Dann ist das Fermi-Niveau nicht mehr räumlich konsant. Die Trennung zwischen Makropotential und externem Potential ist aber nur Konvention.} 


Die Diffusionsspannung $V_D$ ergibt sich also aus der Differenz der Fermi-Niveaus im unverbundenen Zustand. In erster Näherung ist also $e V_D \approx E_g$. Im Bereich der Störstellenerschöpfung ist sie durch Gl.~\ref{eq:5_ef_stoerstellenerschoepfung} gegeben
\begin{eqnarray}
    e \, V_D &= & E_F^n - E_F^p \\
    & =&   E_c - k_b T \ln \left( \frac{\mathcal{N}}{n_D} \right) - 
    E_v - k_b T \ln \left( \frac{\mathcal{P}}{n_A} \right) \\
   & =& E_g -  k_b T \ln \left( \frac{\mathcal{N \, P}}{n_D \, n_A} \right) \\
   & = & k_b T \ln \left( \frac{n_D \, n_A}{n_i^2} \right)  \quad .
\end{eqnarray}

Die Ladungsträgerdichten sind weiterhin durch die Lage des Fermi-Niveaus (Gl.~\ref{eq:5_konz_zugaenglich}) gegeben, nur dass sich dieses nun aus einem konstanten Anteil $E_F$ und einem räumlich variablen Anteil $e \phi(x)$ zusammensetzt. Wir erhalten also 
\begin{equation}
    n(x)  = \mathcal{N} \, e^{- (E_c - E_F - e\phi(x)) / k_b T} \quad \text{und} \quad   
    p(x) =  \mathcal{P} \, e^{+ (E_v - E_F  - e\phi(x)) / k_b T} \quad . \label{eq:5_n_of_x}
\end{equation}
Die  Raumladungsdichte $\rho(x)$ ist damit
\begin{equation}
    \rho(x) = e \left( n_D(x) - n_A(x) - n(x) + p(x) \right) \quad ,
\end{equation}
wenn man annimmt, dass alle Störstellen ionisiert sind. Zusammen mit der Poisson-Gleichung
\begin{equation}
    - \nabla^2 \phi(x) = \frac{1}{\epsilon_r \epsilon_0} \, \rho (x)
\end{equation}
haben wir ein nichtlineares gekoppeltes  System von Differentialgleichungen für $\rho(x)$ und $\phi(x)$, das sich nur numerisch lösen lässt.


\section*{Schottky-Modell der Raumladungszone}


Im Schottky-Modell nähert man den graduellen Verlauf der Ladungsträgerdichten durch Rechteck-Funktionen.\sidenote{Das geht, weil die beweglichen Ladungsträger in ihrer Dichte exponentiell variieren, also sehr schnell irrelevant werden, siehe z.B. \cite{AshcroftMermin2013}.} In einem Intervall der Breite $d_p$ befinden sich nicht durch freie Ladungsträger kompensierte Akzeptoren der Konzentration $n_A$ und andersherum für den n-dotierten Bereich. Wenn das Koordinatensystem so ist, dass $x=0$ an der p-n-Grenzfläche, dann ist beispielsweise $\rho(x) = - e n_A$ im Bereich $- d_p < x < 0$ und  $\rho(x) =0$ für $x < -d_p$. Durch Integration der Poisson-Gleichung bekommt man dann einen parabelförmigen Verlauf 
\begin{equation}
    \phi(x) = \phi_{-\infty} + \frac{e n_A}{\epsilon_r \epsilon_0} \left( d_p + x \right)^2  \quad \text{für} \quad - d_p < x < 0 \quad ,
\end{equation}
wobei $\phi_{-\infty}$ das konstante Makropotential tief im p-dotierten Bereich ist und $V_D =  \phi_{+\infty} - \phi_{-\infty}$. Das Potential erfüllt so schon die Stetigkeitsbedingungen im Übergang zu $\phi_{\pm\infty}$. Bei $x=0$ muss ebenfalls die erste Ableitung  stetig sein, also
\begin{equation}
    n_D \, d_n = n_A \, d_p
\end{equation}
was der Forderung nach Gesamt-Neutralität entspricht. Aus der Stetigkeit von $\phi$ selbst ergibt sich eine weitere Bedingung
\begin{equation}
    \frac{e}{2 \epsilon_r \epsilon_0} \left(  n_D \, d_n^2  + n_A \,d_p^2 \right) =\phi_{+\infty} - \phi_{-\infty} = V_D \quad .
\end{equation}
Damit erhalten wir
\begin{equation}
    d_n = \sqrt{ \frac{{2 \epsilon_r \epsilon_0 \, V_D}}{e}   \, \, \frac{n_A / n_D}{n_A + n_D}}
\end{equation}
und $d_p$ analog. typischerweise ist $e V_D \approx E_g \approx 1 $~eV, und $n_{A,D} \approx 10^{14} \dots 10^{18}$~cm$^{-3}$. Also liegen $d_{p,n}$ bei 10 bis 1000~nm.


\begin{figure}
    \inputtikz{\currfiledir Schottky_modell}
    \caption{Schottky-Modell der Raumladungszone}
\end{figure}

\section*{Externe Spannung und Strom-Spannungs-Kennlinie}

Nun legen wir eine externe Spannung an den p-n-Übergang an. Wir nennen die Spannung $U$ positiv, wenn sie das Potential der p-dotierten Seite anhebt. Nur innerhalb der  Verarmungszone ist die Leitfähigkeit relativ niedrig, so dass die Spannung im Wesentlichen hier abfällt. Die Bänder ändern sich also nur im Bereich der Verarmungszone, außerhalb bleibt alles unverändert. Damit bleiben auch die Gleichungen aus dem letzten Abschnitt gültig, wenn wir jeweils $V_D$ durch $V_D - U$ ersetzen.

Durch Anlegen der externen Spannung ändert sich die Breite der Raumladungszone
\begin{equation}
    d = d_p + d_n = d(U=0) \, \sqrt{1 - \frac{U}{V_D} }\quad .
\end{equation} 
Positive Spannungen (in Durchlassrichtung) reduzieren die Breite der Raumladungszone (=Verarmungszone), negative Spannung (Sperrrichtung) vergrößern sie. 

Um die Wirkungsweise eines p-n-Übergangs als Diode zu verstehen sind die beteiligten Ströme\sidenote{ 
    \cite{Gross_FK} unterscheidet zwei Paare von Strömen: diff \& drift sowie gen \& rec. Die Vorzechen sind aber anders definiert: diff + drift = 0 aber gen = rec !} relevant.

\paragraph*{Rekombinationsstrom  $j^\text{rec}$} Aufgrund des Konzentrationsunterschieds links und rechts der Grenzfläche diffundieren beispielsweise Elektronen aus dem n-dotierten Bereich  in den p-dotierten und rekombinieren mit den dort in großer Zahl vorhandenen Löchern. Dieser Strom wird auch \emph{Diffusionsstrom} genannt.

\paragraph*{Generationsstrom $j^\text{gen}$} Die Raumladungszone bildet einen Kondensator, in dem Ladungsträger beschleunigt werden. Thermisch erzeugte Elektronen im p-dotierten Bereich driften durch das Raumladungs-Feld in Richtung n-dotierten Bereich. Dieser Strom wird auch \emph{Driftstrom} oder \emph{Feldstrom} genannt.

Im thermischen Gleichgewicht sind beide Ströme gleich groß und kompensieren sich gerade $j_n^\text{rec} = j_n^\text{gen}$. Und natürlich kann man genau so mit Löchern argumentieren, so dass es beide Ströme auch mit dem Index $p$ gibt. Die Gesamt-Ströme sind die Summe der beiden Ladungsträger.

Der Diffusionsstrom fließt entgegen der Raumladungs-Potentialschwelle. Die Wahrscheinlichkeit, dass dies gelingt, enthält einen Boltzmann-Faktor
\begin{equation}
    j_n^\text{rec}(U) = a(T) \, e^{-e (V_D - U) / k_b T} =  j_n^\text{rec}(0) \, e^{e U / k_b T}
    = j_n^\text{gen} \, e^{e U / k_b T}
    \quad .
\end{equation}
Bei angelegter Spannung besteht kein thermisches Gleichgewicht mehr, so dass $j_n^\text{rec} = j_n^\text{gen}$ nicht mehr gilt, sondern
\begin{equation}
    j_n(U) = j_n^\text{rec}(U) - j_n^\text{gen} = j_n^\text{gen} \left(  e^{e U / k_b T} - 1 \right) \quad ,
\end{equation}
beziehungsweise für beide Ladungsträger zusammen 
\begin{equation}
    j(U) =  j_s \left(  e^{e U / k_b T} - 1 \right)
\end{equation}
mit dem Sättigungsstrom $j_s = j_n^\text{gen}  + j_p^\text{gen} $.

\newpage
\section{Zusammenfassung}

\textit{Schreiben Sie hier ihre persönliche Zusammenfassung des Kapitels auf. Konzentrieren Sie sich auf die wichtigsten Aspekte und die am Anfang genannten Ziele des Kapitels.}

\vspace*{10cm}

\printbibliography[segment=\therefsegment,heading=subbibliography]

%\renewcommand{\lastmod}{\today}
\renewcommand{\chapterauthors}{Markus Lippitz}
\renewcommand{\lastmod}{11. Juni 2025}

\chapter{Supraleiter}




\section{Ziele}
 


\begin{itemize}
\item Sie können das Konzept der Cooper-Paare im Rahmen der BCS-Theorie benutzen, um grundlegende Eigenschaften von Supraleitern zu erklären.
\item Sie können das Konzept der makroskopischen Wellenfunktion benutzen, um die Flussquantisierung und die Quanteninterferenz in Josephson-Kontakten wie unten dargestellt zu beschreiben.
\end{itemize}

\begin{figure}
    \inputtikz{\currfiledir squid}
    \caption{Quanteninterferenz des Stromes durch zwei parallel geschaltete Josephson-Kontakte (\ch{Sn}/\ch{SnO_x}/\ch{Sn}) als Funktion des Magnetfelds $\bm{B}$ im supraleitenden Zustand ($T=2$K). Daten aus \cite{Jaklevic1965}. \label{fig:6_squid_data}
}
\end{figure}




\section{Überblick}

In diesem Kapitel gehen wir einen Schritt zurück in unserer Beschreibung der Festkörper und vernachlässigen Details der Bandstruktur, indem wir wieder von einem quasi-freien Elektronengas ausgehen. Dafür gehen wir dann aber auch einen Schritt weiter, indem wir nun erstmals Korrelationen zwischen Elektronen berücksichtigen. Es wird nicht mehr ausreichen, ein einziges Elektron zu betrachten, sondern 'synchronisierte' Paare von Elektronen werden wichtig werden.

Wir beginnen mit einem Überblick über experimentelle Beobachtungen an Supraleitern, um dann zunächst ein phänomenologisches  Modell und schließlich ein mikroskopisches Modell zur Beschreibung einzuführen.

\section*{Idealer Leiter}

Ein Supraleiter ist zunächst einmal ein idealer Leiter, in dem Strom widerstandsfrei fließt. Dies wurde 1911 von Heike Kamerlingh Onnes entdeckt. Nachdem ihm 1908 die Verflüssigung von Helium gelungen war, wollte er eigentlich den Grenzwert der Leitfähigkeit bei tiefen Temperaturen untersuchen, analog zu unserem Kapitel~\ref{chap:fermi-gas}. Er verwendete sehr reines Quecksilber (\ch{Hg}) und fand bei 4.2~K einen sprunghaften Übergang zu einem dann von ihm supraleitend genannten Zustand.\sidenote{Zur Geschichte der Helium-Verflüssigung und der Supraleitung siehe \cite{Vandelft2008} und \cite{Vandelft2010}.}

Im supraleitenden Zustand ist der Widerstand nicht nur sehr klein, sondern tatsächlich null. Man kann einen Ringstrom in einer geschlossenen Leiterschleife induzieren. Dieser würde mehr als 100~000 Jahre anhalten. Der Widerstand fällt um mehr als 14 Größenordnungen.

\begin{marginfigure}
    \inputtikz{\currfiledir HKO_Hg}
    \caption{Sprung des Widerstands von Quecksilber (\ch{Hg}) beim Übergang in den supraleitenden Zustand. Daten aus \cite{Kamerlingh_Onnes_1911}.}
\end{marginfigure}

Sehr viele Materialien sind supraleitend. Reine Elemente zeigen eine Sprungtemperatur  von unter 10~K, Legierungen liegen etwas höher. Oxyde mit vier oder fünf verschiedenen Elementen bilden sogenannte Hochtemperatur-Supraleiter mit einer Sprungtemperatur von bis zu 135~K. Bei sehr hohen Drücken werden noch höhere Werte erreicht.

Auffällig ist, dass gerade 'gute' Metalle keine hohe Sprungtemperatur besitzen. Eher ist die Tendenz so, das schlechte Leiter gute Supraleiter sind.




\section*{Meißner-Ochsenfeld-Effekt}

Supraleiter sind perfekte Diamagnete. Ihr Inneres ist immer frei von einem magnetischen Feld. Dies wurde 1933 von Walter Meißner und Robert Ochsenfeld gefunden, als sie das magnetische Feld um einen supraleitenden Zylinder untersuchten\footcite{Meissner1933}. Dies ist eine Eigenschaft, die über die eines idealen Leiters hinausgeht. 


Wir betrachten den durch die Temperatur $T$ und das Magnetfeld $B$ aufgespannten Phasenraum. Oberhalb einer gewissen Temperatur $T_c$ ist das Material normalleitend, darunter entweder supraleitend oder ideal leitend. Wir gehen von Zustand  ($T > T_c$; $B=0$) zum Zustand ($T < T_c$; $B > 0$). Dabei können wir aber die Reihenfolge von Temperatur- und Magnetfeld-Änderung vertauschen.

Beim idealen Leiter ist das Magnetfeld im inneren zeitlich konstant. Dies ergibt sich aus dem Induktionsgesetz
\begin{equation}
    - \frac{\partial \bm{B}}{\partial t} = \nabla \times \bm{E} = 0 \quad ,
\end{equation}
weil $ \bm{E} = 0$ im Inneren eines idealen Leiter sein muss. Wenn man also zunächst das B-Feld einschaltet und dann die Temperatur reduziert, dann bleibt im Inneren ein Feld. Wenn man es andersherum macht, dann bleibt das Innere feldfrei. Beim Einschalten des B-Feldes wird ein Kreisstrom an der Oberfläche des idealen Leiters induziert, der gerade das B-Feld kompensiert.

Für Supraleiter haben nun Meißner und Ochsenfeld gemessen, dass das Innere immer feldfrei ist\sidenote{Eigentlich haben sie das Feld außerhalb des Zylinders gemessen und dann auf das innerhalb geschlossen.}, egal welchen der beiden Wege man geht. Für das Magnetfeld $\bm{B}_i$ im Inneren gilt also
\begin{equation}
    \bm{B}_i = \bm{B}_\text{ext} +  \mu_0 \bm{M} =\bm{B}_\text{ext}  \, (1 + \chi)  = 0 \quad ,
\end{equation}
also ist die magnetische Suszeptibilität $\chi = -1$, Supraleiter also perfekte Diamagnete.

Wir können aus dem Meißner-Ochsenfeld-Effekt weiterhin folgern, dass der supraleitende Zustand ein wirklicher thermodynamischer Zustand ist, also nur von den Zustandsgrößen abhängt und nicht vom Weg dahin.

\begin{questions} 
\item Suchen Sie im Internet nach graphischen Darstellungen der Prozessführung im Meissner-Ochsenfeld-Effekt und vergewissern Sie sich, dass alle den gleichen Effekt zeigen, obwohl er manchmal etwas anders dargestellt ist.
\end{questions}


 
\section*{Kritisches Magnetfeld}

Man beobachtet, dass die Abschirmung des externen Magnetfelds nur bis zu einer gewissen kritischen Feldstärke $B_c$ gelingt und darüber der supraleitende Zustand zusammenbricht. Es gilt also 
\begin{equation}
    - \mu_0 \bm{M} = 
    \left\{
    \begin{matrix}
    \bm{B}_\text{ext} \quad & \text{falls} \quad {B}_\text{ext}  < B_c & \quad \text{supraleitend} \\
    0    & \text{falls} \quad {B}_\text{ext}  \ge B_c  & \quad \text{normalleitend}
\end{matrix}
    \right. \quad .
\end{equation} 
Für die kritische Feldstärke $B_c$ findet man empirisch den Zusammenhang
\begin{equation}
    B_c(T) = B_c(0) \, \left[ 1 - \left( \frac{T}{T_c} \right)^2 \right]
\end{equation}
mit der Sprungtemperatur $T_c$.


\begin{marginfigure}
    \inputtikz{\currfiledir b_crit}
    \caption{Kritisches Magnetfeld für verschiedene Supraleiter (Daten aus  \cite{Hunklinger2014}).}
\end{marginfigure}


Typische kritische Magnetfeldstärken reiner Metalle liegen im Bereich von 10 bis 100~mT. Das ist insbesondere für technische Anwendungen sehr wenig. Ein supraleitender Magnet wäre so nicht zu realisieren. Bei Übergangsmetallen und Legierungen findet man allerdings ein anderes Verhalten, das als \emph{Typ-II-Supraleitung} bezeichnet wird. Dabei tritt eine sogenannte Shubnikov-Phase oder auch Vortex-Phase zwischen dem supraleitenden und normalleitenden Zustand auf. In dieser Phase bilden sich normalleitende Röhren innerhalb des Supraleiters, die das Magnetfeld hindurch leiten. Die eine kritische Feldstärke $B_c$ wird also durch zwei Feldstärken $B_{c1}$ und  $B_{c2}$ ersetzt, die die Grenze der Vortex-Phase beschreiben. Solche Typ-II-Supraleiter sind die, die heute technologisch verwendet werden.  Zur Beschreibung benutzt man die Ginzburg-Landau-Theorie, auf die wir hier wie auf die Typ-II-Supraleitung insgesamt nicht näher eingehen können.



\section{Flussquantisierung}

Der von einem supraleitenden Zylinder oder einer supraleitenden geschlossenen Leiterschleife umschlossene magnetische Fluss\sidenote{Feldstärke pro Fläche} ist quantisiert. Dies haben 1961 gleichzeitig R. Doll \& M. Näbauer in München und B.S. Deaver \& W.M. Fairbank in Stanford experimentell gefunden.

In den Experimenten bildete ein Bleifilm auf einem dünnen Quarz-Stäbchen einen  supraleitenden Hohlzylinder. Ein axiales Magnetfeld war angelegt während der Zylinder unter die Sprungtemperatur abgekühlt wurde. Danach wurde das externe Feld ausgeschaltet. Man beobachtet aber weiterhin ein magnetisches Moment in Zylinderrichtung. Dessen Größe kann durch ein Testfeld bestimmt werden. Man findet die Quantisierung des Flusses mit dem Flussquant
\begin{equation}
    \Phi_0 = \frac{h}{2 e} \quad .
\end{equation}
Wie wir unten sehen werden stammt die Zwei von den zwei Elektronen, die sich korreliert bewegen.

\begin{marginfigure}
    \inputtikz{\currfiledir flussquant}
    \caption{Flussquantisierung in einem supraleitenden Blei-Zylinder (\cite{Doll1961}).}
\end{marginfigure}




\section*{Wärmekapazität und Entropie}

Die spezifische Wärmekapazität eines Supraleiters weicht deutlich von der eines Normalleiters ab. Für ein gewöhnliches Metall hatten wir gefunden (Gl. \ref{eq:2_WK_Metall_ges}), dass  
\begin{equation}
    c_V = \gamma T + A T^3
\end{equation}
mit dem Beitrag der Elektronen proportional zu $T$ und dem der Phononen proportional zu $T^3$. Bei den hier betrachteten niedrigen Temperaturen spielt der Phononen-Beitrag keine Rolle. Aber auch der Elektronen-Beitrag ist anders. Man findet
\begin{equation}
    c_V \propto 
    \left\{
   \begin{matrix}
    e^{- \Delta / k_b T} \quad & \text{für} \quad T < T_c \\
    T  & \text{für} \quad T \ge T_c 
   \end{matrix}
    \right.
\end{equation}
mit einer charakteristischen  Energie $\Delta$.


\begin{marginfigure}
    \inputtikz{\currfiledir cv_al}
    \caption{Wärmekapazität  von \ch{Al}. Durch das Magnetfeld kann der supraleitende Zustand unterdrückt werden, so dass das  normalleitende Verhalten sichtbar wird (\cite{Phillips1959}).}
\end{marginfigure}



Durch Messung von $dS/ dT = c_p$ kann man die Entropie der supraleitenden Phase bestimmen. Man findet einen kleinen Unterschied im Vergleich zur normalleitenden Phase
\begin{equation}
    \Delta S = S_{SC} - S_N < 0 \quad \text{und} \quad |\Delta S| \approx 10^{-4} k_b T / \text{Atom}
    \quad .
\end{equation} 
Die supraleitende Phase ist also geordneter als die normalleitende, aber diese Ordnung betrifft nur sehr wenige Elektronen.


\section*{Isotopen-Effekt}

Die Sprungtemperatur $T_c$ hängt vom verwendeten Isotop ab. Solange das chemische Element (im Beispiel Zinn) das gleiche bleibt, ändert sich die elektronische Struktur nicht, sondern nur die Masse des Atomkerns und damit die Frequenz der Gitterschwingungen. Man findet
\begin{equation}
    T_c \propto \frac{1}{\sqrt{M}} \propto \omega_\text{Debye} \quad .
\end{equation}


\begin{marginfigure}
    \inputtikz{\currfiledir isotope}
    \caption{Isotopen-Effekt: Variation der kritischen Temperatur mit der Atommasse. Angegeben ist die mittlere Masse eines Isotopengemisches von Zinn (\ch{Sn}). (Daten aus  \cite{Hunklinger2014}. \label{fig:6_isotopeneffekt}}
\end{marginfigure}


\section*{London-Modell}

Als erstes Modell zur Erklärung der Supraleitung besprechen wir hier das London-Modell, das 1935 von Fritz und Heinz London aufgestellt wurde\footcite{London1935}. Die Idee ist, die Maxwell-Gleichungen beizubehalten und nur die Materie-Gleichungen so zu modifizieren, dass sie den verschwindenden Widerstand und den perfekten Diamagnetismus erklären können.

Für normalleitende Materie haben wir die Materie-Gleichungen
\begin{equation}
    \bm{H} = \frac{1}{\mu_0} \bm{B} \qquad\qquad
    \bm{D} = \epsilon_0 \bm{E} \qquad\qquad
    \bm{j} = \sigma \bm{E} \quad .
\end{equation}
Das Postulat ist nun\sidenote{siehe Anhang H.3 in \cite{Singleton_band_theory}}, dass in Supraleitern gilt
\begin{equation}
    \bm{j} = - \frac{1}{\mu_0 \lambda^2} \, \bm{A}
\end{equation}
mit dem Vektorpotential\sidenote{mit $\nabla \times \bm{A} = \bm{B}$} $\bm{A}$ und  der London-Länge 
\begin{equation}
    \lambda^2 = \frac{m_s}{n_s \mu_0 q_s^2} \quad .
\end{equation}
Wie wir unten sehen werden sind es nicht direkt die Elektronen, die zur Supraleitung führen. Daher sind hier alle Größen mit dem Index 's' versehen, um die supraleitenden Teilchen zu kennzeichnen. 

Die zeitliche Ableitung der Stromdichte ist dann\sidenote{siehe \cite{Gross_FK} oder \cite{Czycholl_theo_FK2}}
\begin{equation}
    \mu_0 \lambda^2  \frac{\partial  \bm{j}}{\partial t} = \bm{E} \quad .
\end{equation}
Dies ist die 1. London-Gleichung.
Nicht mehr die Stromdichte, sondern ihre zeitliche Ableitung ist proportional zum elektrischen Feld. Ohne Feld fließt also weiterhin Strom. Das ist Supraleitung.

Die Rotation der Stromdichte ist mit dem Magnetfeld verknüpft
\begin{equation}
    \nabla \times \bm{j} =  - \frac{1}{\mu_0 \lambda^2} \, \bm{B} \quad .
\end{equation}
Dies ist die 2. London-Gleichung.


Unter Zuhilfenahme der Maxwell-Gleichungen und ein paar weiteren Umformungen\sidenote{siehe \cite{Singleton_band_theory} Anhang H.3} findet man für das magnetische Feld
\begin{equation}
    \nabla^2 \bm{B} = \frac{1}{\lambda^2} \bm{B} \quad . \label{eq:6_B_decay}
\end{equation}
Betrachten wir dazu eine Grenzfläche zwischen Normalleiter ($x<0$) und Supraleiter ($x>0$) bei einem in z-Richtung orientierten Magnetfeld. Im Normalleiter sei das Feld homogen $\bm{B}_0$. Im Supraleiter ist die Lösung von Gl.\ref{eq:6_B_decay} dann
\begin{equation}
    \bm{B}(x>0) = \bm{B}_0 \, e^{- x / \lambda} \quad .
\end{equation}
Das Magnetfeld fällt also im Supraleiter mit der London-Länge (auch London'sche Eindringtiefe) exponentiell ab. Das Innere eines Supraleiters ist feldfrei, wie es der Meissner-Ochsenfeld-Effekt zeigt. Typische Werte von $\lambda$ liegen im Bereich von 10 bis einige 100~nm.




\section{Cooper-Paare}

Die phänomenologische London-Theorie macht keine Aussage über die mikroskopische Begründung für die geänderte Materiegleichung. Dies kommt erst 1957 mit der BCS-Theorie, nach J. Bardeen, L.N. Cooper und J.R. Schrieffer. Ein zentraler Bestandteil der Theorie sind Cooper-Paare. Hier geben wir nun sowohl die Ein-Elektron-Näherung auf, weil wir zwei korrelierte Elektronen betrachten. Wir verlassen auch die Born-Oppenheimer-Näherung, weil Elektron-Phonon-Wechselwirkungen wichtig werden.

In einem Gedankenexperiment starten wir von einem freien Elektronengas bei $T=0$. Es sind also alle Zustände für Elektronen bei Energien unterhalb der Fermi-Energie $E_F$ besetzt, bzw. alle Zustände im reziproken Raum innerhalb der Fermi-Kugel mit dem Radius $k_F$. Dann fügen wir zwei weitere, aber besondere Elektronen hinzu. Zwischen diesen beiden besonderen Elektronen soll eine schwach attraktive Wechselwirkung bestehen. Der Isotopen-Effekt (Abb.~\ref{fig:6_isotopeneffekt}) liefert die Begründung dazu, dass diese Wechselwirkung über Phononen erfolgt. 

Ein anschauliches Bild ist folgendes: ein Elektron bewegt sich durch den Kristall aus positiven Ionen. Diese werden leicht angezogen und im Kielwasser des Elektrons entsteht eine etwas erhöhte Dichte an Atom-Rümpfen. Diese etwas höhere Ladung wirkt dann anziehend auf das zweite Elektron. Der Abstand der beiden Elektronen ist durch die Zeit bestimmt, die die Atom-Rümpfe brauchen, um sich zu bewegen, also die Phonon-Frequenz. Typische Werte sind eben in der Größe von 100 nm, wie die London-Länge, und so groß, dass die Coulomb-Abstoßung  der Elektronen nicht ins Gewicht fällt, auch weil alle anderen Elektronen das Coulomb-Potential abschirmen.

Jenseits des anschaulichen Bildes kann man die Wechselwirkung als Austausch\sidenote{siehe Austausch-Boson-Modell in der Kernphysik} von virtuellen Phononen mit dem Wellenvektor $\bm{q}$ modellieren. Vor dem Austausch haben die beiden Elektronen die Wellenvektoren $\bm{k}_1$ und $\bm{k}_2$, nach dem Austausch $\bm{k}_1 + \bm{q}$ und $\bm{k}_2 - \bm{q}$. Der Gesamtimpuls bleibt also erhalten. Wir sind weiterhin am absoluten Temperatur-Nullpunkt und alle Zustände unterhalb $E_F$ durch die 'anderen' Elektronen besetzt. Die beiden besonderen Elektronen können  also nur Zustände im Bereich $E_F$ und $E_F + \hbar \omega_D$ annehmen. Im reziproken Raum entspricht das einer Kugelschale zwischen $k_F$ und $k_F + m \omega_D / (\hbar k_F)$. Der Überlapp zwischen den Kugelschalen der beiden Elektronen bestimmt also die Stärke der Wechselwirkung und die  Energieabsenkung. Maximale Absenkung erhalten wir, wenn die Mittelpunkte der Kugelschalen übereinstimmen, also 
\begin{equation}
    \bm{K} = \bm{k}_1 + \bm{k}_2 = 0 \quad \text{bzw.} \quad  \bm{k}_1 = - \bm{k}_2  \quad .
\end{equation}
Dabei ist $\bm{k}_i$ der Wellenvektor der Wellenfunktion des $i$-ten Elektrons, nicht sein Impuls.
Ein Cooper-Paar wird also aus zwei Elektronen gebildet, deren Wellenvektoren sich gerade gegenüberstehen. 


\begin{marginfigure}
    \inputtikz{\currfiledir cooper_sketch}
    \caption{Der Austausch eines Phonons ist möglich im Überlapp der Ringe. Dieser wird maximal, wenn $\bm{K}= 0$.}
\end{marginfigure}

Durch die attraktive Wechselwirkung wird die Energie des Paares um den Betrag $\Delta$ gegenüber den Einzel-Energien abgesenkt. Die Energie eines Cooper-Paares ist also ungefähr
\begin{equation}
    E \approx 2 E_F - \Delta  \quad .
\end{equation}
Mit ein paar Annahmen über die Wechselwirkung kann man ausrechnen\sidenote{siehe \cite{Hunklinger2014} oder \cite{Gross_FK}. Gross diskutiert auch den manchmal auftretenden Unterschied im Faktor 2 (oder 4) im Exponenten.}, dass die Energie-Absenkung $\Delta$
\begin{equation}
    \Delta = 2 \hbar \omega_D \, e^{-2 / D(E_F) V_0}
\end{equation}
beträgt. $V_0$ beschreibt die Stärke der Elektron-Phonon-Wechselwirkung und $D(E_F)$ die Zustandsdichte an der Fermi-Kante.

Ein Cooper-Paar besteht aus zwei Elektronen mit entgegengesetztem Spin, die zu einem Gesamtspin $S=0$ kombinieren. Von außen gesehen ist ein Cooper-Paar ein Boson, auch wenn es aus zwei Fermionen aufgebaut ist.


\section*{Der BCS-Grundzustand}

Die Unterscheidung zwischen „normalen” und „wechselwirkenden” Elektronen im letzten Abschnitt ist natürlich nur ein Gedankenexperiment. In Wirklichkeit wirkt die Elektron-Phonon-Wechselwirkung bei allen Elektronen. Der Abstand der beiden Elektronen in einem Cooper-Paar ist jedoch viel größer als der mittlere Abstand zwischen allen Elektronen. An dieser Stelle bricht das Bild der Cooper-Paare eigentlich zusammen. Man braucht Vielteilchen-Quantenmechanik.\sidenote{siehe z.B. \cite{Czycholl_theo_FK1}}. Dabei stellt man fest, dass sehr viele Elektronen einen gemeinsamen Quantenzustand besetzen.
Das ähnelt der Kondensation von Bosonen, beispielsweise bei der Bose-Einstein-Kondensation von kalten Atomen oder der stimulierten Emission von Photonen (ebenfalls Bosonen) im Laser. Es ist aber nicht völlig dasselbe, weil natürlich alle Elektronen mit allen wechselwirken und nicht nur paarweise.

Es sind also alle Elektronen an der Supraleitung beteiligt und die Temperatur ist am absoluten Nullpunkt. Das ist der 
BCS-Grundzustand. Ich will hier nicht auf die Konstruktion der Wellenfunktion eingehen\sidenote{Siehe dazu \cite{Gross_FK}}, sondern sie nur allgemein schreiben als
\begin{equation}
    \psi(\bm{r}) = \sqrt{n} \, e^{i \Theta(\bm{r})}
\end{equation}
mit der überall konstanten Dichte $n = \braket{\psi | \psi}$ an Cooper-Paaren und der Phase $\Theta(\bm{r})$. Die  Gesamtenergie reduziert sich bei der Kondensation um % Gross (13.5.82)
\begin{equation}
    E_\text{Kondensat} = - \frac{1}{4} \, D(E_F) \, \Delta^2 \quad . \label{eq:6_E_kondensation}
\end{equation}


\section*{Fluss-Quantisierung}

Die gemeinsame Wellenfunktion für alle Cooper-Paare kann die Quantisierung des magnetischen Flusses in einem supraleitenden Ring erklären. Dazu berechnen wir erst den elektrische Strom $\bm{j}$ von Cooper-Paaren (=2 Elektronen) der Ladung $q_s = -2 e$ und Masse $m_s = 2 m_e$   als Erwartungswert des Geschwindigkeits-Operators
\begin{equation}
    \bm{j} = \frac{q_s}{m_s} \, \braket{\psi | \hat{\bm{v}} | \psi}
\end{equation}
 mit 
 \begin{equation}
   \hat{\bm{p}} = m_s \hat{\bm{v}} +  q_s \bm{A}   \quad \text{also} \quad m \hat{\bm{v}} = -i \hbar \nabla - q_s \bm{A}
   \quad .
 \end{equation}
 Damit erhalten wir alles zusammen\sidenote{Der Gradient liefert das $i$, das sich somit aufhebt.}
 \begin{equation}
  \lambda^2 \mu_0 \,  \bm{j} =   \frac{ \hbar }{q_s} \, \nabla \Theta (\bm{r}) - \bm{A} \label{eq:6_suprastrom}
 \end{equation}
 mit der London-Länge $\lambda^2 = m/ (n \mu_0 q)$ wie oben.
 Von hier kommt man also zur London-Theorie zurück. Eine makroskopische Wellenfunktion ist dazu ausreichend.

Nun betrachten wir einen supraleitenden Torus, der von einem Magnetfeld durchsetzt ist.  Das Innere des Supraleiters ist frei von Feldern und Strömen, also ist Gl.~\ref{eq:6_suprastrom} Null und wir schreiben
\begin{align}
     \hbar  \, \nabla \Theta (\bm{r})  &=  q_s \bm{A} \\
     \hbar \oint \nabla \Theta (\bm{r})& =  q_s \oint \bm{A}  \\
     \hbar (\Theta_2 - \Theta_1) & = q_s \int \bm{B} \\
   \hbar \, 2 \pi \, s & = q_s \Phi \qquad s \in \mathbb{N}  \quad ,
\end{align}
wobei wir im zweiten Schritt den Satz von Stokes ausgenutzt haben und im dritten, dass $|\Psi|$ nach einem Umlauf in Kreis eindeutig definiert sein muss, sich die Phase also nur um $2\pi$ unterscheiden darf. Damit bekommen wir die Flussquantisierung ($q_s = -2e$)
\begin{equation}
    \Phi = \frac{h}{2 e} \, s = s \, \Phi_0 \quad \text{mit} \quad s \in \mathbb{N} \quad .
\end{equation}


\section*{Zustandsdichte}


Die Zustandsdichte freier Elektronen in einem Normal-Leiter ist $D_{NL} \propto \sqrt{E}$. Weil aber der hier interessierende Energiebereich nur wenige meV um die Fermi-Energie beträgt, können wir $D_{NL}$ als konstant annehmen. Welche Form hat die Zustandsdichte in einem Supraleiter? Dabei müssen wir aufpassen, welche Teilchen wir betrachten. Cooper-Paare bestehen aus zwei korrelierten Elektronen. Diese können wir nicht direkt mit einzelnen Elektronen vergleichen.\sidenote{Die Wellenfunktion ist entweder eine Funktion von einer oder von zwei Ortskoordinaten} Wir sprechen von Einteilchen- und Zweiteilchen-Zustandsdichten.

Die Zweiteilchen-Zustandsdichte hat zunächst einen deltaförmigen Zustand an der BCS-Grundzustandsenergie, in dem sich alle Cooper-Paare befinden. Wenn man ein Cooper-Paar anregt, dann wird es zerstört. Es bleibt ein Elektronen des Cooper-Paars zurück, jetzt zusammen mit einem Loch. Dieses Elektron-Loch-Zweiteilchen ist ein Quasiteilchen, das  man manchmal Bogolon nennt\footcite{Kopitzki_FK}. Ein weiteres Bogolon entsteht aus dem angeregten Elektron. Die Mindest-Energie zur Anregung eines Cooper-Paares, bzw. zur Erzeugung eines Bogolon aus einem Cooper-Paar, ist gerade die Bindungsenergie $\Delta$. Näher an der Fermi-Energie ist kein Zustand für das angeregte Elektron frei.

Durch die Supraleitung darf sich die Summe der Zustände nicht ändern. Damit ergibt sich für die Bogolonen 
\begin{equation}
    D_{SL}(E_k) = \left\{ 
    \begin{matrix}
    D_{NL} \frac{E_k}{\sqrt{E_k^2 - \Delta^2}} \quad &\text{für} \quad E_k > \Delta \\
     0 & \text{sonst} 
    \end{matrix}
    \right. \label{eq:6_Dos_SL}
\end{equation}
mit der Zweiteilchen-Energie $E_k$. Die Cooper-Paare sind in dieser Darstellung eine Delta-Funktion bei $E_k = 0$ (rot in Abb.~\ref{eq:6_Dos_SL}).

\begin{marginfigure}[-40mm]
    \inputtikz{\currfiledir dos_SL}

    \inputtikz{\currfiledir dos_SL_1T}

    \caption{Zustandsdichte eines Supraleiters im Zweiteilchen-Modell (oben) und im Einteilchen-Modell (unten).}
\end{marginfigure}

Wenn man darauf verzichtet, die Cooper-Paare einzuzeichnen, dann kann man auch eine Einteilchen-Zustandsdichte zeichnen, die dann eine Lücke im Bereich $E_F - \Delta$ bis $E_F + \Delta$ besitzt und außerhalb analog zu Gl.~\ref{eq:6_Dos_SL} verläuft.


\section*{Kritische Temperatur, Strom, Magnetfeld}

Sobald wir nicht mehr am absoluten Nullpunkt der Temperatur sind, existieren Cooper-Paare und Bogolonen gleichzeitig. Mit steigender Temperatur nimmt die Dichte der Cooper-Paare ab. Je weniger Cooper-Paare es aber gibt, desto schlechter ist die Kondensation, und die Energieabsenkung pro Paar wird geringer. $\Delta$ wird temperaturabhängig. Man findet durch numerisches Lösen einer nichtlinearen Differentialgleichung\footcite{Gross_FK} für $T \approx T_c$
\begin{equation}
    \frac{\Delta(T)}{\Delta(T=0)} \approx 1.74 \sqrt{1 - \frac{T}{T_c}}
\end{equation}
und einen Zusammenhang zwischen kritischer Temperatur und Bandlücke bei $T=0$
\begin{equation}
    2 \Delta(0) = 3.52  \, k_b  \, T_c \quad .
\end{equation}

Ebenso bricht die Supraleitung zusammen, wenn der Strom im Supraleiter zu groß wird. Sobald die kinetische Energie der Cooper-Paare die durch die Kondensation gewonnene Energie (Gl.~\ref{eq:6_E_kondensation}) übersteigt, lösen sich die Cooper-Paare selbst auf. Dann wird die kritische Geschwindigkeit $v_c$ erreicht, bei der 
\begin{equation}
    E_\text{kin} = n_s \frac{m_s v_c^2}{2} =  | E_\text{Kondensat} |  = \frac{1}{4} \, D(E_F) \, \Delta^2 \quad .
\end{equation}
Der zugehörige kritische Strom ist % XXX check sign -q_s
\begin{equation}
    j_c = - n_s \, q_s \,  v_c = \frac{\sqrt{6 } \,  e \, n_s}{\hbar k_F} \, \Delta \quad ,
\end{equation}
wobei wir $D(E_F)$ durch $k_F$ ausgedrückt haben. Dieser kritische Strom produziert ein Magnetfeld an der Oberfläche des Drahtes ($R \gg \lambda$), wodurch wir eine kritische Feldstärke erhalten von\footcite{Hunklinger2014}
\begin{equation}
 B_c = \mu_0 \, \lambda \, j_c  \propto \Delta \quad .
\end{equation}
Die Bandlücke $\Delta$ des Supraleiters in der BCS-Theorie ist also ausreichend, um alle obengenannten kritischen Temperaturen, Ströme und Magnetfelder zu erklären.

\section{Tunneln von Elektronen}

Die Zustandsdichte in der Nähe des Fermi-Niveaus und insbesondere die Lücke darin lässt sich sehr elegant durch die Tunnelspektroskopie untersuchen. Dazu benötigt man einen Tunnel-Kontakt zwischen zwei Leitern. Technisch einfach geht das, wenn man erst einen Streifen des ersten Materials aufdampft, dann einen dünnen (wenige Nanometer) Isolator, und dann quer dazu einen Streifen  des anderen Materials. So kann man beide Streifen einzeln kontaktieren, eine Potentialdifferenz $U$ über den Isolator anlegen und einen Tunnel-Strom $I$ fließen lassen. Man misst dann den Strom als Funktion der angelegten Potentialdifferenz.

Mikroskopisch gesehen verschiebt die Spannung $U$ die Fermi-Niveaus rechts und links der Tunnel-Barriere gegeneinander. Dadurch kommen besetzte Niveaus auf der einen Seite energetisch auf die gleiche Höhe wie unbesetzte Niveaus auf der anderen Seite. In diesem Fall können die Elektronen mit einer gewissen Wahrscheinlichkeit durch die Barriere tunneln und ein Strom fließt. Der Strom ist also das Integral über die gegeneinander verschobenen Zustandsdichten multipliziert mit ihrer jeweiligen Besetzung
\begin{equation}
    I_\text{L $\rightarrow$ R} \propto \int D_\text{L}(E) f(E) \, \, D_\text{R}(E + e U) [1 - f(E + e U)] \, \, dE \quad .
\end{equation}
Genauso kann auch ein Strom von rechts nach links fließen, so dass der Netto-Strom die Differenz der beiden ist:
\begin{equation}
    I \propto \int D_\text{L}(E) \, D_\text{R}(E + e U)  \, [f(E) - f(E + e U)] \, dE \quad .
\end{equation}

\begin{marginfigure}
    \inputtikz{\currfiledir tunnel_junction}
    \caption{Tunnelstrom durch einen \ch{Al}/\ch{Al_2O_3}/\ch{Pb} - Tunnelkontakt bei 4.2~K bzw 1.6~K. Im zweiten Fall ist \ch{Pb} supraleitend. $dI/dV$ ist proportional zur Zustandsdichte, ausgeschmiert mit $k_b T$
(\cite{Giaever1960}).}
\end{marginfigure}
  

Üblicherweise ist $eU \ll E_F$, und bei Normalleitern kann die Zustandsdichte  im relevanten Energiebereich als konstant angesehen werden, also $D_{n}(E_F)  \approx  D_{n}(E_F + e U)$. Damit erhalten wir
\begin{eqnarray}
    I_{nn} \propto D_{n}(E_F)  D_{n}(E_F) \, e U = G_{nn} U \quad .
\end{eqnarray}
Das ist ein Ohm'scher Verlauf mit dem konstanten Leitwert $G_{nn}$ zwischen zwei Normalleitern.

Nun ersetzen  wir einen der beiden Normalleiter durch einen Supraleiter mit der Zustandsdichte nach 
Gl.~\ref{eq:6_Dos_SL}. Der Strom ist dann 
\begin{align}
    I_{ns} &\propto D_n(E_F) \, D_{n}(E_F)  \, \int  \frac{D_s(E)}{D_n(E_F)} [f(E) - f(E + e U)] \, dE \\
      & = \frac{ G_{nn}}{e} \, \int  \frac{D_s(E)}{D_n(E_F)} [f(E) - f(E + e U)] \, dE  \quad . \label{eq:6_Ins}
\end{align}
Wir betrachten  wieder den Leitwert, hier $G_{ns}$ 
\begin{align}
    G_{ns} & = \frac{d I_{ns}}{d U} =  G_{nn} \, \int  \frac{D_s(E)}{D_n(E_F)} 
    \left[ - \frac{\partial f(E+ eU)}{\partial (eU)} \right] \, dE  \\
      & \approx   G_{nn} \frac{D_s(eU)}{D_n(E_F)}  =  
      G_{nn} \, \Re \left\{ \frac{eU}{\sqrt{ (eU)^2 - \Delta^2(eU)}} \right\} \quad .
\end{align}
Der Term in eckigen Klammern ist ähnlich einer Delta-Funktion bei $eU$ mit einer Breite $4 k_b T$ und Fläche Eins und löst so das Integral auf. Im zweiten Schritt haben wir $T \rightarrow 0$ angenommen. Der Leitwert beim Tunnel, also $dI/dU$, liefert somit bei tiefen Temperaturen direkt die Zustandsdichte, ggf. ausgeschmiert mit $2 k_b T$.


\section*{Josephson-Effekt}


Nun liegt es nahe, auch zwei Supraleitern durch einen Tunnelkontakt zu verbinden und dann nicht Elektronen, sondern Cooper-Paare tunneln zu lassen, wenn die Barriere dünn genug ist\sidenote{Nobelpreis Brian David Josephson 1973}. Dabei müssen wir dann aber die makroskopische Wellenfunktion berücksichtigen. Wir nehmen an, dass beide Supraleiter identisch sind. Jede Seite ($i=1, 2$) wird beschrieben durch die Wellenfunktion 
\begin{equation}
  \Psi_i = \sqrt{n_i} \, e^{i \Theta_i} \quad .
\end{equation}
In der Schrödinger-Gleichung gibt es einen schwachen Kopplungsterm $T$, der die Wellenfunktionen nicht wesentlich ändern soll, so dass wir Störungstheorie betreiben können:
\begin{align}
    i \hbar \dot{\Psi}_1 & =  E_1 \Psi_1 + T \Psi_2 \\ 
    i \hbar \dot{\Psi}_2 & =  E_2 \Psi_2 + T \Psi_1 \quad .
\end{align}
Die Potentialdifferenz über die Tunnelbarriere verschiebt die Energie-Eigenwerte, also $E_2 - E_1 = e U$. Das setzen wir alles ein und separieren nach Real- und Imaginärteil. Wir erhalten mit der Phasendifferenz $\delta = \Theta_2 - \Theta_1$
\begin{align}
    \dot{n}_1 & = \frac{2 T}{\hbar} \, \sqrt{n_1 n_2} \,  \sin(\delta) \\
    \dot{n}_2 & = -\frac{2 T}{\hbar} \, \sqrt{n_1 n_2} \,  \sin(\delta) \\
    \dot{\Theta}_1 &= \frac{T}{\hbar} \, \sqrt{\frac{n_2}{n_1}} \cos(\delta) - \frac{E_1}{\hbar} \\
    \dot{\Theta}_2 &= \frac{T}{\hbar} \, \sqrt{\frac{n_1}{n_2}} \cos(\delta) + \frac{E_2}{\hbar} \quad .
\end{align} 
Die Differenz der letzten beiden Gleichungen ergibt
\begin{equation}
   \hbar \dot{\delta} =  \hbar ( \dot{\Theta}_2 -  \dot{\Theta}_1 ) = - (E_2 - E_1) = 2eU \quad .
\end{equation}


\begin{marginfigure}
    \inputtikz{\currfiledir josephson}
    \caption{Strom durch einen \ch{Pb}/\ch{PbO_x}/\ch{Pb} Tunnelkontakt (Daten aus \cite{Langenberg1966})
    \label{fig:6_SC_tunnel}}
\end{marginfigure}


Beim Experiment in Abbildung \ref{fig:6_SC_tunnel} wird die Spannung $U_{ext}$ einer Stromquelle variiert und dabei der Strom $I$ durch den Tunnelkontakt und die Potentialdifferenz $U$ über den Kontakt gemessen. Der Stromkreis und die Spannungsquelle besitzen einen (Innen-) Widerstand, der den Strom limitiert.

Betrachten wir zunächst die Situation, dass keine  Potentialdifferenz $U$ über die Barriere gemessen wird. Damit ist die Phasendifferenz $\delta$ zeitlich konstant und $ \dot{n}_1 = - \dot{n}_2$. Es fließt ein Suprastrom (Strom von Cooper-Paaren), ohne dass ein Potential an der Barriere abfällt. Der Strom hängt periodisch von der Phasendifferenz der beiden makroskopischen Wellenfunktionen ab:
\begin{equation}
    J_s(\delta) = J_c \, \sin \delta \quad . \label{eq:6_Josephson_DC}
\end{equation}
 Dies ist der Josephson-Gleichstrom-Effekt (Josephson-DC-Effekt). Die Maximalstromstärke $J_c$ ist durch die  Stärke $T$ des Tunnelkontakts gegeben. Die beobachtete Stromstärke $I$ wird von der  Stromquelle bestimmt. Diese stellt die Phasendifferenz $\delta$ ein. Gleichung \ref{eq:6_Josephson_DC} ist also kausal rückwärts zu verstehen.



Nun erhöhen wir die externe Spannung $U_{ext}$, der Strom steigt und übersteigt irgendwann $J_c$. Ab dann ist die gemessene Potentialdifferenz $U$ nicht mehr Null, und die Phasendifferenz $\delta$ wird zeitabhängig:
\begin{equation}
    \delta(t) = \frac{2eU}{\hbar} t + \delta(0)  = \omega_J t +  \delta(0)  \quad .
\end{equation}
Es fließen nun zwei Ströme gleichzeitig durch den Tunnelkontakt: weiterhin ein Suprastrom aus Cooper-Paaren, der zeitlich oszilliert mit 
\begin{equation}
    J_s(t) = J_c \, \sin  ( \omega_J t +  \delta(0) ) \quad .
\end{equation}
Hinzu kommt ein Strom aus Quasiteilchen (Bogolonen). Dieser berechnet sich analog zu Gleichung ~\ref{eq:6_Ins}, wenn man für jeden Kontakt die Zustandsdichte des Supraleiters einsetzt. Die Kante in der Strom-Spannungs-Kennlinie liegt daher bei $e U = 2 \Delta$. In Abbildung  \ref{fig:6_SC_tunnel} ist nur dieser zeitlich konstante Strom zu sehen, da die zeitliche Oszillation des Suprastroms viel zu schnell ist. Im Experiment muss berücksichtigt werden, dass der Tunnelkontakt zusätzlich eine Kapazität und einen ohmschen Widerstand aufweist. Dies führt zur  Hysterese in der Kennlinie.
Die schnelle zeitliche Oszillation des Suprastroms wird  als Josephson-Wechselstrom-Effekt (Josephson-AC-Effekt) bezeichnet. Dabei treten sehr hohe Frequenzen auf. Bei 100~\textmu V werden beispielsweise Frequenzen von 50~GHz erreicht. Dadurch kann einerseits $e/h$ bestimmt werden und andererseits kann die Spannung sehr genau über eine Frequenzmessung ermittelt werden.

Beschleunigte Ladungen emittieren Strahlung. Ein hochfrequenter Wechselstrom ist demnach ein Emitter. Die tunnelnden Cooper-Paare im Josephson-AC-Effekt werden in der Potentialdifferenz $U$ über die Tunnelbarriere beschleunigt. Da die Zustandsdichte deltaförmig ist, können sie diese Energie jedoch nicht aufnehmen. Daher strahlen sie die Energiedifferenz als Radiofrequenz-Photon ab.



\section*{Quanteninterferenz}


Nun schalten wir zwei Josephson-Kontakte parallel: zwischen den Kontakten 1 und 2 gibt es zwei supraleitenden Pfade, je über die Tunnelkontakte A und B. Uns interessiert der (Tunnel-) Strom zwischen 1 und 2. Gleichzeitig durchsetzt ein Magnetfeld die so gebildete Leiterschleife.

\begin{marginfigure}
    \inputtikz{\currfiledir squid_sketch}
    \caption{Zwei parallel geschaltete Josephson-Kontakte.}
\end{marginfigure}


Sei $\delta_A$ der Phasenunterschied der makroskopischen Wellenfunktion der Cooper-Paare auf dem Pfad 1--A--2, und $\delta_B$ analog über Tunnelkontakt B. Dann ist die Phasendifferenz entlang des geschlossenen Kreises 1--A--2--B--1 $\delta_A - \delta_B$ und die Flussquantisierung verlangt
\begin{equation}
    \delta_A - \delta_B = \frac{2e}{\hbar} \Phi = 2 \pi \frac{\Phi}{\Phi_0}
\end{equation}
mit dem magnetischen Fluss $\Phi$ durch die Schleife. Damit können wir jede der beiden Phasen schreiben als
\begin{equation}
    \delta_{A,B} = \delta_0 \pm  \frac{e}{\hbar} \Phi = \delta_0 \pm \pi \frac{\Phi}{\Phi_0}
\end{equation}
mit einer mittleren Phase $\delta_0$. Der Gesamtstrom durch dieses Konstrukt ist nun die Summe der Ströme über die Tunnelkontakte A und B, wie im letzten Abschnitt, also
\begin{equation}
    I = I_A + I_B = J_c \left( \sin \delta_A +  \sin \delta_B \right) =
     2 J_c \sin \delta_0 \,  \cos \left( \frac{\pi \Phi}{\Phi_0}  \right)  \quad .
\end{equation}
Die Oszillationen im Strom zählen also die Flussquanten in der Schleife.  Abbildung \ref{fig:6_squid_data} am Anfang des Kapitels zeigt ein Beispiel.

Diese Anordnung nennt man 'superconducting quantum interference device' (SQUID). Sie wird zur sehr empfindlichen Messung von Magnetfeldern benutzt, beispielsweise in der Medizin (Hirnströme!) oder der Archäologie.


\newpage

\section{Zusammenfassung}

\textit{Schreiben Sie hier ihre persönliche Zusammenfassung des Kapitels auf. Konzentrieren Sie sich auf die wichtigsten Aspekte und die am Anfang genannten Ziele des Kapitels.}

\vspace*{10cm}
\printbibliography[segment=\therefsegment,heading=subbibliography]




 

     
%%-----------------------


\renewcommand{\kapitelname}{Anhang\ }

\addcontentsline{toc}{part}{Anhang} 

\appendix
\appendixpage


\renewcommand{\lastmod}{4. April 2023}
\renewcommand{\chapterauthors}{Markus Lippitz}


\chapter{Julia und Pluto}


Wir benutzen in dieser Veranstaltung die Programmiersprache \emph{Julia}\sidenote{\url{https://julialang.org}} für graphische Veranschaulichungen und numerische 'Experimente'. Ich  bin überzeugt, erst  wenn man einen Computer überreden kann, etwas zu tun, ein Model darzustellen, einen Wert auszurechnen, erst dann hat man es wirklich verstanden. Vorher hat man nur die ganzen Probleme noch nicht gesehen.  

Man kann Julia mit verschiedenen Benutzeroberflächen verwenden. Wir benutzen \emph{Pluto}.\sidenote{\url{https://github.com/fonsp/Pluto.jl}}

\section{Julia}

Julia ist eine Programmiersprache, die für Numerik und wissenschaftliches Rechnen entwickelt wurde. Sie ist ein Mittelding zwischen Matlab, Python und R. Aus meiner Sicht übernimmt sie jeweils das Beste aus diesen Welten und eignet sich so gerade für Einsteiger. Wir werden im Laufe des Semesters verschiedene Beispiel-Skripte zusammen besprechen, und es wird auch numerische Übungsaufgaben geben.


\subsection{Ein Beispiel}

Lassen Sie uns zunächst ein einfaches Beispiel betrachten.

\begin{jllisting}
using Plots
x = range(0, 2 * pi; length=100)
plot(x, sin.(x); label="ein Sinus")
\end{jllisting}

Für manche Dinge benötigt man Bibliotheken, die man mit \jlinl{using} laden kann. Halten sie sich bei der Auswahl der Bibliotheken zunächst an die Beispiele, die ich zeige.

Dann definieren wir eine Variable  \jlinl{x} (einfach durch benutzen) als äquidistanter 'Zahlenstrang' zwischen 0 und $2 \pi$ mit 100 Werten. Funktionen wie  \jlinl{range} haben immer benötigte Parameter, die über ihre Position in der Parameterliste definiert sind (hier: Anfangs- und End-Wert), sowie weitere optionale. Diese folgen nach einem Semikolon in der Form \texttt{<Parameter>=<Wert>}.

Schließlich zeichnen wir die Sinus-Funktion über diesen Wertebereich. Beachten sie den Punkt in \jlinl{sin.(x)}. Er bedeutet 'wende \jlinl{sin} auf alle Elemente von \jlinl{x} an'. Das ist sehr praktisch.


\subsection{Informationsquellen}

Aktuell ist die Version 1.8.5. Mit der Version 1.0 hat sich einiges geändert. Ignorieren sie Webseiten, die älter als 3 Jahre sind, bzw. die sich auf eine Version vor 1.0 beziehen.

\begin{description}

\item[Offizielle Dokumentation] auf der website\sidenote{\url{https://docs.julialang.org/en/v1/}}. Oder fragen Sie google mit 'Julia' als Stichwort oder mit der Bibliothek / Funktion und angehängter Endung '.jl' .

\item[Beispiele] Julia by example\sidenote{\url{https://juliabyexample.helpmanual.io/}}, Think julia\sidenote{\url{ https://benlauwens.github.io/ThinkJulia.jl/latest/book.html}}, Introduction to Computational Thinking\sidenote{\url{https://computationalthinking.mit.edu/Fall22/ }}

\item[Unterschiede] Vergleich\sidenote{\url{https://docs.julialang.org/en/v1/manual/noteworthy-differences/
}}  mit  Matlab,  Python und anderen Sprachen. Und als Übersichtstabelle\sidenote{\url{https://cheatsheets.quantecon.org/}}

\item[Cheat Sheets] Allgemein\sidenote{\url{https://juliadocs.github.io/Julia-Cheat-Sheet/
}} und für Plots\sidenote{\url{https://github.com/sswatson/cheatsheets/}} 

\end{description}



\section{Benutzeroberflächen}


Es gibt verschiedene Möglichkeiten, wie man kürzere oder längere Programme in Julia schreiben kann. Hier eine Auswahl

\begin{description}
\item[Kommandozeile und Editor] Man kann Julia interaktiv an der Kommandozeile (REPL, read-eval-print loop) benutzen. In einem externen Editor  könnte man wiederholende Kommandos in Skript-Dateien schreiben.

\item[IDE] Das geht komfortabler mit einer integrierten Umgebung, beispielsweise einer Julia-Erweiterung \sidenote{\url{https://www.julia-vscode.org/}}
für Visual Studio Code. Das ist sicherlich die Herangehensweise bei  größeren Projekten.

\item[Jupyter notebook] Jupyter\sidenote{\url{https://jupyter.org/}} setzt sich zusammen aus Julia, Python und R. Diese drei Sprachen kann man in einem Notebook-Format benutzen. Programmcode steht dabei in Zellen, die Ausgabe und auch beschreibender Text und Grafiken dazwischen. Das eignet sich besonders, wenn Rechnungen von Beschreibungen oder Gleichungen begleitet werden sollen, beispielsweise in (Praktikums-)Protokollen oder Übungsaufgaben. 

Mathematica hat ein ähnliches Zellen-Konzept. Ein Nachteil ist, dass Zellen den Zustand des Kernels in der Reihenfolge ihrer Ausführung beeinflussen. Die Reihenfolge muss aber nicht der in der Datei entsprechen; insbesondere ändert ein Löschen der Zellen den Kernel nicht. Das kann sehr verwirrend sein, oder man muss der Kernel oft neu starten.

\item[Pluto]  Man kann auch in Pluto\sidenote{\url{https://github.com/fonsp/Pluto.jl}} Programmcode, Text und Grafik mischen. Das Zellen-Konzept von Pluto ist das aber von Excel, limitiert auf eine Excel-Spalte. Die Anordnung der Gleichungen in den Zellen spielt keine Rolle. Alles wird nach jeder Eingabe neu evaluiert. Eine Logik im Hintergrund sorgt dafür, dass nur unbedingt notwendige Berechnungen neu ausgeführt werden. Aus meiner Sicht sollte das für Anfänger intuitiv zu bedienen sein und für kleiner Projekte völlig ausreichen sein. \emph{Wir benutzen Pluto als Oberfläche in dieser Veranstaltungen.}

\end{description}



\section{Installation}


 Installieren Sie   Julia und Pluto auf ihrem Computer. Eine gute Anleitung ist am MIT\sidenote{\url{https://computationalthinking.mit.edu/Fall22/installation/}}. Kurzfassung: Julia vom website installieren, dann in Julia das Pluto-Paket installieren (\jlinl{import Pkg; Pkg.add("Pluto")}) und aufrufen via \jlinl{using Pluto; Pluto.run()}. 





\section{Benutzung von Pluto}

Eine schöne Einführung in Pluto (und Julia) gibt es auf der Pluto homepage\sidenote{\url{https://github.com/fonsp/Pluto.jl/wiki}},
 am MIT 
(hier\sidenote{\url{https://computationalthinking.mit.edu/Fall22/basic_syntax/}}
bzw. eigentlich die ganze site)
und am WIAS.\sidenote{\url{https://www.wias-berlin.de/people/fuhrmann/SciComp-WS2021/assets/nb01-first-contact-pluto.html}}

\begin{itemize}
\item Shift-Enter führt eine Zelle aus

\item Der Ausführungs-Optimierer verlangt, dass jede Zelle einen geschlossenen Block bildet. Also darf da nur ein Kommando stehe, oder mehrere müssen mit  \jlinl{begin} ... \jlinl{end}  geschachtelt werden.

\item Jede Zelle hat nur eine Ausgabe, die der letzten Zeile. Die Ausgabe steht über der Zelle selbst.

\item Pluto verwaltet Bibliotheken selbständig, einfach mit \jlinl{using}  benutzen, installiert wird automatisch.

\item Pluto speichert automatisch alles. Man kann aber die Datei umbenennen / bewegen.



\end{itemize}



%-------------------



\renewcommand{\lastmod}{11. Juni 2023}
\renewcommand{\chapterauthors}{Markus Lippitz}


\chapter{Bandlücken}


\section*{Was bisher geschah}


Im Kapitel \ref{chap:bandstruktur} zur Bandstruktur hatten wir die Bandlücke anhand der energetisch am tiefsten liegende Kreuzung zweier Parabeln diskutiert. Wir haben Gl \ref{eq:3_SG_rezi} eingeschränkt auf nur drei Koeffizienten $C$, nämlich $C_k$,  $C_{k - g}$ und  $C_{k + g}$ und sind so bei Gl.  \ref{eq:3_SG_empty_lattice}  gelandet. Dann haben wir festgestellt, dass in der Nähe der Grenze der Brillouinzone eine Resonanz in der Energie auftritt und nur zwei der 3 Koeffizienten wirklich relevant sind. Das führte dann zu Gl. \ref{eq:3_SG_empty_lattice_2}, die nur noch $C_k$ und  $C_{k - g}$ enthält. Die haben wir schließlich gelöst und eine Aufspaltung der Breite $2 V_g$ gefunden.

Im darauf folgenden Abschnitt 'Anschauliche Interpretation II' haben wur die Analogie zur Beugung von Wellen an Gittern und zur Laue-Bedingung  gezogen: Die Koeffizienten $C_x$ beschreiben ja gerade ebene Wellen mit dem Wellenvektor $x$. Die Laue-Beugung addiert dann einen reziproken Gittervektor des Kristalls auf diese Welle. Wenn das Potential also einen Koeffizienten $V_g$ besitzt, dann kann das Gitter einen Vektor $g$ addieren oder subtrahieren. Genau dies Koppelt die ebene Wellen bei $k$ und $k+g$. 

\section*{Verallgemeinerung}

Der Koeffizient $V_g$ koppelt in diesem Argument aber nicht nur ebene Wellen mit $k_1 = k$ und $k_2 =k+g$, sondern alle Paare von ebenen Wellen mit 
\begin{equation}
    | k_1 - k_2 | = g
\end{equation}
also beispielsweise $k+17g$ mit $k+16g$.

In Analogie damit ist es naheliegend, dass ein Fourier-Koeffizient des Potentials $V_{n g}$ gerade solche ebenen Wellen miteinander  koppelt, die $n g$v auseinander liegen, also 
\begin{equation}
    | k_1 - k_2 | = n g \quad . \label{eq:anhang_bandluecke_n}
\end{equation}

\begin{figure}
    \inputtikz{\currfiledir fig_zone_scheme_2}
   \caption{Dispersionsrelation freier Elektronen. Manche Kreuzungen sind mit dem $n$ aus 
   Gl.~B.2 bezeichnet. 
   }
\end{figure}

Das sieht man auch, wenn man  Gl.~\ref{eq:3_SG_rezi} für ein paar mehr als die drei Koeffizienten in Gl.~\ref{eq:3_SG_empty_lattice}  hinschreibt\sidenote{Tun Sie das!}. Das ist eine quadratische Matrix, die auf ihrer Diagonalen Einträge der Form
\begin{equation}
    E_{k + ng}^2 - E
\end{equation}
hat, wobei $n$ hier von $-N$ bis $+N$ läuft (und $N=1$ in  Gl.~\ref{eq:3_SG_empty_lattice}).
In der beiden Diagonalen darüber und darunter steht ein $V_g$, in den beiden Diagonalen nochmals darüber und darunter steht ein $V_{2g}$ und so weiter.\sidenote{Achtung: Gl.~\ref{eq:3_SG_empty_lattice} ist anders sortiert.} 





%-------------------




%-------
%
%%\nocite{*}

\printbibliography



\end{document}
