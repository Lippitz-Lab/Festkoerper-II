%\renewcommand{\lastmod}{\today}
\renewcommand{\chapterauthors}{Markus Lippitz}
\renewcommand{\lastmod}{3. Juni 2025}

\chapter{Halbleiter}

\label{chap:halbleiter}



\section{Ziele}
 


\begin{itemize}
\item Sie können die Ladungsträgerdichten in dotierten und undotierten Halbleitern im Zusammenspiel von Zustandsdichte und Fermi-Dirac-Verteilung erklären und so beispielsweise die unten gezeigten Messungen an Germanium interpretieren.
\item Sie können die Strom-Spannungs-Kennlinie eines p-n-Kontakts und die zugrundeliegende Bandstruktur erklären.
\end{itemize}



\begin{figure}
    \inputtikz{\currfiledir germanium_electron_density}
     \caption{
        Dichte der Elektronen im Leitungsband von Germanium, gemessen mit dem Hall-Effekt, als Funktion der Konzentration der Arsen-Donatoren. Daten aus \cite{Conwell1952}. Zum Vergleich ist $p_i(T) = n_i(T)$ (gestrichelt) und $p(T,n_D)$ (grau) eingezeichnet.
    \label{fig:5_Ge_n_density} 
     }
\end{figure}
 


% \begin{questions} 
% \item Wie groß ist ein Molekül?
% \item Welche physikalische Eigenschaft eine Moleküls wird bei Röntgenstreuung, STM und AFM abgebildet?
% \end{questions}
 


% Das Pluto-Skript hydrogen\_wave\_functions\pluto{hydrogen_wave_functions} ermöglicht es Ihnen, mit verschiedenen Varianten der grafischen Darstellung zu experimentieren.


\section{Überblick}


Wir haben bereits im Kapitel zur Bandstruktur gesehen, dass der Übergang zwischen Isolator und Halbleiter fließend ist. Beide haben ein voll besetztes Valenzband und ein unbesetztes Leitungsband mit der Bandlücke $E_g$. Thermische Anregung führt zu einer Besetzung des Leitungsbandes proportional zu $\exp(- E_g / 2 k_b T)$, wie wir weiter unten sehen werden. Beispielsweise werden bei Raumtemperatur je nach Bandlücke $10^{-5}$ (bei $E_g$ = 0.5 eV) oder $10^{-26}$ (bei $E_g$ = 3 eV) aller Ladungsträger angeregt sein. Dies führt zu einer von Null verschiedenen Leitfähigkeit. Als Halbleiter bezeichnet man ein Material, dessen spezifische Widerstand bei Raumtemperatur zwischen $10^{-2}$ und $10^9$ $\Omega$m liegt. Halbleiter können  nach verschiedenen Eigenschaften klassifiziert werden:


\paragraph*{intrinsisch oder dotiert} Wenn man ein Material gezielt mit einer Verunreinigung versieht, also dotiert, dann kann dies zu mehr freien Ladungsträgern führen, weil diese dann vom Fremdatom stammen. Bei intrinsischen Halbleitern stammen die Elektronen im Leitungsband immer aus dem Valenzband.

\paragraph*{direkt oder indirekt} In einem direkten Halbleiter befindet sich das Maximum des Valenzbandes an der gleichen Stelle im reziproken Raum wie das Minimum des Leitungsbandes. Dies ermöglicht optische  Übergänge ohne Änderung des Wellenvektors bei niedrigen Energien. Bei indirekten Halbleitern ist dies nicht der Fall. Optische Übergänge benötigen hier die Beteiligung eines Phonons.

\paragraph*{kristallin oder amorph} Eine perfekte periodische Kristallstruktur ist nicht unbedingt erforderlich ($\alpha$-Silizium ist für Solarzellen relevant), aber wir betrachten hier nur kristalline Materialien.

\paragraph*{organisch oder anorganisch} Wir betrachten hier anorganische Halbleiter, die aus einem oder zwei Elementen bestehen. Organische Halbleiter auf der Basis von Kohlenstoffverbindungen organischer Moleküle sind ein Thema für sich.


\section{Intrinsische kristalline Halbleiter}

Die Bandstruktur und die Lage der Fermi-Energie hängt sowohl in der 'empty lattice approximation' als auch im 'tight binding' Modell von der Elektronenkonfiguration der beteiligten Atome ab. Es ist daher nicht verwunderlich, dass Halbleiter aus im Periodensystem benachbarten Atomen gebildet werden. Dies sind zum einen die Elemente der Gruppe IV, also Kohlenstoff (\ch{C}), Silizium (\ch{Si}) und Germanium (\ch{Ge}). Die Bandlücken betragen 5.47~eV, 1.12~eV und 0.66~eV und sind jeweils indirekt. Diamant ist also ein Isolator.

Verbindungshalbleiter bestehen aus zwei Elementen, deren Elektronenzahl sich gerade auf die Gruppe IV mittelt. III-V-Halbleiter sind beispielsweise \ch{GaP} (2.26 eV, indirekt), \ch{GaAs} (1.43 eV, direkt), \ch{InSb} (0.18 eV, direkt). II-VI-Halbleiter sind beispielsweise \ch{ZnS} (sc: 3.54 eV) und \ch{CdSe} (1.74 eV).



\begin{questions}
    \item Warum ist die Bandlücke von Silizium größer als die von Germanium, obwohl beide Elemente in der gleichen Gruppe des Periodensystems stehen?
\end{questions}


\section{Optische Übergänge}

Als Beispiel für den Unterschied zwischen direkten und indirekten Halbleitern möchte ich hier dem späteren Kapitel über optische Eigenschaften vorgreifen und optische Übergänge diskutieren. Ein Photon wird absorbiert und transportiert ein Elektron vom Valenzband in das Leitungsband. Dies wird als Interbandübergang bezeichnet. Gleichzeitig kann ein Phonon absorbiert oder emittiert werden. Die Energie- und Impulserhaltung lautet dann
\begin{align}
    E_g = & \hbar \omega_\gamma(\mathbf{k}_\gamma ) \pm \hbar \Omega (\mathbf{q}) \\
    \hbar \Delta \mathbf{k} = &\hbar \mathbf{k}_\gamma \pm \hbar \mathbf{q} \quad .
\end{align}
Größen des Photons sind mit $\gamma$ gekennzeichnet. Das eventuell beteiligte Phonon hat die Frequenz $\Omega$ und den Impuls $\mathbf{q}$. Das Elektron ändert seine Energie um $E_g$ und seinen Impuls um $\Delta \mathbf{k}$.



Betrachten wir zunächst \emph{direkte Übergänge}, also \emph{Prozesse ohne Beteiligung eines Phonons}. Der Impuls eines Photons ist viel kleiner als relevante Impulse von Elektronen in der Brillouinzone:
\begin{equation}
    k_\gamma = \frac{2 \pi }{\lambda} \approx \frac{2 \pi }{500\text{ nm}}
    \ll k_{BZ} = \frac{\pi}{a} \approx \frac{\pi}{0.5 \text{ nm}} \quad .
\end{equation}
Photonen ermöglichen also nur senkrechte Übergänge in der Elektronen-Dispersionsrelation bei quasi konstantem Elektronen-Impuls $\mathbf{k}$ ($\hbar \Delta \mathbf{k} \approx 0$). Diese direkten Prozesse dominieren in direkten Halbleitern. Bei gegebenem $\mathbf{k}$ ist die Energiedifferenz zwischen Valenz- und Leitungsband in der Nähe der Extrema
\begin{equation}
    \Delta E = \hbar \omega_\gamma = E_c(\mathbf{k})-  E_v(\mathbf{k}) = 
    \left( \frac{1}{m_e^\star} + \frac{1}{m_h^\star} \right) \frac{\hbar^2 k^2}{2} =  \frac{\hbar^2 k^2}{2 m^\star_\text{komb}}
\end{equation}
mit der kombinierten effektiven Masse $m^\star_\text{komb}$ einer kombinierten Bandstruktur $ \Delta E(\mathbf{k})$. Für diese können wir auch eine kombinierte Zustandsdichte angeben
\begin{equation}
    D_\text{komb}( \Delta E) = \frac{V}{2 \pi^2} \left( \frac{2 m^\star_\text{komb}}{\hbar^2} \right)^{3/2} \sqrt{\Delta E - E_g} \quad .
\end{equation}
Diese Zustandsdichte bestimmt über  Fermis  Goldener Regel
\begin{equation}
            \Gamma_{if} = \frac{2 \pi}{\hbar} \, \braket{f | \hat{H}' | i} \, D( \Delta E)
\end{equation}
 die Übergangsrate $\Gamma_{if}$ und damit den Absorptionskoeffizienten $\alpha \propto \Gamma_{if}$. Dieser beschreibt die verbleibende Intensität $I$, nachdem ein Lichtstrahl gegebener Frequenz eine Probe der Dicke $L$ durchquert hat
\begin{equation}
    I(\omega_\gamma) = I_0 e^{- \alpha(\omega_\gamma) L} \quad .
\end{equation}
Insgesamt bekommen wir also mit
\begin{equation}
    \alpha(\omega_\gamma) \propto ( m^\star_\text{komb})^{3/2}  \sqrt{\hbar \omega_\gamma - E_g} \label{eq:5_absorption_direct_HL}
\end{equation}
einen wurzelförmigen Verlauf des Absorptionsspektrums, der bei der Energie der Bandlücke einsetzt. Dabei haben wir die Annahme gemacht, dass  das Übergangs-Dipolmoment $\braket{f | \hat{H}' | i} $  im relevanten Frequenzbereich konstant ist.

\begin{marginfigure}[-170mm]
    \inputtikz{\currfiledir InSb_absorption}
    \caption{Absorption von Indiumantimonid (\ch{InSb}), einem direkten Halbleiter, in der Nähe der Bandkante (\cite{Johnson1967}).}
\end{marginfigure}

\begin{marginfigure}[-50mm]
    \inputtikz{\currfiledir Si_absorption}
    \caption{Absorption von Silizium (\ch{Si}), einem indirekten Halbleiter, in der Nähe der Bandkante (\cite{Macfarlane1955}).}
\end{marginfigure}


Bei \emph{indirekten Übergängen} unter Beteiligung eines Phonons wird die Rechnung aufwändiger. Man muss über alle Ausgangs- und Endzustände der Elektronen integrieren, die über irgendein Phonon verbunden sind.  Außerdem kann die Reihenfolge der Wechselwirkung mit Photon und Phonon getauscht werden und beide Möglichkeiten interferieren in der Quantenmechanik.\sidenote{siehe zB \cite{yu_cardona}} Das Ergebnis der Rechnung ist 
\begin{equation}
    \alpha(\omega_\gamma) \propto \left( \hbar \omega_\gamma -( E_g  \pm \hbar \Omega) \right)^2 \quad \text{für} \quad 
    \hbar \omega_\gamma >  E_g  \pm \hbar \Omega \quad ,
\end{equation}
je nachdem, ob ein Phonon absorbiert oder emittiert wird.
Die Absorption setzt also mit einem quadratischen Verlauf bei der indirekten Bandlücke ein. Der Anstieg ist mit steigender Frequenz also viel flacher als bei einer direkten Bandlücke. Auch bei indirekten Halbleitern gibt es ab einer gewissen Energie direkte Übergänge. Diese benötigen keine Phononen und sind somit wahrscheinlicher. Ab der direkten Bandlücke dominiert dann also auch der wurzelförmige Verlauf.



\begin{questions}
    \item Warum ist die Absorption bei direkten Halbleitern bei der Bandkante so viel stärker als bei indirekten Halbleitern?
\end{questions}


\section{Thermische Besetzung der Bänder}

Am absoluten Nullpunkt der Temperatur ist das Valenzband vollständig gefüllt, das Leitungsband vollständig leer. Durch thermische Anregung gelangen aber Elektronen von Valenz- ins Leitungsband. Wir berechnen nun, wie viele das sind. Dabei behandeln wir Elektronen und Löcher immer parallel zu einander.

Die Dichte $n$ der Elektronen im Leitungsband bzw. die Dichte $p$ der Löcher im Valenzband können wir einfach durch Integration über die Zustandsdichte $D$ (im jeweiligen Band) und die Fermi-Dirac-Funktion $f_{FD}$ bestimmen
\begin{eqnarray}
    n &= & \int_{E_c}^\infty f_{FD}(E,T) \, D_c(E) \, dE \\
    p &= & \int_{-\infty}^{E_v} (1 - f_{FD}(E,T)) \, D_v(E) \, dE  \quad .
\end{eqnarray}
Die Energien $E_{v,c}$ bezeichnen die jeweiligen Bandkanten ($E_g = E_c - E_v$). Bei den Löchern integrieren wir über $1-f_{FD}$, weil diese ja gerade unbesetzte Elektronenzustände sind. Die Zustandsdichte ist jeweils
\begin{eqnarray}
    D_c(E) &= & \frac{ (2 m^\star_n)^{3/2} } {2 \pi^2 \hbar^3} \, \sqrt{E - E_c}   \quad \text{für} \quad E > E_c \\
    D_v(E) &= & \frac{ (2 m^\star_p)^{3/2} } {2 \pi^2 \hbar^3} \, \sqrt{E_v - E} \quad \text{für} \quad E < E_v \quad ,
\end{eqnarray}
wobei die Massen hier die jeweiligen effektiven Massen an der Bandkante sind.

Nun müssen wir noch die Fermi-Dirac-Funktion $f_{FD}$ geeignet nähern und dazu zunächst die Unterscheidung zwischen Fermi-Energie und Fermi-Niveau einführen. In der Fermi-Dirac-Funktion steht eigentlich das chemische Potential $\mu$. Die Fermi-Energie hatten wir als $E_F = \mu(T=0)$ definiert. Hier interessieren uns aber höhere Temperaturen. \cite{Gross_FK} verwendet direkt das chemische Potential $\mu(T)$, \cite{Hunklinger2014} führt den Begriff des Ferminiveaus als Äquivalent zum chemischen Potential ein. Um Verwechslungen mit der Beweglichkeit zu vermeiden, wird dort $E_F(T)$ statt $\mu$ verwendet. Ich folge hier Hunklinger.

Wir machen die Annahme, dass das Fermi-Niveau $E_F(T)$ weit\sidenote{auf einer $k_b T$-Skala} von den Bandkanten entfernt ist. Dies ist die \emph{Näherung der Nichtentartung} und gilt bei intrinsischen oder nur schwach dotierten Halbleitern. Sie gilt bei stark dotierten (entarteten)  Halbleitern nicht. Dazu mehr unten. Wenn diese Näherung also gemacht werden kann, also $|E - E_F| \gg k_b T$ für die relevanten Werte von $E$ ist, dann können wir die Fermi-Dirac-Verteilungen als Boltzmann-Verteilungen nähern:
\begin{eqnarray}
    f_{FD} = \frac{1}{e^{ (E- E_F) / k_B T} +1 } \approx e^{ -(E- E_F) / k_B T}  \quad \text{für} \quad E > E_F \\
  1-  f_{FD} = \frac{1}{e^{ (E_F - E) / k_B T} +1 } \approx e^{ -(E_F - E) / k_B T}  \quad \text{für} \quad E < E_F  \quad .
\end{eqnarray}

Damit können wir nun alles einsetzen und die Integrale berechnen:\sidenote{Man integriert über $x = (E-E_F)/k_bT$ und benutzt $\int \sqrt{y} e^{-y} = \sqrt{\pi}/2$.}
\begin{eqnarray}
    n &=& 2 \left( \frac{ k_b T \,  m^\star_n } {2 \pi \hbar^2} \right)^{3/2} \, e^{- (E_c - E_F) / k_b T} = \mathcal{N} \, e^{- (E_c - E_F) / k_b T} \\*  
    p &=& 2 \left( \frac{ k_b T \,  m^\star_p } {2 \pi \hbar^2} \right)^{3/2} \, e^{+ (E_v - E_F) / k_b T} = \mathcal{P} \, e^{+ (E_v - E_F) / k_b T}  \quad . \label{eq:5_konz_zugaenglich}
\end{eqnarray}
Wir haben die effektiven ('zugänglichen') Zustandsdichten $\mathcal{N}$ und $\mathcal{P}$ eingeführt, die nur schwach von der Temperatur abhängen. Wenn wir sie als konstant annehmen, haben wir die beiden Bänder zu zwei diskreten Zuständen vereinfacht, die bei den Energien $E_c$ bzw. $E_v$ liegen. Dies wird im Folgenden vieles vereinfachen.

\begin{figure}
    \inputtikz{\currfiledir thermal_occupation}
    \caption{Thermische Besetzung der Zustände eines intrinsischen Halbleiters mit 
    $m^\star_p = 1.5 m^\star_n$ und $E_g = k_b T$. 
    Die besetzten Zustände sind grau unterlegt. Man erkennt $n=p$. }
\end{figure}

\begin{questions}
    \item Wie beeinflusst das Verhältnis der effektiven Massen  $m^\star_p / m^\star_n$ die Lage des Fermi-Niveaus $E_F(T)$ im undotierten Halbleiter?
\end{questions}



\section{Massenwirkungsgesetz}

Interessant ist, dass sowohl $n$ als auch $p$ von der Lage des Fermi-Niveaus $E_F$ abhängt, deren Produkt aber nicht
\begin{equation}
    n \, p = \mathcal{N} \mathcal{P} \, e^{- E_g / k_b T} =   \mathcal{W} \, T^3 \, e^{- E_g / k_b T}  \quad . \label{eq_5_mwg}
\end{equation}
Das Produkt $n p$ hängt nur von Materialparametern und der Temperatur ab. Die Beziehung  $n p = $~const. wird in Analogie zu einer chemischen Reaktion \emph{Massenwirkungsgesetz} genannt. Bei einer  Reaktion der Form
\begin{equation}
    \ch{A} +  \ch{B} \rightleftharpoons  \ch{C} +  \ch{D} 
\end{equation}
ergibt sich in stark verdünnter Lösung eine Gleichgewichtskonstante $K$ mit
\begin{equation}
    K = \frac{ [\ch{C}] \, [\ch{D}] }{ [\ch{A}] \, [\ch{B}]} \quad ,
\end{equation}
wobei eckige Klammern hier die Konzentration der jeweiligen Stoffe bezeichnen. Man kann die Existenz von Elektronen im Leitungsband und Löchern im Valenzband als chemische Reaktion der Art
\begin{equation}
    \ch{<nichts>} \rightleftharpoons  \ch{Elektronen} +  \ch{Löcher} 
\end{equation}
sehen. Durch thermische Anregung entstehen Elektronen (im Leitungsband) und Löcher (im Valenzband) aus dem Nichts. Die rechte Seite von Gl. \ref{eq_5_mwg} entspricht also der Gleichgewichtskonstante $K$.

Man beachte, dass zwar  $n p = $~const., aber bislang noch nicht $n=p$ gefordert wurde. Das Massenwirkungsgesetz  $n p = $~const gilt also auch für dotierte Halbleiter.


\begin{questions}
    \item Warum ist die Beziehung $n p = $~const. unabhängig von der Lage des Fermi-Niveaus $E_F$?
\end{questions}


\section{Intrinsische Ladungsträgerdichte}

Bei dotierten Halbleitern (s.u.) kann ein Elektron nicht nur aus dem Valenzband, sondern auch von einem Fremdatom stammen. 
Für intrinsische, d.h. undotierte Halbleiter gilt jedoch, dass für jedes Elektron im Leitungsband ein Loch im Valenzband entstehen muss, d.h. $n=p$. Die Dichte der so paarweise erzeugten Ladungsträger wird als intrinsische Ladungsträgerdichte $n_i = p_i$ bezeichnet. Sie ergibt sich aus dem Massenwirkungsgesetz
\begin{equation}
    n_i = p_i = \sqrt{n p} = \sqrt{\mathcal{N}\mathcal{P}} \, e^{- E_g / 2 k_b T} \quad .
\end{equation}
Diese Gleichung hatten wir in der Einführung schon verwendet, um Ladungsträgerdichten abzuschätzen. Man beachte hier die Zwei im Exponenten. Die Bandlücke erscheint nur halb so groß, weil wir gleichzeitig ein Elektron und ein Loch aus dem Nichts erzeugen.



\section{Temperaturabhängigkeit des Fermi-Niveaus}

Wir können uns einen Zusammenhang für die Temperaturabhängigkeit des Fermi-Niveaus   undotierter Halbleiter beschaffen, in dem wir $n_i = p_i$ umformen zu
\begin{equation}
    \frac{\mathcal{P}}{\mathcal{N}} = \left( \frac{m^\star_n}{m^\star_p} \right)^{3/2} = e^{(2 E_F - E_c - E_v)/(k_b T)}
\end{equation}
und dann nach $E_F$ auflösen
\begin{equation}
    E_F(T) = \frac{E_c + E_v}{2} + \frac{3}{4} k_B T \ln \left( \frac{m^\star_p}{m^\star_n} \right) \quad . \label{eq:5_Efermi_intr}
\end{equation}
Bei $T=0$ liegt das Fermi-Niveau also in der Mitte der Bandlücke. Falls Leitungs- und Valenzband gleich stark gekrümmt sind, also Elektronen- und Loch-Masse identisch, dann bleibt es dort auch bei steigender Temperatur. Unterscheiden sich die Massen, dann bewegt sich das Fermi-Niveau leicht in Richtung der leichteren Ladungsträger. Da $k_B T \ll E_g$ ist dies aber ein kleiner Effekt.

\begin{questions}
    \item Warum ist die Temperaturabhängigkeit des Fermi-Niveaus $E_F(T)$ bei Halbleitern mit stark unterschiedlichen effektiven Massen $m^\star_n$ und $m^\star_p$ größer als bei Halbleitern mit ähnlichen Massen?
\end{questions}


\section{Dotierung}

Kein Halbleiterkristall ist wirklich rein, sondern enthält immer Fremdatome. Diese Fremdatome verursachen zusätzliche Ladungsträger, Elektronen oder Löcher und beeinflussen damit die Leitfähigkeit. In einem idealen \ch{GaAs} beträgt die intrinsische Ladungsträgerdichte bei Raumtemperatur beispielsweise $n_i \approx 10^7$~cm$^{-3}$. Aber selbst in den besten \ch{GaAs}-Kristallen findet man $n \approx 10^{16}$~cm$^{-3}$. Daher kann man einen Kristall gezielt dotieren (eigentlich 'verunreinigen'), um die Ladungsträgerdichte einzustellen. Man unterscheidet zwischen n- und p-Dotierung, da man ein Atom der Valenz $V$ im Ausgangsmaterial durch ein Fremdatom der Valenz $V+1$ oder $V-1$ ersetzen kann. Im Fall $V+1$ spricht man von Donatoren, also Atomen, die Elektronen abgeben, oder von n-Dotierung. Im Fall von $V-1$ spricht man von Akzeptoren (Atome, die Elektronen aufnehmen) oder p-Dotierung.

Ein Donator-Atom ist für 'sein' Elektron ähnlich einem Proton für das Elektron im Wasserstoff-Atom, und analog ein Akzeptor für das Loch. Wir können also die Beschreibung des Wasserstoff-Atoms benutzen und müssen nur die effektive Masse des Ladungsträgers und die relative Permittivität $\epsilon_r$ des Materials anpassen. Für einen Donator sind die Energie-Eigenwerte also 
\begin{equation}
    E_n = - \frac{1}{2} \, \frac{m_e^\star e^4}{(4 \pi \epsilon_r \epsilon_0)^2 \hbar^2}
\frac{1}{n^2} =     
   - \frac{m_e^\star}{m_\text{frei}} \, \frac{1}{\epsilon_r^2} \, \frac{13.6 \text{ eV}}{n^2}
\end{equation}
mit $n$ hier der Quantenzahl und nicht der Ladungsträgerdichte. Der Bohr-Radius ist
\begin{equation}
    r_\text{Bohr} = \frac{\epsilon_r}{m_e^\star / m_\text{frei}} \, 0.53 \text{ \AA} \quad .
\end{equation}
Für Silizium findet man mit $m_e^\star = 0.1 m_\text{frei}$ und $\epsilon_r = 11.7$ die Werte $E_1 = -10$~meV und $ r_\text{Bohr} = 62$~\AA. Der Betrag der Bindungsenergie ist also viel kleiner als die Bandlücke (1.14~eV) und kleiner als $k_b T$ (25~meV) bei Raumtemperatur. Der Bohr-Radius ist viel größer als die Gitterkonstante (5.4~\AA). Es liegen also sehr viele Silizium-Atome innerhalb der Bahn des Elektrons um das eine Donator-Fremdatom.

Welche Bedeutung hat die Bindungsenergie des Donator-Elektrons für die Bandstruktur? Die Ionisation des Wasserstoffatoms entspricht dem Übergang des gebundenen Donator-Elektrons in das Leitungsband. Dort kann es sich frei bewegen. Die Donatorzustände liegen daher im Abstand $E_n$ unterhalb der Unterkante des Leitungsbandes. Analog dazu liegen die Akzeptorzustände etwas oberhalb der Oberkante des Valenzbandes. In unserem Wasserstoffmodell sind die Bindungsenergien (Abstände der Zustände von der Bandkante) unabhängig vom Fremdatom, es gehen nur die Eigenschaften des Ausgangsmaterials ein. Dies ist in der Realität auch fast so, da der Bohrradius so groß ist, dass Details der Elektronenkonfiguration des Fremdatoms nur sehr schwach eingehen.

Bei Raumtemperatur sind die Störstellen mit großer Wahrscheinlichkeit ionisiert. Bei tiefen Temperaturen kann man durch Infrarotabsorption die Lage der Donator- und Akzeptor-Niveaus bestimmen.

\begin{questions}
    \item Warum ist die Bindungsenergie der Donatoren und Akzeptoren in Halbleitern so viel kleiner als die Bandlücke?
    \item Warum ist sie unabhängig von der Art des Fremdatoms? Und warum ist sie unabhängig von der Temperatur?
\end{questions}


\section{Temperaturabhängigkeit der Ladungsträgerdichte}

Durch die Dotierung lässt sich die Ladungsträgerdichte einstellen, die dann aber natürlich stark temperaturabhängig ist, weil die Bindungsenergie der Störstellen relativ klein ist. Dies werden wir nun etwas genauer betrachten. Wir machen weiterhin die Näherung der Nichtentartung, also dass das Fermi-Niveau\sidenote{=chemisches Potential $\mu$} weit genug von der Bandkante entfernt ist, so dass wir die Fermi-Dirac-Verteilung mit einer Boltzmann-Verteilung nähern können. Die Ladungsträgerdichten im Leitungsband ($n$) und Valenzband ($p$) sind also wie oben
\begin{equation}
    n  = \mathcal{N} \, e^{- (E_c - E_F) / k_b T} \quad \text{und} \quad   
    p =  \mathcal{P} \, e^{+ (E_v - E_F) / k_b T} \quad , \label{eq:5_np_eff}
\end{equation}
unabhängig davon, ob die Ladungsträger aus dem gegenüberliegen Band oder aus Störstellen stammen. Die intrinsischen Dichten hatten wir erst später eingeführt.

Jedes Donator- und Akzeptor-Niveau kann nur mit einem Ladungsträger besetzt werden. Ein unterschiedlicher Spin reicht nicht aus, weil die Coulomb-Abstoßung der Ladungsträger größer ist also die Bindungsenergie.\sidenote{Man darf nicht die Lage der Zustände im \ch{H}-Atom vergleichen, sondern die zwischen \ch{H} und \ch{H-} oder \ch{He} und \ch{He+}.} Sei $n_D$ also die Dichte der Donatoren, die entweder besetzt (neutral) sein können ($n_D^0$) oder ionisiert  ($n_D^+$) und damit natürlich
\begin{equation}
    n_D = n_D^0 + n_D^+ \quad \text{und} \quad   n_A = n_A^0 + n_A^-   \quad .
\end{equation}
Für deren Temperaturabhängigkeit kann man die Fermi-Dirac-Verteilung nicht mehr durch eine Boltzmann-Verteilung nähern, sondern muss schreiben
\begin{equation}
    \frac{n_D^0}{n_D} = \frac{1}{e^{(E_D - E_F)/ k_b T } +1}
    \quad \text{und} \quad 
    \frac{n_A^0}{n_A} = \frac{1}{e^{(E_F - E_A)/ k_b T } +1} \quad .
\end{equation}

Alle Ladungen im Halbleiter müssen sich jederzeit neutralisieren, also 
\begin{equation}
    n + n_A^- = p + n_D^+ \quad .
\end{equation}

Diese vier Gleichungen beschreiben zusammen die Temperaturabhängigkeit. Wir nehmen nun  einen n-Halbleiter an, also dass viel mehr Donatoren als Akzeptoren vorhanden sind, also $n_D \gg n_A$. In diesem Fall findet sich sicherlich für jeden Akzeptor ein Elektron aus einem Donator, so dass alle Akzeptoren negativ geladen sind ($n_A \approx n_A^-$).
Und wir nehmen an, dass die Dotierung so stark ist, dass sie die Leitfähigkeit dominiert (\emph{Störstellenleitung}). Dies ist dann der Fall, wenn $n_D^+ \gg p$, also viel mehr Donatoren ionisiert sind als Löcher vorhanden, oder die Mehrzahl der Elektronen im Leitungsband von Donatoren stammt, also die intrinsische Dichte vernachlässigt werden kann ($n_i \ll n_D^+$). Damit wird die Dichte $n$ der Elektronen im Leitungsband
\begin{eqnarray}
    n  &= & p + n_D^+ -  n_A^- \approx n_D^+ -  n_A = (n_D - n_D^0) - n_A \\
    & =& n_D \left( 1- \frac{1}{e^{(E_D - E_F)/ k_b T } +1} \right) - n_A \quad .
\end{eqnarray}
Das Fermi-Niveau $E_F$ können wir durch Umformen mit Gl.~\ref{eq:5_np_eff} entfernen
\begin{equation}
    \frac{n (n_A + n)}{n_D - n_A - n} = \mathcal{N} \, e^{- E_d / k_b T} \quad , \label{eq:5_n_of_T}
\end{equation}
wobei $E_d = E_c - E_D$ der energetische Abstand der Donator-Niveaus von der Unterkante der Leitungsbandes ist. Aus dieser Gleichung kann man bei gegebener Fremdatom-Konzentrationen  $n_A$, $n_D$ und Energien die Ladungsträgerdichte $n$ berechnen. Zusammen mit  
Gl.~\ref{eq:5_np_eff} erhält man dann auch die Lage der Fermi-Niveaus $E_F$. Dies ist in Abb.~\ref{fig:5_dpoing_temp} als Funktion der reziproken Temperatur $T$ gezeigt und in der Abb.~\ref{fig:5_Ge_n_density} am Anfang des Kapitels linear. Man findet vier Temperaturbereiche, die im Folgenden mit steigender Temperatur diskutiert werden.

\begin{figure} 
    \inputtikz{\currfiledir doping_temp}
    \caption{Temperaturabhängigkeit der Ladungsträgerdichte $n$. In dieser Darstellung ist $E_d = E_g / 10$ gewählt. \label{fig:5_dpoing_temp}}
\end{figure}

\paragraph*{Kompensationsbereich} Bei sehr tiefen Temperaturen ist $k_b T \lll E_d$ und nur sehr wenige Ladungsträger sind im Leitungsband, also $n \ll n_A \ll n_D$ und Gl. \ref{eq:5_n_of_T} wird
\begin{equation}
    n  \approx \frac{n_D \, \mathcal{N}}{n_A}  \, e^{- E_d / k_b T}  \quad .
\end{equation}
Das Fermi-Niveau wird mit Gl.~\ref{eq:5_np_eff}
\begin{equation}
    E_F \approx E_c - E_d + k_b T \ln \left( \frac{n_D}{n_A} \right) \quad .
\end{equation}
Wenn es nur kalt genug ist, liegt das Fermi-Niveau auf den Donator-Zuständen. Die Donatoren sind teilweise geladen, aber nicht so sehr, weil deren Ladungsträger im Leitungsband sind, sondern weil sie die Akzeptoren besetzt haben und diese keine Löcher mehr liefern können. Die Akzeptoren werden also kompensiert. Wenn $T$ größer wird, dann verschiebt sich $E_F$ in Richtung Bandkante und mehr Ladungsträger gelangen ins Leitungsband.

\paragraph*{Störstellenreserve} Mit steigender Temperatur ist dann der Punkt erreicht, dass $n_A \ll n \ll n_D$. Damit wird   Gl. \ref{eq:5_n_of_T} 
\begin{equation}
    n \approx \sqrt{n_D \, \mathcal{N}} \,  \, e^{- E_d / 2 k_b T} 
\end{equation}
(man beachte die 2 im Exponenten) und das Fermi-Niveau wird 
\begin{equation}
    E_F \approx E_c - \frac{E_d}{2} - \frac{k_b T}{2} \ln \left( \frac{\mathcal{N}}{n_D} \right)  \quad .
\end{equation}
Das Fermi-Niveau ist als in etwa in der Mitte zwischen Leitungsband-Unterkante und Donator-Niveau. Es sind noch nicht alle Donatoren ionisiert. Die Donatoren übernehmen die Rolle des Valenzbandes bei intrinsischen Halbeleitern und $E_d$ die Rolle der Bandlücke.

\paragraph*{Störstellenerschöpfung} Jetzt erreichen wir Raumtemperatur und $k_b T \approx E_d$. Damit wird Gl. \ref{eq:5_n_of_T} 
\begin{equation}
    \frac{n^2}{n_D - n} \approx \mathcal{N}  \quad \text{bzw.} \quad n \approx n_D = \text{const.}
\end{equation}
weil $n \ll  \mathcal{N}$ und 
\begin{equation}
    E_F \approx E_c - k_b T \ln \left( \frac{\mathcal{N}}{n_D} \right) \quad .  \label{eq:5_ef_stoerstellenerschoepfung}
\end{equation}
Alle Störstellen sind ionisiert, aber die direkte Anregung von Elektronen aus dem Valenz- ins Leitungsband spielt noch keine Rolle.

\paragraph*{Eigenleitung} Die Temperatur ist schließlich so hoch, dass $k_b T \gg E_d$. Ladungsträger werden vom Valenz- ins Leitungsband angeregt und die Annahme $p \ll n_D$ gilt nicht mehr. Die Dotierung spielt keine Rolle mehr und der Halbleiter verhält sich wie ein intrinsischer Halbleiter mit den Gleichungen  \ref{eq:5_konz_zugaenglich} und  \ref{eq:5_Efermi_intr} für $n$ und $E_F$.


\begin{questions}
    \item Warum ist die Beziehung $n p = $~const. auch für dotierte Halbleiter gültig? 
\end{questions}

    


\section{p-n-Übergang}

\begin{marginfigure}
    \inputtikz{\currfiledir pn_sketch}
    \caption{Räumliche Verteilung der festen und beweglichen Ladungen an einem p-n-Übergang und das sich daraus ergebende Potential.}
\end{marginfigure}

Die Dotierung eines Halbleiter-Kristalls kann räumlich variieren. Dazu kann man beispielsweise durch UV-Lithographie eine Maske erzeugen, die das Eindringen von Fremdatomen an manchen Stellen verhindert. Wir betrachten hier einen Kristall, bei dem sich auf einer Längenskala von wenigen Nanometern die Dotierung von p nach n ändert.  Seien die Konzentration der Störstellen beispielsweise
\begin{equation}
    n_A(x) = n_A \Theta(x) \quad \text{und}  \quad  n_D(x) = n_D \Theta(-x) 
\end{equation}
mit der Stufenfunktion $\Theta$. Der dabei entstehende p-n-Übergang besitzt einen charakteristischen Verlauf der Bandkanten und dient technologisch beispielsweise als Diode.




\begin{marginfigure}[-10mm]
  \inputtikz{\currfiledir pn_level}
    \caption{Bandstruktur und Lage des Fermi-Niveaus vor und nach dem Verbinden  von zwei unterschiedlich dotierten Halbleitern.}
\end{marginfigure}

Stellen wir uns zunächst vor, dass die beiden Bereiche nicht miteinander in Verbindung stünden. Im p-dotierten Bereich gibt es bei Raumtemperatur viel mehr  Löcher im Valenzband  als Elektronen im Leitungsband, im n-dotierten Bereich ist es gerade anders herum. Wenn man die beiden Bereiche miteinander verbindet, dann diffundieren die Ladungsträger jeweils in den anderen Bereich, um die Konzentration auszugleichen. Dort rekombinieren aber Elektronen und Löcher und vernichten sich so gegenseitig. In den Ausgangsbereichen bleiben die gegenteilig geladenen immobilen Störstellen zurück. Dadurch bildet sich im p-dotierten Bereich eine negative Raumladung, im n-dotierten eine positive. Diese wirkt der Diffusion entgegen und ein Gleichgewicht stellt sich ein. Die dabei entstehende Potentialdifferenz nennt man \emph{Diffusionsspannung} $V_D$. Diese verschiebt die Gesamtenergie der Elektronen um den Beitrag  $-e \, V_D$ und verbiegt so die Bänder, im n-dotierten Bereich hin zu niedrigeren Energien. Die Raumladungszone nennt man auch Verarmungszone, weil dort fast keine mobilen Ladungsträger mehr vorhanden sind.\sidenote{Das Massenwirkungsgesetz fordert $n \cdot p =$const, aber $n+p$ kann stark variieren. } 

Man kommt zum gleichen Ergebnis auch mit einer anderen Argumentation. Vor dem Zusammenführen hat man im p-dotierten Bereich bei Raumtemperatur ein Fermi-Niveau etwas oberhalb des Akzeptor-Niveaus, weil wir im Bereich der Störstellenerschöpfung sind. Im n-dotierten Teil ist die Bandlücke identisch, weil es dasselbe Halbleiter-Material ist. Nur liegt dort das Fermi-Niveau  etwas unterhalb der Donator-Niveaus. Im thermischen Gleichgewicht kann es aber nur ein Fermi-Niveau (= chemisches Potential) geben.\sidenote{In der Chemie pro Stoff, hier aber alles Elektronen.} Es bildet sich also ein Makropotential $\phi(x)$, das zur potentiellen Energie $q \phi(x)$ der Ladungsträger ($q = \pm e$) beiträgt, so dass das Fermi-Niveau wieder räumlich konstant ist.\sidenote{Wenn man eine externe Spannung anlegt, so wird dieses externe Potential typischerweise als separat vom Fermi-Niveau angesehen. Dann ist das Fermi-Niveau nicht mehr räumlich konsant. Die Trennung zwischen Makropotential und externem Potential ist aber nur Konvention.} 


Die Diffusionsspannung $V_D$ ergibt sich also aus der Differenz der Fermi-Niveaus im unverbundenen Zustand. In erster Näherung ist also $e V_D \approx E_g$. Im Bereich der Störstellenerschöpfung ist sie durch Gl.~\ref{eq:5_ef_stoerstellenerschoepfung} gegeben
\begin{eqnarray}
    e \, V_D &= & E_F^n - E_F^p \\
    & =&   E_c - k_b T \ln \left( \frac{\mathcal{N}}{n_D} \right) - 
    E_v - k_b T \ln \left( \frac{\mathcal{P}}{n_A} \right) \\
   & =& E_g -  k_b T \ln \left( \frac{\mathcal{N \, P}}{n_D \, n_A} \right) \\
   & = & k_b T \ln \left( \frac{n_D \, n_A}{n_i^2} \right)  \quad .
\end{eqnarray}

Die Ladungsträgerdichten sind weiterhin durch die Lage des Fermi-Niveaus (Gl.~\ref{eq:5_konz_zugaenglich}) gegeben, nur dass sich dieses nun aus einem konstanten Anteil $E_F$ und einem räumlich variablen Anteil $e \phi(x)$ zusammensetzt. Wir erhalten also 
\begin{equation}
    n(x)  = \mathcal{N} \, e^{- (E_c - E_F - e\phi(x)) / k_b T} \quad \text{und} \quad   
    p(x) =  \mathcal{P} \, e^{+ (E_v - E_F  - e\phi(x)) / k_b T} \quad . \label{eq:5_n_of_x}
\end{equation}
Die  Raumladungsdichte $\rho(x)$ ist damit
\begin{equation}
    \rho(x) = e \left( n_D(x) - n_A(x) - n(x) + p(x) \right) \quad ,
\end{equation}
wenn man annimmt, dass alle Störstellen ionisiert sind. Zusammen mit der Poisson-Gleichung
\begin{equation}
    - \nabla^2 \phi(x) = \frac{1}{\epsilon_r \epsilon_0} \, \rho (x)
\end{equation}
haben wir ein nichtlineares gekoppeltes  System von Differentialgleichungen für $\rho(x)$ und $\phi(x)$, das sich nur numerisch lösen lässt.




\section*{Schottky-Modell der Raumladungszone}


Im Schottky-Modell nähert man den graduellen Verlauf der Ladungsträgerdichten durch Rechteck-Funktionen.\sidenote{Das geht, weil die beweglichen Ladungsträger in ihrer Dichte exponentiell variieren, also sehr schnell irrelevant werden, siehe z.B. \cite{AshcroftMermin2013}.} In einem Intervall der Breite $d_p$ befinden sich nicht durch freie Ladungsträger kompensierte Akzeptoren der Konzentration $n_A$ und andersherum für den n-dotierten Bereich. Wenn das Koordinatensystem so ist, dass $x=0$ an der p-n-Grenzfläche, dann ist beispielsweise $\rho(x) = - e n_A$ im Bereich $- d_p < x < 0$ und  $\rho(x) =0$ für $x < -d_p$. Durch Integration der Poisson-Gleichung bekommt man dann einen parabelförmigen Verlauf 
\begin{equation}
    \phi(x) = \phi_{-\infty} + \frac{e n_A}{\epsilon_r \epsilon_0} \left( d_p + x \right)^2  \quad \text{für} \quad - d_p < x < 0 \quad ,
\end{equation}
wobei $\phi_{-\infty}$ das konstante Makropotential tief im p-dotierten Bereich ist und $V_D =  \phi_{+\infty} - \phi_{-\infty}$. Das Potential erfüllt so schon die Stetigkeitsbedingungen im Übergang zu $\phi_{\pm\infty}$. Bei $x=0$ muss ebenfalls die erste Ableitung  stetig sein, also
\begin{equation}
    n_D \, d_n = n_A \, d_p
\end{equation}
was der Forderung nach Gesamt-Neutralität entspricht. Aus der Stetigkeit von $\phi$ selbst ergibt sich eine weitere Bedingung
\begin{equation}
    \frac{e}{2 \epsilon_r \epsilon_0} \left(  n_D \, d_n^2  + n_A \,d_p^2 \right) =\phi_{+\infty} - \phi_{-\infty} = V_D \quad .
\end{equation}
Damit erhalten wir
\begin{equation}
    d_n = \sqrt{ \frac{{2 \epsilon_r \epsilon_0 \, V_D}}{e}   \, \, \frac{n_A / n_D}{n_A + n_D}}
\end{equation}
und $d_p$ analog. typischerweise ist $e V_D \approx E_g \approx 1 $~eV, und $n_{A,D} \approx 10^{14} \dots 10^{18}$~cm$^{-3}$. Also liegen $d_{p,n}$ bei 10 bis 1000~nm.


\begin{figure}
    \inputtikz{\currfiledir Schottky_modell}
    \caption{Schottky-Modell der Raumladungszone}
\end{figure}

\begin{questions}
    \item Warum ist die Breite der Raumladungszone $d$ unabhängig von der Temperatur?
\end{questions}

\section*{Externe Spannung und Strom-Spannungs-Kennlinie}

Nun legen wir eine externe Spannung an den p-n-Übergang an. Wir nennen die Spannung $U$ positiv, wenn sie das Potential der p-dotierten Seite anhebt. Nur innerhalb der  Verarmungszone ist die Leitfähigkeit relativ niedrig, so dass die Spannung im Wesentlichen hier abfällt. Die Bänder ändern sich also nur im Bereich der Verarmungszone, außerhalb bleibt alles unverändert. Damit bleiben auch die Gleichungen aus dem letzten Abschnitt gültig, wenn wir jeweils $V_D$ durch $V_D - U$ ersetzen.

Durch Anlegen der externen Spannung ändert sich die Breite der Raumladungszone
\begin{equation}
    d = d_p + d_n = d(U=0) \, \sqrt{1 - \frac{U}{V_D} }\quad .
\end{equation} 
Positive Spannungen (in Durchlassrichtung) reduzieren die Breite der Raumladungszone (=Verarmungszone), negative Spannung (Sperrrichtung) vergrößern sie. 

Um die Wirkungsweise eines p-n-Übergangs als Diode zu verstehen sind die beteiligten Ströme\sidenote{ 
    \cite{Gross_FK} unterscheidet zwei Paare von Strömen: diff \& drift sowie gen \& rec. Die Vorzeichen sind aber anders definiert: diff + drift = 0 aber gen = rec !} relevant.

\paragraph*{Rekombinationsstrom  $j^\text{rec}$} Aufgrund des Konzentrationsunterschieds links und rechts der Grenzfläche diffundieren beispielsweise Elektronen aus dem n-dotierten Bereich  in den p-dotierten und rekombinieren mit den dort in großer Zahl vorhandenen Löchern. Dieser Strom wird auch \emph{Diffusionsstrom} genannt.

\paragraph*{Generationsstrom $j^\text{gen}$} Die Raumladungszone bildet einen Kondensator, in dem Ladungsträger beschleunigt werden. Thermisch erzeugte Elektronen im p-dotierten Bereich driften durch das Raumladungs-Feld in Richtung n-dotierten Bereich. Dieser Strom wird auch \emph{Driftstrom} oder \emph{Feldstrom} genannt.

Im thermischen Gleichgewicht sind beide Ströme gleich groß und kompensieren sich gerade $j_n^\text{rec} = j_n^\text{gen}$. Und natürlich kann man genau so mit Löchern argumentieren, so dass es beide Ströme auch mit dem Index $p$ gibt. Die Gesamt-Ströme sind die Summe der beiden Ladungsträger.

Der Diffusionsstrom fließt entgegen der Raumladungs-Potentialschwelle. Die Wahrscheinlichkeit, dass dies gelingt, enthält einen Boltzmann-Faktor
\begin{equation}
    j_n^\text{rec}(U) = a(T) \, e^{-e (V_D - U) / k_b T} =  j_n^\text{rec}(0) \, e^{e U / k_b T}
    = j_n^\text{gen} \, e^{e U / k_b T}
    \quad .
\end{equation}
Bei angelegter Spannung besteht kein thermisches Gleichgewicht mehr, so dass $j_n^\text{rec} = j_n^\text{gen}$ nicht mehr gilt, sondern
\begin{equation}
    j_n(U) = j_n^\text{rec}(U) - j_n^\text{gen} = j_n^\text{gen} \left(  e^{e U / k_b T} - 1 \right) \quad ,
\end{equation}
beziehungsweise für beide Ladungsträger zusammen 
\begin{equation}
    j(U) =  j_s \left(  e^{e U / k_b T} - 1 \right)
\end{equation}
mit dem Sättigungsstrom $j_s = j_n^\text{gen}  + j_p^\text{gen} $.

\begin{questions}
    \item Warum ist der Diffusionsstrom exponentiell von der Spannung $U$ abhängig?
\end{questions}


\section{Zusammenfassung}

\textit{Schreiben Sie hier ihre persönliche Zusammenfassung des Kapitels auf. Konzentrieren Sie sich auf die wichtigsten Aspekte und die am Anfang genannten Ziele des Kapitels.}

\vspace*{10cm}

\printbibliography[segment=\therefsegment,heading=subbibliography]
