%\renewcommand{\lastmod}{\today}
\renewcommand{\chapterauthors}{Markus Lippitz}
\renewcommand{\lastmod}{25. April 2023}

\chapter{Fermi-Gas}




\section{Ziele}
 


\begin{itemize}  
\item Sie können den elektrischen Widerstand von Metallen erklären.
\item Sie können den mikroskopischen  Ursprung des Wiedemann-Franz-Gesetzes erklären, also warum die Temperaturabhängigkeit der elektrischen und  der thermischen Leitfähigkeit in Metallen so ähnlich bis identisch ist.

\end{itemize}

\begin{figure}
    \inputtikz{\currfiledir Cu_Lorenz}
    \caption{Temperaturabhängigkeit der thermischen ($K$) und elektrischen ($\sigma$) Leitfähigkeit von Kupfer und die daraus abgeleitet Lorenz-Zahl $L = K/ \sigma T$. Das Wiedemann-Franz-Gesetz besagt, dass diese konstant ist. 
        Daten aus \cite{Hust1984}. \label{fig:2_Cu_Lorenz}}
\end{figure} 

\section{Überblick}

Mit diesem Kapitel beginnen wir mit den elektronischen Eigenschaften der Festkörper. Bislang hatten wir diese dadurch ignoriert, dass die Beispiele immer als Isolatoren gewählt waren, Elektronen also keine Rolle gespielt haben. Von nun an stehen die Elektronen im Mittelpunkt. Wie in der Molekülphysik auch machen wir die adiabatische Näherung. Wir nehmen also an, dass die Elektronen viel schneller sind als die Kerne, die Kerne aber das Potential vorgeben, in dem sich die Elektronen bewegen. Weiterhin machen wir die Ein-Elektron-Näherung. Wir betrachten also nur ein Elektron. Die Anwesenheit aller anderen Elektronen beeinflusst nur das Potential, auch über das Pauli-Prinzip. Korrelationen zwischen Elektronen berücksichtigen wir  erst in den Kapiteln zur Supraleitung und zum Magnetismus.

In diesem Kapitel bildet der Kristall ein großes Kastenpotential, aber die Kerne selbst kommen nicht vor. Im folgenden Kapitel wird dann das Kristallgitter wichtig werden.




% \begin{questions} 
% \item Wie groß ist ein Molekül?
% \item Welche physikalische Eigenschaft eine Moleküls wird bei Röntgenstreuung, STM und AFM abgebildet?
% \end{questions}
 
% Das Pluto-Skript hydrogen\_wave\_functions\pluto{hydrogen_wave_functions} ermöglicht es Ihnen, mit verschiedenen Varianten der grafischen Darstellung zu experimentieren.


\section{Freies Elektronengas}

In beispielsweise Alkali-Metallen sind die meisten Elektronen an 'ihren' Atomkern gebunden und nur sehr wenige Elektronen pro Atomkern tragen zur Bindung bei. Diese Elektronen sehen nicht das vollständige Coulomb-Potential der stark positiv geladenen Atomrümpfe. Die gebundenen Elektronen schirmen dies ab, so dass nur ein schwaches und räumlich eher konstantes Potential verbleit. In diesem Potential bewegen sich die Valenz-Elektronen der Alkali-Atome wie ein Gas. Man spricht daher von freien Elektronengas oder auch Fermi-Gas.

Wir könnten die freien Elektronen als Teilchen im 3D-Kasten modellieren. Die Schrödingergleichung innerhalb des Kastens beinhaltet dann nur noch die kinetische Energie
\begin{equation}
    - \frac{\hbar^2}{2m} \nabla  \psi(\mathbf{r}) = E  \psi(\mathbf{r})
\end{equation}
und ihre Lösung sind ebene Wellen 
\begin{equation}
    \psi(\mathbf{r}) = \frac{1}{\sqrt{V}} e^{i \mathbf{k} \cdot \mathbf{r}}
\end{equation}
mit dem Wellenvektor $\mathbf{k}$ und der Normierung auf das Volumen $V$ des Kastens. Die Energie beträgt dann
\begin{equation}
    E = \frac{\hbar^2 }{2m} |\mathbf{k}|^2 = \frac{|\mathbf{p}|^2 }{2m} 
\end{equation}
und ist von der Richtung natürlich unabhängig.

Die Gleichungen werden angenehmen, wenn wir (wie bei den Phononen) periodische Randbedingungen einführen: alle Eigenschaften sollen im Ort periodisch mit der Kasten-Größe $L$ sein, also 
\begin{equation}
    \psi(\mathbf{r}) =  \psi(\mathbf{r} + L \mathbf{\hat{e}_i}) \quad ,
\end{equation}
mit $\mathbf{\hat{e}_i}$ einem kartesischen Einheitsvektor.
Wie bei den Phononen sind die möglichen Werte des Wellenvektors  $\mathbf{k}$ diskret
\begin{equation}
    k_i = \frac{2 \pi}{L} \, m_i \quad \text{mit} \quad i = x,y,z \quad . \label{eq:2:k_randbed} 
\end{equation}




\section*{Zustandsdichte}

Die Zustandsdichte im reziproken Raum ist konstant:
\begin{equation}
    D(k) dk = \frac{V}{(2 \pi)^3} dk \quad .
\end{equation}
Um sie als Funktion der Energie zu erhalten benötigen wir wie bei den Phononen die Gruppengeschwindigkeit 
\begin{equation}
    v_g = \frac{\partial \omega}{ \partial k} = \frac{\partial E}{\partial (\hbar k)} = \frac{\hbar k}{m}
\end{equation}
und erhalten\sidenote{Für 3 Dimensionen. Niedrigdimensionale Strukturen kommen später.} 
\begin{equation}
    D'(E) dE = \frac{V}{\hbar (2 \pi)^3} \, dE \, \int_{E = \text{const.}} \frac{d S_E}{v_g}  \quad .
\end{equation}
Jetzt müssen wir noch berücksichtigen, dass wir jeden Zustand nach dem Pauli-Prinzip mit zwei Elektronen unterschiedlichen Spins besetzen können:
\begin{equation}
    D(E) dE = 2 D'(E) dE =  \frac{(2m)^{3/2}}{2 \pi^2 \hbar^3} \, V \,  \sqrt{E} \, dE \quad .
\end{equation}


\begin{marginfigure}
    \inputtikz{\currfiledir fermi-gas}
    \caption{Dispersionsrelation $E(k)$ und Zustandsdichte $D(E)$ eines Fermi-Gases in 3 Dimensionen.}
\end{marginfigure}


\begin{questions} 
    \item Wie kann man den wurzelförmigen Verlauf der Zustandsdichte $D(E)$ verstehen?
\end{questions}


\section*{Fermi-Energie und Fermi-Kugel}

Elektronen sind Fermionen, haben einen halbzahligen Spin und unterliegen  dem Pauli-Prinzip und der Fermi-Dirac-Statistik. Im thermischen Gleichgewicht ist also jeder Zustand besetzt wie 
\begin{equation}
    f(E) = \frac{1}{e^{(E-\mu)/k_B T} + 1}
\end{equation}
mit dem chemischen Potential $\mu$. Die Fermi-Dirac-Verteilung ist (um das Pauli-Verbot zu erfüllen) maximal Eins. Wenn $f(E) \ll 1$, also $E- \mu \gg k_B T$, dann geht sie in die Boltzmann-Verteilung über.

\begin{marginfigure}
    \inputtikz{\currfiledir FD_stat}
    \caption{Fermi-Dirac-Statistik (fett) in Vergleich zur Bose-Einstein-Statistik (gestrichelt) und Boltzmann-Statistik (dünn).}
\end{marginfigure}

Das chemische Potential $\mu$ kommt aus der Ableitung beispielsweise der inneren Energie $U$ nach der Stoffmenge $n_i$. Bei mehreren Stoffen gibt es also mehrere $\mu_i$.
\begin{equation}
    \mu_i = \left( \frac{\partial U}{\partial n_i} \right)_{V,S,n_j \neq n_i}
\end{equation}
bzw.
\begin{equation}
    dU = T dS - p dV + \sum_i \mu_i \, d n_i \quad .
\end{equation}
Bei uns ist der Stoff natürlich die Elektronen, daher brauchen wir im folgenden kein Index an $\mu$.

Bei $E = \mu$ geht die Fermi-Dirac-Verteilungsfunktion immer durch $1/2$. Am absoluten Nullpunkt ($T=0$) ist sie konstant Eins für $E < \mu$ und konstant Null darüber. Wir bezeichnen als \emph{Fermi-Energie} $E_F$ die Energie, bis zu der alle Zustände lückenlos gefüllt sind. Das entspräche dem chemischen Potential, wenn letzteres nicht temperaturabhängig wäre. So definieren wir
\begin{equation}
    E_F = \mu (T = 0) \quad .
\end{equation}
Damit ist die Fermi-Energie \emph{nicht} temperaturabhängig. Später werden wir den Begriff 'Fermi-Niveau' einführen, der nur ein anderes Wort für chemisches Potential ist und damit temperaturabhängig.


Wir berechnen die Fermi-Energie $E_F$, indem wir bei $T=0$ nach und nach Elektronen in unseren Kasten einfüllen, also 
die Zustandsdichte $D(E)$ soweit aufintegrieren, bis wir $N$ Elektronen untergebracht haben. Die Elektronendichte $n$ ist also\sidenote{Analog kann man ein temperaturabhängiges chemisches Potential ausrechnen. Siehe \cite{Hunklinger2014} oder \cite{Gross_FK}.  }
\begin{equation}
    n = \frac{N}{V} = \int_0^{E_F} D(E) \, dE \quad .
\end{equation}
Damit erhalten wir
\begin{equation}
    E_F = \frac{\hbar^2}{2m} (3 \pi^2)^{2/3} \, n^{2/3} \quad .
\end{equation}
Alle Komposita mit 'Fermi-' sind entsprechend definiert. Der Fermi-Impuls $k_F$ ist einfach
\begin{equation}
    k_F = (3 \pi^2 \, n)^{1/3} \quad .
\end{equation}
Die \emph{Fermi-Kugel} ist die Kugel im reziproken Raum mit dem Radius $k_F$. Am absoluten Nullpunkt sind also alle Elektronen innerhalb dieser Kugel. Später werden wir Beispiele dafür sehen, dass die Form keine Kugel mehr ist, sondern durch eine mehr oder weniger komplexe \emph{Fermi-Fläche} eingeschlossen wird.


Die Fermi-Energie von den hier betrachteten Metallen liegt typischerweise im Bereich von einigen Elektronvolt und die Fermi-Temperatur damit bei einigen 10~000~K, weit jenseits der Schmelztemperatur. Für Elektronen im Festkörper besteht also kein so großer Unterschied zum absoluten Nullpunkt. Die Stufenfunktion der Fermi-Dirac-Verteilung wird etwas abgerundet. Wenn man es maßstabsgerecht zeichnen würde, dann könnte man aber bei Raumtemperatur keinen Unterschied erkennen.


\begin{questions} 
\item Woran liegt es, dass hier die Fermi-Fläche gerade eine Kugeloberfläche ist? Was ist notwendig, damit andere Formen  entstehen?
\item Wieviel Elektronen pro Atom muss ein Material ungefähr besitzen, damit die Fermi-Kugel den Rand der Brillouinzone berührt?
\item Was bedeutet 'Für Elektronen im Festkörper besteht also kein so großer Unterschied zum absoluten Nullpunkt' ?
\end{questions}



\section*{Wärmekapazität der freien Elektronen}

Analog zum Vorgehen bei den Phononen berechnen wir die Wärmekapazität der Elektronen als Ableitung der inneren Energie nach der Temperatur. Wir beginnen\sidenote{\cite{Hunklinger2014} folgend} mit der spezifischen inneren Energie $u$
\begin{equation}
    u = \frac{U}{V} = \int_0^\infty E \, D(E) \, f(E) \,  dE \quad .
\end{equation}
Am absoluten Nullpunkt läuft das Integral nur bis $E_F$ und $f(E)$ ist so einfach, dass wir es weglassen können
\begin{equation}
    u_0 = u(T=0) = \int_0^{E_F} E \, D(E) \, dE = \frac{3n}{5} \, k_B \, T_F \quad .
\end{equation}
Bei einem klassischen freien Gas von Teilchen hätten wir 
\begin{equation}
    u_\text{klassisch} = \frac{3 n}{2} \, k_B \, T \quad \text{und} \quad c_\text{klassisch} = \frac{3 n}{2} \, k_B  \quad .
\end{equation}
Weil $T_F \gg T$ ist die innere Energie eines freien Elektronengases sehr hoch, was am Ende ein Effekt des Pauli-Verbots ist. Wir müssen zu sehr hochenergetischen Zuständen ausweichen, um noch Elektronen zufügen zu können.

\begin{marginfigure}
    \inputtikz{\currfiledir fermi-dirac}
    \caption{Nur Zustände in der Nähe der Fermi-Energie tragen zur Wärmekapazität bei. Die graue Kurve ist um den Faktor 10 kühler.}
    \label{fig:2_fermi_dirac_T}
\end{marginfigure}


Die Ableitung $\partial u / \partial T$ ist aufwändig. Lehrbücher zeigen ein paar Schritte. Ich möchte das hier abkürzen und so argumentieren: ein freies Elektronengas ist quasi ein klassisches Gas, nur kann aufgrund der Fermi-Dirac-Statistik nur der Anteil $T/T_F$ weitere Energie aufnehmen und so zur Wärmekapazität beitragen. Zustände, die weiter von $E_F$ entfernt sind, sind entweder vollständig besetzt, so dass im Abstand $k_B T$ kein freier Zustand vorhanden ist, oder sie sind vollständig unbesetzt. Die Abschätzung ist also
\begin{equation}
    c_\text{geschätzt} = c_\text{klassisch}  \, \frac{T}{T_F} = \frac{3 n \, k_B}{2}  \, \frac{T}{T_F} \quad .
\end{equation}
Eine etwas bessere Rechnung ergibt einen um den Faktor $\pi^2/3$ größeren Wert
\begin{equation}
    c_\text{el} = \frac{\pi^2 \, n \, k_B}{2}  \, \frac{T}{T_F} = \gamma T \label{eq:2_WK_elek}
\end{equation}
mit der Sommerfeld-Konstanten $\gamma$.


Zusammen mit dem Debye-Modell für die Phononen können wir so nun auch die Wärmekapazität von Metallen beschreiben. Wir erhalten
\begin{equation}
    c_\text{ges} = \gamma \, T \, + \, 
        \left\{ 
        \begin{matrix}
            3 n_A k_B & \text{für} \quad T \gg \Theta \\
            \beta T^3 & \text{für} \quad T \ll \Theta 
        \end{matrix}
        \right.
\end{equation}
mit $\beta$ aus dem Debye-Modell und $n_A$ der Teilchenzahl-Dichte der Atom-Kerne.



\begin{marginfigure}
    \inputtikz{\currfiledir fig_Cu_WK}
    \caption{Wärmekapazität von Kupfer bei tiefen Temperaturen nach \cite{Rayne1956}. Elektronen und Phononen tragen bei. \label{fig:2_Cu_WK}}
\end{marginfigure}


Dieses Modell beschreibt die Wärmekapazität von Alkali-Metallen und anderen 'einfachen' Metallen gut (Abb.~\ref{fig:2_Cu_WK}). In anderen Fällen finden sich deutliche Abweichung, beispielsweise bei Nickel. Hier ist die gemessene Wärmekapazität um etwa den Faktor 15 höher als die wie oben berechnete. Bei den Alkali-Metallen ist die Annahme des freien Elektronengases gerechtfertigt. Bei Nickel tragen aber Elektronen zur Wärmekapazität bei, die aus atomaren d-Orbitalen stammen, daher eine Vorzugsrichtung haben und keine isotrope Zustandsdichte im Kristall besitzen. Dies führt zu einer hohen Zustandsdichte an der Fermi-Energie und so zu einer erhöhten Wärmekapazität.\sidenote{siehe \cite{Hunklinger2014}, Abbildung 8.9}


\begin{questions} 
    \item Wo zeigt sich in Abb.~\ref{fig:2_fermi_dirac_T} die Wärmekapazität?
\end{questions}
    
    



\section{Drude-Modell}

Lassen Sie uns zunächst das Drude-Modell besprechen. Dies liefert das Ohm'sche Gesetz, also ein richtiges Ergebnis, aber aus heutiger Sicht aus den falschen Gründen. Es wurde 1900 von Paul Drude eingeführt. Die Annahmen sind ein freies Elektronengas, das eine mittlere thermische Geschwindigkeit $v_{th}$ besitzt. Die Elektronen stoßen mit den Atomrümpfen. Das externe elektrische Feld $\mathcal{E}$ überlagert der thermischen Bewegung eine Driftbewegung mit der Geschwindigkeit $v_d$. Die Bewegungsgleichung ist 
\begin{equation}
   m \frac{d \mathbf{v}}{dt} = -e \, \mathcal{E} - m \frac{\mathbf{v}_d}{\tau} \quad .
\end{equation}
Der letzte Term ist eine Art Reibungskraft, die durch die Stöße der Elektronen entsteht. $\tau$ ist dabei die mittlere Zeit zwischen zwei Stößen. Im stationären Fall 
($d \mathbf{v} / dt = 0$) erhält man   
\begin{equation}
   \mathbf{v}_d = - \frac{e \tau}{m} \mathcal{E} = - \mu \mathcal{E} 
   \quad \text{mit} \quad
    \mu = \frac{| \mathbf{v}_d |}{|\mathcal{E}|} = \frac{e \tau}{m}
\end{equation}
mit der Beweglichkeit $\mu$. Die Stromdichte ist dann
\begin{equation}
   \mathbf{j} = -e n  \mathbf{v}_d = n e \mu \mathcal{E} = \sigma \mathcal{E} 
   \quad \text{mit} \quad 
   \sigma = n e \mu  = \frac{n e^2 \tau}{m}
\end{equation}
mit der Elektronendichte $n$ und der Leitfähigkeit $\sigma$. Damit haben wir den linearen Zusammenhang zwischen Strom und Spannung des Ohm'schen Gesetzes erhalten. Der makroskopische Widerstand (bzw. dessen reziproker Wert, die Leitfähigkeit $\sigma$) ist verknüpft mit zwei mikroskopischen  Größen, der Elektronendichte $n$ und der mittleren Stoßzeit $\tau$. Erstere ergibt sich aus der Zahl der Valenz-Elektronen pro Atom und der Gitterkonstanten des Kristalls. Letzte liegt wie oben schon erwähnt bei etwa 10~fs.

Dieses Modell ignoriert völlig das Pauli-Prinzip und dass es eine Fermi-Dirac-Verteilung gibt, bei der quasi alle Zustände besetzt sind. Die allermeisten Elektronen können gar nicht streuen, weil sie dazu einen leeren Endzustand bräuchten, den es nicht gibt. Wir werden aber sehen, dass ein besseres Modell das gleiche Ergebnis liefert.

\section{Drude-Sommerfeld-Modell}

Arnold Sommerfeld entwickelte eine verbesserte Theorie. Die Elektronen sind quasi frei, es gilt die Schrödinger-Gleichung und das Pauli-Prinzip. Wir machen aber die Annahme, dass die Fermi-Fläche eine Kugel ist, die Bandstruktur also isotrop und insbesondere das Band nur halb gefüllt, so dass die Fermi-Fläche weit von der Grenze der Brillouinzone entfernt ist.

Ohne externes Feld fliest im thermischen Gleichgewicht kein Strom, weil die Fermi-Kugel um $\mathbf{k} = 0$ zentriert ist. Weil wir Isotropie angenommen haben gibt es auch in nur teilweise gefüllten Bändern für jedes Elektron mit $\mathbf{k}$ eines mit  $-\mathbf{k}$.

Eine externes Feld $\bm{\mathcal{E}}$ bewegt  jedes Elektron und damit die gesamte Fermikugel kontinuierlich immer weiter von der Gleichgewichtsposition weg:
\begin{equation}
   \hbar \frac{d \mathbf{k}}{dt} = -e \bm{\mathcal{E}} = \mathbf{F} \quad .
\end{equation}
Streuprozesse können dann aber Elektronen von 'vorne' an der Fermikugel in den frei werdenden Bereich 'hinter' der Kugel umlagern. Die Stöße wirken also rückstellend auf die Bewegung der Fermikugel. Im sich einstellenden Gleichgewicht wird die Fermikugel bei einer mittleren Stoßzeit $\tau$  um 
\begin{equation}
 \delta k = \frac{-e \tau \mathcal{E}}{\hbar}     
\end{equation}  
aus dem Ursprung verschoben sein. Nur der Anteil $\delta k / k_F$ der Elektronen trägt zum Ladungstransport bei. Das sind aber die an der Fermi-Kante, also die schnellsten von allen. Für die Leitfähigkeit ergibt das Sommerfeld-Modell also
\begin{equation}
   \sigma = n e \mu  = \frac{n e^2 }{m^\star} \, \tau(E_F)
\end{equation}
mit $\tau(E_F)$ der Stoßzeit der Elektronen an der Fermi-Kante. Der Unterschied zum Drude-Modell besteht nur in einer etwas anderen Bedeutung zweier Parameter. Damit ist nicht überraschend, dass auch das Drude-Modell die experimentellen Ergebnisse richtig wiedergibt.

\begin{questions} 
    \item Erklären Sie, warum das Drude- und das Sommerfeld-Modell zum (scheinbar ?) gleichen Ergebnis kommen.
\end{questions}
    
    


\section{Temperaturabhängigkeit der elektrischen Leitfähigkeit}

Die elektrische Leitfähigkeit $\sigma$ ist temperaturabhängig über die mittlere Stoßzeit $\tau$. Elektronen können mit verschiedenen anderen Partnern streuen (stoßen): mit Phononen, mit Defekten und mit der Probenoberfläche. Dabei addieren sich die Streu-Raten, also die reziproken Stoß-Zeiten. Nur die Streuung an Phononen ist temperaturabhängig. Die anderen Effekte führen zu einem konstanten Wert, der bei tiefen Temperaturen erreicht wird, wenn keine Phononen besetzt sind.

Bereits in Kapitel 1 hatten wir die mittlere freie Weglänge definiert als (Gl. \ref{eq:1_def_weglaenge} )
\begin{equation}
   \ell = \frac{1}{n \, \sigma_{st}} = \tau \, v_F \quad ,
\end{equation}
wobei wir hier den Streuquerschnitt $\sigma_{st}$ genannt haben, um ihn von der Leitfähigkeit $\sigma$ zu unterscheiden. $\sigma_{st}$ ist für unsere Zwecke konstant. Elektronen bewegen sich mit der Fermi-Geschwindigkeit $v_F$, die so den Zusammenhang zwischen Stoßzeit und Weglänge herstellt.

Bei einer Temperatur $T$ (viel) größer als der Debye-Temperatur $\Theta$ ändert sich die Dichte $n$ der Phononen wie $T/\Theta$, so dass wir für die Leitfähigkeit $\sigma$ erhalten
\begin{equation}
   \sigma \propto \left\{ 
      \begin{matrix}
         \text{const} & \text{für} \quad T \ll \Theta \\
   \frac{1}{T} &  \text{für} \quad T \gg \Theta \\
      \end{matrix}
   \right.  \quad .
\end{equation}
Der Übergangsbereich ist wie immer aufwändig und durch das Bloch-Grüneisen-Gesetz beschrieben, das einen $T^{-5}$-Term liefert.\sidenote{Siehe z.B. \cite{Hunklinger2014}}


\section{Thermische Leitfähigkeit der Elektronen}

Abschließend wollen wir noch die thermische Leitfähigkeit der Elektronen diskutieren, nachdem die anderen Kombinationen aus Elektronen oder Phononen mit Wärmekapazität oder Wärmeleitfähigkeit schon früher besprochen wurden. Im täglichen Leben machen wir die Erfahrung, dass Metalle Wärme besser leiten als Isolatoren. Elektronen scheinen also einen hohen Beitrag zur Wärmeleitfähigkeit zu liefern.

Die  Wärmeleitfähigkeit der Phononen hatten wir in Gl. \ref{eq:2_def_waermeleitf} definiert. Hier gehen wir analog vor:
\begin{equation}
   K  = \frac{1}{3} \, C \, v \, \ell \quad , 
\end{equation}
wobei jetzt alle Größen als elektronische zu verstehen sind, also $K$ die elektronische Wärmeleitfähigkeit, $C$ deren Wärmekapazität, $v$ deren Geschwindigkeit und $\ell =  v_F \tau$ die mittlere freie Weglänge. Wir setzen Gl.~\ref{eq:2_WK_elek} für $C$ ein sowie die Fermi-Geschwindigkeit $v_F$ für $v$  und erhalten
\begin{equation}
   K  =  \frac{1}{3} \frac{\pi^2 \, n \, k_B}{2}  \, \frac{T}{T_F}  \, v_F \, \ell
   =     \frac{\pi^2 }{3} \frac{ n \, k_B^2 \tau}{m}  \, T 
\end{equation}
mit $T_F = m v_F^2 / (2 k_B)$. Die Temperaturabhängigkeit der Stoßzeit $\tau$ der Elektronen mit Phononen muss aber wie oben mit berücksichtigt werden. 


\section{Wiedemann-Franz-Gesetz}

Da in der Temperaturabhängigkeit der elektrischen und auch der thermischen Leitfähigkeit die Temperaturabhängigkeit der Elektron-Phonon-Streuung auftaucht, ist es nicht verwunderlich, dass beide Leitfähigkeiten miteinander in Beziehung stehen. Das ist das Wiedemann-Franz-Gesetz
\begin{equation}
   \frac{K_{el}}{\sigma} = \frac{\pi^2}{e} \, \left( \frac{k_B}{e} \right)^2 \, T = L \, T
\end{equation}
mit der universellen Lorenz-Zahl $L \approx 2.5 \cdot 10^{-8}$~$\Omega$WK$^{-2}$. Gute Wärmeleiter sind also auch gute elektrischer Leiter. In der Realität gewichten die beiden Transportprozesse die Streuung etwas unterschiedlich, so dass es zu Abweichungen bei mittleren Temperaturen kommt, siehe Abb.~\ref{fig:2_Cu_Lorenz}.

Das Drude-Modell sagt das Wiedemann-Franz-Gesetz richtig voraus. Dabei kompensieren sich allerdings der Fehler in der Wärmekapazität der Elektronen mit dem in ihrer Geschwindigkeit\sidenote{siehe \cite{Gross_FK}, Kap. 7.3.2.1}.



\newpage
\section{Zusammenfassung}

\textit{Schreiben Sie hier ihre persönliche Zusammenfassung des Kapitels auf. Konzentrieren Sie sich auf die wichtigsten Aspekte und die am Anfang genannten Ziele des Kapitels.}

 \vspace*{10cm}

\printbibliography[segment=\therefsegment,heading=subbibliography]
