% \documentclass{standalone}
% \usepackage{currfile,hyperxmp}

% \input{../tikz_header.tex}

% \begin{document}



\begin{tikzpicture}[
    declare function={coth(\x)  = 1/tanh(\x) ; 
    af(\J) = (2*\J +1 )/(2*\J);
    bf(\J) = 1/(2*\J);
                 },
]
%\useasboundingbox (-1.3,-1.2) rectangle (10.2,4.7);
%\draw (-1,-1) rectangle +(12,5);

    \begin{axis}[ xlabel={ $B/T$ (T/K)}, ylabel= {$\mu$ ($\mu_B$/Ion)}, 
         width=50mm, height=50mm, 
    %    ymode=log, 
    %    xmode=log,
           xmin = 0,
    %     xmax = 4.4,
         ymin  = 0,
         xmax = 4,
         ytick = {0,3,5,7},
         %xmax=5.5, ymin = 0, ymax=7.5,
        % axis x line=bottom,
         %axis y line=left,
         % xmax= 2e5, unbounded coords=jump, ymin=0, ymax = 4
        % label style={font=\tiny},
        % tick label style={font=\tiny}
       % ytick= \empty  
     %  legend pos= north west,
     %  legend style={draw=none, font=\footnotesize}
    ]


    % \addplot[no marks,   thin, gray, domain=0.7:70]{x}; 
     



\addplot[,only marks,  mark size = 1pt,
] table [ col sep=comma,
%x index = 0,
 y index = 1,
 x expr = \thisrowno{0} / 10,
] {\currfiledir data/Henry_paramag.csv};

\addplot[no marks,  mark size = 1pt,
] table [ col sep=comma,
x index = 0,
 y index = 1,
] {\currfiledir pluto/henry_paramag_model.csv};


\addplot[,only marks,  mark size = 1pt,
] table [ col sep=comma,
x expr = \thisrowno{0} / 10,
y index = 2,
] {\currfiledir data/Henry_paramag.csv};

\addplot[no marks,  mark size = 1pt,
] table [ col sep=comma,
x index = 0,
 y index = 2,
] {\currfiledir pluto/henry_paramag_model.csv};


\addplot[,only marks,  mark size = 1pt,
] table [ col sep=comma,
x expr = \thisrowno{0} / 10,
y index = 3,
] {\currfiledir data/Henry_paramag.csv};

\addplot[no marks,  mark size = 1pt,
] table [ col sep=comma,
x index = 0,
 y index = 3,
] {\currfiledir pluto/henry_paramag_model.csv};

\node[right, font=\tiny] at (3,2.5) {Gd$^{3+}$};
\node[right, font=\tiny] at (3,4.5) {Fe$^{3+}$};
\node[right, font=\tiny] at (3,6.5) {Cr$^{3+}$};

    \end{axis}
\end{tikzpicture}

%\end{document}