% \documentclass{standalone}
% \usepackage{currfile,hyperxmp}

% \usepackage{tikz,tikz-3dplot}

\DeclareUnicodeCharacter{0393}{$\Gamma$} 
\DeclareUnicodeCharacter{03C3}{$\sigma$} 


\newcommand{\inputtikz}[1]{%

 \tikzexternalenable
  \tikzsetnextfilename{#1}%
  \input{#1.tikz}%
  \tikzexternaldisable

}


\usetikzlibrary{math,matrix,fit,positioning,intersections}

\usetikzlibrary{calc}
\usetikzlibrary{arrows.meta} %needed tikz library

\usepackage{standalone}
\usepackage{pgfplots}
 \pgfplotsset{compat=newest}
\usepgfplotslibrary{groupplots}
\usepgfplotslibrary{fillbetween}

\tikzset{>=latex}

\usepackage{tikzorbital}
 \usepackage{tikzsymbols}
\usetikzlibrary{quotes,angles}

\usepackage{currfile,hyperxmp}


\pgfplotsset{
tufte line/.style={
    axis line style={draw opacity=0},
    ytick=\empty,
    axis x line*=bottom,
    x axis line style={
      draw opacity=1,
      gray,
      thick
},
 %   yticklabel=\pgfmathprintnumber{\tick}
  }
  }

\tikzset{
mymat/.style={
    matrix of math nodes,
    left delimiter=|, right delimiter=|,
    align=center,
    column sep=-\pgflinewidth,
}
%,mymats/.style={
%    mymat,
%    nodes={draw,fill=#1}
%} 
 }
 
\newcommand{\myarrow}[5]{\draw[#4](#1.south -| #2)  -- ++(#3 :6mm) node[above,pos=0.55]{$#5$};
} 

\newcommand{\interactLp}[3]{\myarrow{#1-#2-1}{#1.west}{-135}{<-}{#3}} 
\newcommand{\interactLm}[3]{\myarrow{#1-#2-1}{#1.west}{+135}{->}{#3}} 
\newcommand{\interactRp}[3]{\myarrow{#1-#2-2}{#1.east}{ -45}{<-}{#3}} 
\newcommand{\interactRm}[3]{\myarrow{#1-#2-2}{#1.east}{ +45}{->}{#3}}  

\newcommand{\interactout}[2]{\myarrow{#1-1-1}{#1.west}{+135}{->,dashed}{#2}} 


\newcommand{\benzene}[8]{%
\tikzmath{\x1 = #1; \dx1 = 0.5; \dx2 = 0.9; \ps=0.5;}
\tikzmath{\x2 = \x1 + \dx1 ;}
\tikzmath{\x3 = \x2 + \dx2 ;}
\tikzmath{\x4 = \x3 + \dx1 ;}

\tikzmath{\y1 = #2; \dy = 0.5;}
\tikzmath{\y2 = \y1 + \dy ;}
\tikzmath{\y3 = \y2 + \dy ;}

\orbital[pos = {(\x1,\y2)},scale=#3 * \ps]{pz}
\orbital[pos = {(\x2,\y1)},scale=#4 * \ps]{pz}
\orbital[pos = {(\x3,\y1)},scale=#5 * \ps]{pz}
\orbital[pos = {(\x4,\y2)},scale=#6 * \ps]{pz}
\orbital[pos = {(\x3,\y3)},scale=#7 * \ps]{pz}
\orbital[pos = {(\x2,\y3)},scale=#8 * \ps]{pz}

\draw (\x1,\y2) -- (\x2,\y1) -- (\x3,\y1) -- (\x4,\y2) --(\x3,\y3) 
-- (\x2,\y3) -- (\x1,\y2);
}

% \begin{document}



\begin{tikzpicture}
%\useasboundingbox (-1.3,-1.2) rectangle (10.2,4.7);
%\draw (-1,-1) rectangle +(12,5);

    \begin{axis}[ xlabel={Magnetfeld $B$  (T)}, ylabel={Widerstand  (k$\Omega$)},  width=110mm, height=55mm, 
        %xmode=log, ymode=log,
         xmin = 0,
         xmax = 7.5,
         ymax = 15,
         %xmax=5.5, ymin = 0, ymax=7.5,
        % axis x line=bottom,
         %axis y line=left,
         % xmax= 2e5, unbounded coords=jump, ymin=0, ymax = 4
        % label style={font=\tiny},
        % tick label style={font=\tiny}
       % ytick= \empty  
    ]


     \addplot[no marks,   thin, gray, domain=0:7]{25.8128 /2}; 
     \addplot[no marks,   thin, gray, domain=0:5]{25.8128 /3}; 
     \addplot[no marks,   thin, gray, domain=0:3.5]{25.8128 /4}; 
     \addplot[no marks,   thin, gray, domain=0:3.3]{25.8128 /5}; 
     \addplot[no marks,   thin, gray, domain=0:3]{25.8128 /6}; 
     \addplot[no marks,   thin, gray, domain=0:2.5]{25.8128 /7}; 
     \addplot[no marks,   thin, gray, domain=0:2]{25.8128 /8}; 
     \addplot[no marks,   thin, gray, domain=0:2]{25.8128 /9}; 
     \addplot[no marks,   thin, gray, domain=0:1.6]{25.8128 /10}; 
     \addplot[no marks,   thin, gray, domain=0:1.6]{25.8128 /11}; 
     \addplot[no marks,   thin, gray, domain=0:1.4]{25.8128 /12}; 
     \addplot[no marks,   thin, gray, domain=0:1.4]{25.8128 /13};   
     \addplot[no marks,   thin, gray, domain=0:1.2]{25.8128 /14}; 
     


     \addplot[no marks,color=black, thick] table [ col sep=comma,
     % expr=\thisrowno{0}/1000, % y expr=\thisrowno{1}*0.69, 
     x index = 0,
      y index = 1
     ] {\currfiledir data/qhe_xy.csv};


     \addplot[no marks,color=black,mark=o, mark size=1pt] table [ col sep=comma,
     % expr=\thisrowno{0}/1000, 
     y expr=\thisrowno{1}*0.01, 
     x index = 0,
     % y index = 1
     ] {\currfiledir data/qhe_xx.csv};


     \node at (axis cs:6.75, 11) {\footnotesize $\rho_{xy}$} ;
     \node at (axis cs:7, 5) {\footnotesize $\rho_{xx} \times 10$} ;
  
     \node[above left, align=right, yshift=-1mm, gray] at (axis cs:2, 25.8128 /2) {\footnotesize $p=2$} ;
     \node[above left, align=right, yshift=-1mm,gray] at (axis cs:2, 25.8128 /3) {\footnotesize $3$} ;
     \node[above left, align=right, yshift=-1mm,gray] at (axis cs:2, 25.8128 /4) {\footnotesize $4$} ;
     \node[above left, align=right, yshift=-1mm,gray] at (axis cs:2, 25.8128 /5) {\footnotesize $5$} ;
  
    
    %  \draw  (axis cs:0.0705, 0) -- (axis cs:0.0705, 3); 
    %  \draw  (axis cs:0.194312, 0) -- (axis cs:0.194312, 3); 
    %  \draw  (axis cs:0.15061, 0) -- (axis cs:0.15061, 3); 

     % \draw [|-|] (axis cs:6.1, 4) -- node[above] {$\frac{ e }{h \, S_F}$ } (axis cs: 7.1,4);



    \end{axis}
\end{tikzpicture}

%\end{document}