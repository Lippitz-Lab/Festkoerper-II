% \documentclass{standalone}
% \usepackage{currfile,hyperxmp}

% \input{../tikz_header.tex}

% \begin{document}



\begin{tikzpicture}
%\useasboundingbox (-1.3,-1.2) rectangle (10.2,4.7);
%\draw (-1,-1) rectangle +(12,5);

    \begin{axis}[ xlabel={Magnetfeld $B$  (T)}, ylabel={Widerstand  (k$\Omega$)},  width=110mm, height=55mm, 
        %xmode=log, ymode=log,
         xmin = 0,
         xmax = 7.5,
         ymax = 15,
         %xmax=5.5, ymin = 0, ymax=7.5,
        % axis x line=bottom,
         %axis y line=left,
         % xmax= 2e5, unbounded coords=jump, ymin=0, ymax = 4
        % label style={font=\tiny},
        % tick label style={font=\tiny}
       % ytick= \empty  
    ]


     \addplot[no marks,   thin, gray, domain=0:7]{25.8128 /2}; 
     \addplot[no marks,   thin, gray, domain=0:5]{25.8128 /3}; 
     \addplot[no marks,   thin, gray, domain=0:3.5]{25.8128 /4}; 
     \addplot[no marks,   thin, gray, domain=0:3.3]{25.8128 /5}; 
     \addplot[no marks,   thin, gray, domain=0:3]{25.8128 /6}; 
     \addplot[no marks,   thin, gray, domain=0:2.5]{25.8128 /7}; 
     \addplot[no marks,   thin, gray, domain=0:2]{25.8128 /8}; 
     \addplot[no marks,   thin, gray, domain=0:2]{25.8128 /9}; 
     \addplot[no marks,   thin, gray, domain=0:1.6]{25.8128 /10}; 
     \addplot[no marks,   thin, gray, domain=0:1.6]{25.8128 /11}; 
     \addplot[no marks,   thin, gray, domain=0:1.4]{25.8128 /12}; 
     \addplot[no marks,   thin, gray, domain=0:1.4]{25.8128 /13};   
     \addplot[no marks,   thin, gray, domain=0:1.2]{25.8128 /14}; 
     


     \addplot[no marks,color=black, thick] table [ col sep=comma,
     % expr=\thisrowno{0}/1000, % y expr=\thisrowno{1}*0.69, 
     x index = 0,
      y index = 1
     ] {\currfiledir data/qhe_xy.csv};


     \addplot[no marks,color=black,mark=o, mark size=1pt] table [ col sep=comma,
     % expr=\thisrowno{0}/1000, 
     y expr=\thisrowno{1}*0.01, 
     x index = 0,
     % y index = 1
     ] {\currfiledir data/qhe_xx.csv};


     \node at (axis cs:6.75, 11) {\footnotesize $\rho_{xy}$} ;
     \node at (axis cs:7, 5) {\footnotesize $\rho_{xx} \times 10$} ;
  
     \node[above left, align=right, yshift=-1mm, gray] at (axis cs:2, 25.8128 /2) {\footnotesize $p=2$} ;
     \node[above left, align=right, yshift=-1mm,gray] at (axis cs:2, 25.8128 /3) {\footnotesize $3$} ;
     \node[above left, align=right, yshift=-1mm,gray] at (axis cs:2, 25.8128 /4) {\footnotesize $4$} ;
     \node[above left, align=right, yshift=-1mm,gray] at (axis cs:2, 25.8128 /5) {\footnotesize $5$} ;
  
    
    %  \draw  (axis cs:0.0705, 0) -- (axis cs:0.0705, 3); 
    %  \draw  (axis cs:0.194312, 0) -- (axis cs:0.194312, 3); 
    %  \draw  (axis cs:0.15061, 0) -- (axis cs:0.15061, 3); 

     % \draw [|-|] (axis cs:6.1, 4) -- node[above] {$\frac{ e }{h \, S_F}$ } (axis cs: 7.1,4);



    \end{axis}
\end{tikzpicture}

%\end{document}