% \documentclass{standalone}
% \usepackage{currfile,hyperxmp}

% \input{../tikz_header.tex}

% \begin{document}



\begin{tikzpicture}
%\useasboundingbox (-1.3,-1.2) rectangle (10.2,4.7);
%\draw (-1,-1) rectangle +(12,5);

    \begin{axis}[ xlabel={Gate-Spannung $U_g$  (V) }, ylabel= {Leitfähigkeit $G$ ($e^2 /2 h$)}, 
         width=50mm, height=50mm, 
        %ymode=log, 
     %   ymode=log,
      %   xmin = 0,
      %  xmax = 55,
    %     xmax = 4.4,
        ymin  = 0,
       %  ymax = 8,
         %xmax=5.5, ymin = 0, ymax=7.5,
       %  axis x line=bottom,
       %  axis y line=left,
         % xmax= 2e5, unbounded coords=jump, ymin=0, ymax = 4
        % label style={font=\tiny},
        % tick label style={font=\tiny}
       % ytick= \empty ,
       % xtick= \empty, 
     %  legend pos= north west,
     %  legend style={draw=none, font=\footnotesize}
    %clip = false,
    ]

    %\fill[fill=gray!20!white] (0,0) rectangle (3,7);


    % para     
    \addplot[smooth, thick] table
   [col sep=comma,  
   %x expr = -(\thisrowno{0}),  
   x  index = 0,
   y index = 1] 
    {\currfiledir data/van_wees_cond_quant.csv};


    \draw[dotted] (-3,1) -- ++(3,0);
    \draw[dotted] (-3,2) -- ++(3,0);
    \draw[dotted] (-3,3) -- ++(3,0);
    \draw[dotted] (-3,4) -- ++(3,0);
    \draw[dotted] (-3,5) -- ++(3,0);
    \draw[dotted] (-3,6) -- ++(3,0);
    \draw[dotted] (-3,7) -- ++(3,0);
    \draw[dotted] (-3,8) -- ++(3,0);
    \draw[dotted] (-3,9) -- ++(3,0);
    \draw[dotted] (-3,10) -- ++(3,0);
    \draw[dotted] (-3,11) -- ++(3,0);


 
  %\addplot[dashed, domain=0:2.5, samples = 100] {25 *exp(- x^2 /2 )};

% \addplot[ mark=o, mark size = 1pt,  thick
% ] table [ col sep=comma,  
% x index = 0,
%  y expr = (\thisrowno{1} -10)  ,
% ] {\currfiledir data/MnO_Strauser_Wollan_80K.csv};

    %\draw[dashed] (1,-0.02) -- (1, 2.);   
    \node[right] at (15.3, 7) {$\omega_P$}; 
    \node[] at (8, 5.5) {$\omega_{SP}$}; 



    \end{axis}
\end{tikzpicture}

%\end{document}